% -*- coding: utf-8 -*-

\input macros

%\beginchapter Appendix F. Font Tables
\beginchapter Appendix F. 字体表

%The purpose of this appendix is to summarize the chief characteristics of
%the ^{Computer Modern} typefaces.  \TeX\ is able to typeset
%documents with any fonts, having any arrangement of characters; the
%fonts and layouts to be described here are the particular ones that
%correspond to plain~\TeX\ format, i.e., to the macros in Appendix~B\null.
%\ (Complete information about the Computer Modern family, including the
%\MF\ programs that draw the characters, can be found in the author's book
%{\sl Computer Modern Typefaces}.)
%^^{METAFONT}
这个附录的目的是总结^{计算机现代}字体族的主要特征。
\TeX\ 能够用任何字体排版文档,不管字体中的字符怎么排列;
这里描述的字体和布局对应于 plain~\TeX\ 格式,即对应于附录~B 中的宏。%
(关于计算机现代字体族的完整信息,包括绘制字符的 \MF\ 程序,
可以在作者的书籍{\sl Computer Modern Typefaces}中找到。)
^^{METAFONT}

%The first pages of this appendix show what the fonts contain; the last
%pages show what the symbols are called when they're used in math formulas.
%\ (See Appendix~B for the conventions that apply in non-mathematical text.)
这个附录的前几页显示各字体包含哪些字符;而后几页显示这些符号用于数学公式时的名称。%
(见附录~B 中用于非数学文本时的方法。)

%There are exactly 128 different characters in each of the Computer
%Modern fonts, although \TeX\ can work up to~256 characters per font.
%The text fonts are laid out as shown in the table below, which illustrates
%font |cmr10| (Computer Modern Roman 10~point). Thus, for example,
%if you ask for ^|\char||'35| when ^|cmr10| is the current font, you
%get the symbol \char'35. These text fonts include the ^{ligatures} and
%^{accents} described in Chapter~9; each symbol that happens to be
%a visible ASCII character appears in its ASCII position.
% Some of the ASCII symbols (namely
%|"|~|<|~|>|~|\|~|_|~|{|~\|~|}|) are not included because
%they don't occur in normal printer's fonts. If you mistakenly
%type |"|, you get "; and |<| outside of math mode yields <\thinspace!
%\ Incidentally, the ten ^{digits} all have width |0.5em|.
尽管在 \TeX\ 中每个字体最多可以有 256 个字符,
每个计算机现代字体都只包含 128 个不同字符。
文本字体的布局如下面表格所示,以 |cmr10| 字体(10~点计算机现代罗马字体)为例。
因此,在当前字体为 ^|cmr10|时,键入 ^|\char||'35| 将得到符号 \char'35。
这些文本字体包含第~9~章所描述的^{连写}和^{重音符} ;
每个 ASCII 可见字符都正好出现在它的 ASCII 位置。
有些 ASCII 符号(即 |"|~|<|~|>|~|\|~|_|~|{|~\|~|}|)不在其中,
因为它们并不包含在平常的打印机字体中。
如果错误地键入 |"|,你将得到~";而在数学模式外键入 |<| 将得到 <~!
顺便提一下,十个^{数字}的宽度都为 |0.5em|。

\medskip\vfill
\noindent {\bf Figure 1.\enspace}Text font layout,
showing |cmr10| (^|\rm|, |\textfont0|).
\beginchart\tenrm
\normalchart
\endchart

%\noindent Plain \TeX\ makes use of sixteen basic fonts:
%$$\halign{\indent$\nulldelimiterspace=0pt
%  \left.\ninepoint\vcenter{#}\,\hfil\right\}$ &#\hfil\cr
%\halign{\strut\tt cm# \ \hfil&(Computer Modern #)\hfil&\hidebrace{#}\hfil\cr
%r10&Roman 10 point\cr
%r7&Roman 7 point\cr
%r5&Roman 5 point\cr
%bx10&Bold Extended 10 point\cr
%bx7&Bold Extended 7 point\cr
%bx5&Bold Extended 5 point\cr
%sl10&Slanted Roman 10 point\cr
%ti10&Text Italic 10 point\cr}&text\cr
%\halign{\strut\tt cm# \ \hfil&(Computer Modern #)\hfil&\hidebrace{#}\hfil\cr
%tt10&Typewriter Type 10 point\cr
%mi10&Math Italic 10 point\cr
%mi7&Math Italic 7 point\cr
%mi5&Math Italic 5 point\cr
%sy10&Math Symbols 10 point\cr
%sy7&Math Symbols 7 point\cr
%sy5&Math Symbols 5 point\cr
%ex10&Math Extension 10 point\cr}&special\cr}$$
%The first eight of these all have essentially the same layout;
%but |cmr5| needs no ligatures, and many of the symbols of |cmti10|
%have different shapes.
%For example, the ^{ampersand} becomes an `^{E.T.}', and the
%^{dollar} changes to ^{pound} ^{sterling}:
\noindent Plain \TeX\ 使用了十六种基本字体:
$$\halign{\indent$\nulldelimiterspace=0pt
  \left.\ninepoint\baselineskip=11pt\vcenter{#}\,\hfil\right\}$ &#\hfil\cr
\halign{\strut\tt cm# \ \hfil&(Computer Modern #)\hfil&\hidebrace{#}\hfil\cr
r10&Roman 10 point\cr
r7&Roman 7 point\cr
r5&Roman 5 point\cr
bx10&Bold Extended 10 point\cr
bx7&Bold Extended 7 point\cr
bx5&Bold Extended 5 point\cr
sl10&Slanted Roman 10 point\cr
ti10&Text Italic 10 point\cr}&text\cr
\halign{\strut\tt cm# \ \hfil&(Computer Modern #)\hfil&\hidebrace{#}\hfil\cr
tt10&Typewriter Type 10 point\cr
mi10&Math Italic 10 point\cr
mi7&Math Italic 7 point\cr
mi5&Math Italic 5 point\cr
sy10&Math Symbols 10 point\cr
sy7&Math Symbols 7 point\cr
sy5&Math Symbols 5 point\cr
ex10&Math Extension 10 point\cr}&special\cr}$$
前面八个字体有着本质上相同的布局;
只是 |cmr5| 不含连写,且 |cmti10| 的很多符号的形状不同。
例如,|&| 变成了 `^{E.T.}',而 ^{美元符}变成^{英镑符}:

\medskip\vfill
\noindent {\bf Figure 2.\enspace}Text font layout,
showing |cmti10| (^|\it|).
\beginchart\tenit
\normalchart
\endchart

%\noindent The ^{typewriter font} ^|cmtt10| is almost like the fonts for
%ordinary text, but it includes all of the visible ASCII characters, in
%their correct positions. It also has vertical arrows \up\ and
%\dn, as well as an undirected single quote mark, {\tt\char'15}.
%Fourteen of the~128 positions are changed from the normal text layout
%conventions,
%namely codes \oct{013}--\oct{017}, \oct{040}, \oct{042}, \oct{074},
%\oct{076}, \oct{134}, \oct{137}, and \oct{173}--\oct{175}. All of the
%ligatures are absent, except for the Spanish |!`| and |?`|. \ (The
%characters for Spanish
%ligatures appear in different positions, but that makes no difference
%to the user, because each font tells \TeX\ where to locate its own
%ligatures.) \ The Polish \l, the d\.ot accent, and the
%long Hung\H arian umlaut have disappeared to make room for new symbols.
%In a sense, positions \oct{052} and \oct{055} also differ from the normal
%text conventions: The asterisk is not up as high as usual, and the hyphen is
%just like a minus sign.
\noindent ^{打字机字体} ^|cmtt10| 和普通文本字体几乎一样,
但它在正确位置包含所有可见 ASCII 字符。
它还包含竖直箭头 \up\ 和 \dn ,以及无向单引号~{\tt\char'15}。
相比普通文本字体,128 个位置中的其中 14 个被改变了,即编码
\oct{013}--\oct{017}、\oct{040}、\oct{042}、\oct{074}、
\oct{076}、\oct{134}、\oct{137} 以及 \oct{173}--\oct{175}。
除了西班牙语的 |!`| 和 |?`|\thinspace ,所有的连写都不见了。%
(西班牙语的连写字符出现在不同位置,但是这对用户来说没有区别,
因为每个字体都会告诉 \TeX\ 如何找到它的连写符。)波兰语 \l 、
上点重音符(d\.ot)以及匈牙利分音符(Hung\H arian)都被去掉以给新符号留出空间。
从某种意义上说,位置 \oct{052} 和 \oct{055} 也与普通文本的不同;
星号没有原来那么高,而连字符很像减号。

%Each character in |cmtt10| has the same width, namely |0.5em|;
%the spaces between words also have this width, and they will not
%stretch or shrink. \TeX\ puts two spaces at the end of each sentence
%when you are typesetting with a typewriter font. \ (These spacing
%conventions can be changed by assigning nonzero values to
%^|\spaceskip| and ^|\xspaceskip|; or you can assign new values to the
%|\fontdimen| parameters, which will be described shortly.)
在 |cmtt10| 里面每个字符都有相同的宽度,即 |0.5em|;
单词之间的空格也是这么宽,而且不能伸缩。
在用打字机字体排版时,\TeX\ 在每个句子之后留下两个空格。%
(要改变这些空格的大小,你可以给 ^|\spaceskip| 和 ^|\xspaceskip| 指定非零值;
或者给 |\fontdimen| 参数指定一个新值,这会在稍后说到。)

\medskip\vfill
\noindent {\bf Figure 3.\enspace}Typewriter text font layout,
showing |cmtt10| (^|\tt|).
\beginchart\tentt
\normalchart
\endchart

%\noindent You can see at a glance that the math italic font, ^|cmmi10|,
%is quite different from text italic. It contains lowercase Greek letters
%as well as uppercase ones; this, of course, is mathematicians' ^{Greek},
%not a text font that would be suitable for typesetting classical
%Greek literature. And if you look closely at the non-Greek italic
%letters, you will find that their proportions and spacing have been
%changed from |cmti10| to make them work better in \TeX's mathematics mode.
\noindent 看一眼你便知道数学意大利字体,^|cmmi10|,与文本意大利字体很不相同。
它除了包含大写希腊字母外,还包含小写希腊字母;
当然,这是数学家的^{希腊字母},而不是适合排版古典希腊文的文本希腊字母。
而如果你仔细观察非希腊意大利体字母,你会发现它们的比例和间距改自 |cmti10|,
以让它们在 \TeX\ 的数学模式中有更好的效果。

%Some special unslanted characters appear in positions \oct{050}--\oct{077}
%and \oct{133}--\oct{137}, including ``^{oldstyle numerals}'':
%`|$\mit1984$|' and `|$\oldstyle1984$|' both yield `$\mit1984$'. Some of the
%characters are intended to be combined with others; for example, \oct{054}
%forms the first part of the symbol `$\hookrightarrow$'. \ (See the definition
%of\/ ^|\hookrightarrow| in Appendix~B\null.) \ This portion of the font doesn't
%deserve the name math italic; it's really a resting place for characters
%that don't fit anywhere else. \ (The author didn't want to leave any places
%unfilled, since that would tempt people to create incompatible
%ways to fill them.)
在位置 \oct{050}--\oct{077} 和 \oct{133}--\oct{137} 有一些特殊的非倾斜字符,
其中包括 ``^{老式数字}'':`|$\mit1984$|' 和 `|$\oldstyle1984$|' 都能得到 `$\mit1984$'。
有些字符是要和其他字符组合使用的;例如,\oct{054} 组成 `$\hookrightarrow$'
符号的前面部分。(见附录~B 中 ^|\hookrightarrow| 的定义。)%
字体这些位置与数学意大利体的名称不符;它们只是给不适合放在别处的字符提供了空间。%
(作者不想留下任何未填充的位置,因为这将诱使人们用互不兼容的方式填充它们。)

%Plain \TeX\ takes its ^{comma}, ^{period}, and ^{slash} from |cmmi10| in
%math mode, so that appropriate kerning will be computed in certain
%formulas that would otherwise be spaced poorly. For the correct
%positioning of math accents with this font, you should set its
%^|\skewchar| to \oct{177}.
在 Plain \TeX\ 中,数学模式的^{逗号}、^{句号}和^{斜线号}都取自 |cmmi10|,
这样在某些公式中可以计算出适当的紧排,以免它们的间距变得很难看。
用这个字体时,要得到正确的数学重音符,你应该设定 ^|\skewchar| 为 \oct{177}。

\medskip\vfill
\noindent {\bf Figure 4.\enspace}Math italic font layout,
showing |cmmi10| (^|\mit|, |\textfont1|).
\beginchart\teni
\normalchart
\endchart

%\noindent When \TeX\ typesets mathematics it assumes that family~0 contains
%normal roman fonts and that families 1, 2, and~3 contain math italic,
%math symbol, and math extension fonts. The special characters in these
%fonts are usually given symbolic names by a |\mathchardef|
%instruction, which assigns a hexadecimal code to the symbol.
%This code has four digits, where the first tells what kind of symbol
%is involved, the second specifies the family, and the other two give the font
%position. For example,
%\begintt
%\mathchardef\ll="321C
%\endtt
%says that ^|\ll| is character \hex{1C} of the math symbol font (family~2),
%and that it's a ``relation'' (class~3). A complete list of the symbolic names
%provided by the plain \TeX\ format appears later in this appendix.
\noindent 在排版数学公式时,\TeX\ 假定第~0~族包含普通罗马字体,
而第~2、3、4~族分别包含数学意大利字体、数学符号字体和数学扩展字体。
这些字体中的特殊字符通常会用 |\mathchardef| 命令给出特殊的符号名称。
该命令给符号指定一个十六进制编码。这个编码有四位,
其中第一位表示该符号的类型、第二位表示它所属的族,而后两位给出它的字体位置。
例如:
\begintt
\mathchardef\ll="321C
\endtt
说明 ^|\ll| 是数学符号字体(第~2~族)的位置 \hex{1C} 的字符,
它是一个``关系符号''(第~3~类)。
Plain \TeX\ 格式定义的符号名称的完整列表出现在这个附录的后面部分。

%\smallskip
%Font ^|cmsy10| is plain \TeX's math symbol font, and it contains 128
%symbols laid out as shown below. Its ^|\skewchar| should be set to
%\oct{060} so that math accents will be positioned properly over the
%^{calligraphic capital letters}.
\smallskip
字体 ^|cmsy10| 是 plain \TeX\ 的数学符号字体,它包含 128 个符号,
字体布局如同按照下表所示。为了将数学重音放在^{手写大写字母}上边的合适位置,
这个字体的 ^|\skewchar| 应该设定为 \oct{060}。

\medskip\vfill
\noindent {\bf Figure 5.\enspace}Math symbol font layout,
showing |cmsy10| (^|\cal|, |\textfont2|).
\beginchart\tensy
\normalchart
\endchart

%The final font of plain \TeX\ is ^|cmex10|, which includes large symbols
%and pieces that can be used to build even larger ones. For example,
%arbitrarily large left parentheses can be constructed by putting
%\oct{060} at the top and \oct{100} at the bottom,
%and by using as many copies of \oct{102} as necessary in the middle.
%Large ^{square root signs} are made from \oct{164}, \oct{165}, and
%\oct{166}; large left braces have four component parts:
%\oct{070}, \oct{072}, \oct{074}, \oct{076}.
Plain \TeX\ 的最后一个字体是 ^|cmex10|,它包含大型符号以及用于构造更大符号的组件。
例如,将 \oct{060} 放在顶部,\oct{100} 放在底部,
中间接上多个 \oct{102},可以构造出任意大的左圆括号;
利用 \oct{164}、\oct{165} 和 \oct{166} 可以构造出大型的^{二次根号};
利用 \oct{070}、\oct{072}、\oct{074} 和 \oct{076} 可以构造出大型的左花括号。

\medskip\vfill
\noindent {\bf Figure 6.\enspace}^{Math extension font} layout,
showing |cmex10| (|\textfont3|).
\beginchart\tenex
\normalchart
\endchart

\noindent When \TeX\ ``loads'' a font into its memory, it doesn't look at
the actual shapes of the characters; it only loads the {\sl ^{font metric}\/}
information (e.g., |cmr10.tfm|), which includes the heights, widths,
depths, and italic corrections, together with information about ligatures
and kerning.  Furthermore, the metric information that comes with a font
like |cmex10| tells \TeX\ that certain characters form a series; for
example, all of the left parentheses are linked together in order of
increasing size: \oct{000}, \oct{020}, \oct{022}, and \oct{040}, followed
by the extensible left parenthesis, which is $\oct{060}+[\oct{102}]^n+
\oct{100}$. Similarly, the two ^{summation signs} (\oct{120}, \oct{130})
and the three ^|\widehat| accents (\oct{142}, \oct{143}, \oct{144}) are
linked together.  Appendix~G explains how \TeX\ goes about choosing
particular sizes for math delimiters, math operators, and math accents.

\smallskip
Each font also has at least seven ^|\fontdimen| parameters, which have the
following significance and typical values (rounded to two decimal places):
$$\halign to\hsize{#\tabskip0pt plus 10pt&#\hfil&
$\hfil#\pt$&
$\hfil#\pt$&
$\hfil#\pt$&
$\hfil#\pt$&
$\hfil#\pt$&
$\hfil#\pt$\tabskip0pt\cr
\hidewidth\#\hidewidth&Meaning&
\omit\hidewidth Value in |cmr10| &
\omit\hfil|cmbx10| &
\omit\hfil|cmsl10| &
\omit\hfil|cmti10| &
\omit\hfil|cmtt10| &
\omit\hfil|cmmi10| \cr
\noalign{\vskip2pt}
1&slant per pt&0.00&0.00&0.17&0.25&0.00&0.25\cr
2&^{interword space}&3.33&3.83&3.33&3.58&5.25&0.00\cr
3&interword stretch&1.67&1.92&1.67&1.53&0.00&0.00\cr
4&interword shrink&1.11&1.28&1.11&1.02&0.00&0.00\cr
5&^{x-height}&4.31&4.44&4.31&4.31&4.31&4.31\cr
6&^{quad} width&10.00&11.50&10.00&10.22&10.50&10.00\cr
7&extra space&1.11&1.28&1.11&1.02&5.25&0.00\cr}$$
The ^{slant} parameter is used to position accents;
the next three parameters define interword spaces when text is being typeset;
the next two define the font-oriented dimensions |1ex| and |1em|;
^^|ex| ^^|em|
and the last is the additional amount that is added to interword spaces
at the end of sentences
(i.e., when ^|\spacefactor| is 2000 or~more and ^|\xspaceskip| is zero).
When a font is ^{magnified} (using `^|at|' or `^|scaled|'), all of the
parameters except the slant are subject to magnification at the time the
font is loaded into \TeX's memory.

Notice that |cmmi10| has zero spacing. This is the mark of a font
that is intended only for mathematical typesetting; ^^{math fonts}
the rules in Appendix~G state that the italic correction is added
between adjacent characters from such fonts.

Math symbol fonts (i.e., fonts in family 2) are required to have at least
22~^|\fontdimen| parameters instead of the usual seven; similarly, math
extension fonts must have at least~13. The significance of these
additional parameters is explained in Appendix~G\null. If you want to increase
the number of parameters past the number that actually appear in a font's
metric information file, you can assign new values immediately after that font
has been loaded.  For example, if some font |\ff| with seven parameters
has just entered \TeX's memory, the command |\fontdimen13\ff=5pt| will set
parameter number~13 to $5\pt$; the intervening parameters, numbers 8--12,
will be set to zero. You can even give more than seven parameters to
^|\nullfont|, provided that you assign the values before any actual fonts
have been loaded.

\goodbreak
Now that the font layouts have all been displayed, it's time to
consider the names of the various mathematical symbols.
^^{symbols in math, table}
Plain \TeX\ defines more than 200 control sequences by which you can refer
to math symbols without having to find their numerical positions in the
layouts. It's generally best to call a symbol by its name, for then you can
easily adapt your manuscripts to other fonts, and your manuscript will
be much more readable.

The symbols divide naturally into groups based on their mathematical
class (Ord, Op, Bin, Rel, Open, Close, or Punct), so we shall follow
that order as we discuss them. N.B.: Unless otherwise stated, math symbols
are available only in math modes. For example, if you say `|\alpha|' in
horizontal mode, \TeX\ will report an error and try to insert a |$| sign.

\def\beginsymbols{$$\displayindent=16pt
  \halign\bgroup
  &\qquad\hbox to10pt{\hss$##$\hss}\enspace&\hbox to80pt{##\hss}\cr}
\let\endsymbols=\enddisplay

%\subsection Lowercase Greek letters.
\subsection 小写希腊字母.
^^|\alpha|^^|\iota|^^|\varrho|
^^|\beta|^^|\kappa|^^|\sigma|
^^|\gamma|^^|\lambda|^^|\varsigma|
^^|\delta|^^|\mu|^^|\tau|
^^|\epsilon|^^|\nu|^^|\upsilon|
^^|\varepsilon|^^|\xi|^^|\phi|
^^|\zeta|^^|\varphi|
^^|\eta|^^|\pi|^^|\chi|
^^|\theta|^^|\varpi|^^|\psi|
^^|\vartheta|^^|\rho|^^|\omega|
\beginsymbols
\alpha&|\alpha|&\iota&|\iota|&\varrho&|\varrho|\cr
\beta&|\beta|&\kappa&|\kappa|&\sigma&|\sigma|\cr
\gamma&|\gamma|&\lambda&|\lambda|&\varsigma&|\varsigma|\cr
\delta&|\delta|&\mu&|\mu|&\tau&|\tau|\cr
\epsilon&|\epsilon|&\nu&|\nu|&\upsilon&|\upsilon|\cr
\varepsilon&|\varepsilon|&\xi&|\xi|&\phi&|\phi|\cr
\zeta&|\zeta|&o&|o|&\varphi&|\varphi|\cr
\eta&|\eta|&\pi&|\pi|&\chi&|\chi|\cr
\theta&|\theta|&\varpi&|\varpi|&\psi&|\psi|\cr
\vartheta&|\vartheta|&\rho&|\rho|&\omega&|\omega|\cr
\endsymbols
There's no ^|\omicron|, because it would look the same as |o|. Notice
that the letter |\upsilon|~($\upsilon$) is a bit wider than |v|~($v$); both
of them should be distinguished from |\nu|~($\nu$). Similarly,
|\varsigma|~($\varsigma$) should not be confused with |\zeta|~($\zeta$).
It turns out that |\varsigma| and |\upsilon| are almost never used in
math formulas; they are included in plain \TeX\ primarily because they are
sometimes needed in short Greek citations (cf.~Appendix~J).

%\subsection Uppercase Greek letters.
\subsection 大写希腊字母.
^^|\Gamma|^^|\Xi|^^|\Phi|
^^|\Delta|^^|\Pi|^^|\Psi|
^^|\Theta|^^|\Sigma|^^|\Omega|
^^|\Lambda|^^|\Upsilon|
\beginsymbols
\Gamma&|\Gamma|&\Xi&|\Xi|&\Phi&|\Phi|\cr
\Delta&|\Delta|&\Pi&|\Pi|&\Psi&|\Psi|\cr
\Theta&|\Theta|&\Sigma&|\Sigma|&\Omega&|\Omega|\cr
\Lambda&|\Lambda|&\Upsilon&|\Upsilon|\cr
\endsymbols
The other Greek capitals appear in the roman alphabet
(^|\Alpha|${}\equiv{}$|{\rm A}|, ^|\Beta|${}\equiv{}$|{\rm B}|, etc.).
It's conventional to use unslanted letters for uppercase Greek, and
slanted letters for lowercase Greek; but you can obtain
$(\mit\Gamma,\Delta,\ldots,\Omega)$ by typing
|$({\mit\Gamma},| |{\mit\Delta},| |\ldots,| |{\mit\Omega})$|. ^^|\mit|

%\subsection Calligraphic capitals. To get the letters ^^{calligraphic letters}
%$\cal A\ldots Z$ that appear in Figure~5, type |${\cal A}\ldots{\cal Z}$|.
%Several other alphabets are also used with mathematics (notably ^{Fraktur},
%^{script}, and ``^{blackboard bold}''); they don't come with plain \TeX,
%but more elaborate formats like \AmSTeX\ do provide them.
\subsection 手写大写字母. To get the letters ^^{calligraphic letters}
$\cal A\ldots Z$ that appear in Figure~5, type |${\cal A}\ldots{\cal Z}$|.
Several other alphabets are also used with mathematics (notably ^{Fraktur},
^{script}, and ``^{blackboard bold}''); they don't come with plain \TeX,
but more elaborate formats like \AmSTeX\ do provide them.

%\subsection Miscellaneous symbols of type Ord.
\subsection 各种 Ord 类型的符号.
^^|\aleph|^^|\prime|^^|\forall|
^^|\hbar|^^|\emptyset|^^|\exists|
^^|\imath|^^|\nabla|^^|\neg|
^^|\jmath|^^|\surd|^^|\flat|
^^|\ell|^^|\top|^^|\natural|
^^|\wp|^^|\bot|^^|\sharp|
^^|\Re|^^{escvert}^^|\clubsuit|
^^|\Im|^^|\angle|^^|\diamondsuit|
^^|\partial|^^|\triangle|^^|\heartsuit|
^^|\infty|^^|\backslash|^^|\spadesuit|
^^{Weierstrass, see wp}
\beginsymbols
\aleph&|\aleph|&\prime&|\prime|&\forall&|\forall|\cr
\hbar&|\hbar|&\emptyset&|\emptyset|&\exists&|\exists|\cr
\imath&|\imath|&\nabla&|\nabla|&\neg&|\neg|\cr
\jmath&|\jmath|&\surd&|\surd|&\flat&|\flat|\cr
\ell&|\ell|&\top&|\top|&\natural&|\natural|\cr
\wp&|\wp|&\bot&|\bot|&\sharp&|\sharp|\cr
\Re&|\Re|&\Vert&|\|\|&\clubsuit&|\clubsuit|\cr
\Im&|\Im|&\angle&|\angle|&\diamondsuit&|\diamondsuit|\cr
\partial&|\partial|&\triangle&|\triangle|&\heartsuit&|\heartsuit|\cr
\infty&|\infty|&\backslash&|\backslash|&\spadesuit&|\spadesuit|\cr
\endsymbols
The ^{dotless letters} |\imath| and |\jmath| should be used when $i$ and~$j$
are ^{accent}ed; for example, |$\hat\imath$| yields $\hat\imath$.
The |\prime| symbol is intended for use in subscripts and superscripts,
as explained in Chapter~16, so you usually see it in a smaller size.
On the other hand, the |\angle| symbol has been built up from other pieces;
it does not get smaller when it appears in a subscript or superscript.

%\subsection Digits. To get italic ^{digits} {\it0123456789}, say
%|{\it0123456789}|; to get boldface digits {\bf0123456789}, say
%|{\bf0123456789}|; to get oldstyle digits {\oldstyle0123456789}, say
%|{\oldstyle0123456789}|. These conventions work also outside of math mode.
\subsection 数字. To get italic ^{digits} {\it0123456789}, say
|{\it0123456789}|; to get boldface digits {\bf0123456789}, say
|{\bf0123456789}|; to get oldstyle digits {\oldstyle0123456789}, say
|{\oldstyle0123456789}|. These conventions work also outside of math mode.

%\subsection ``Large'' operators. The following symbols come in two sizes,
%for text and display styles:
\subsection 巨算符. The following symbols come in two sizes,
for text and display styles:
^^|\sum|^^|\bigcap|^^|\bigodot|
^^|\prod|^^|\bigcup|^^|\bigotimes|
^^|\coprod|^^|\bigsqcup|^^|\bigoplus|
^^|\int|^^|\bigvee|^^|\biguplus|
^^|\oint|^^|\bigwedge|
$$\displayindent=16pt \openup3pt
\halign{&\qquad\hbox to10pt{\hss$#$\hss}\enspace&
  \hbox to10pt{\hss$\displaystyle#$\hss}\enspace&
  \hbox to60pt{#\hss}\cr
\sum&\sum&|\sum|&\bigcap&\bigcap&|\bigcap|&
  \bigodot&\bigodot&|\bigodot|\cr
\prod&\prod&|\prod|&\bigcup&\bigcup&|\bigcup|&
  \bigotimes&\bigotimes&|\bigotimes|\cr
\coprod&\coprod&|\coprod|&\bigsqcup&\bigsqcup&|\bigsqcup|&
  \bigoplus&\bigoplus&|\bigoplus|\cr
\int&\int&|\int|&\bigvee&\bigvee&|\bigvee|&
  \biguplus&\biguplus&|\biguplus|\cr
\oint&\oint&|\oint|&\bigwedge&\bigwedge&|\bigwedge|\cr
}$$
It is important to distinguish these large Op symbols from the similar but
smaller Bin symbols whose names are the same except for a `|big|' prefix.
Large operators usually occur at the beginning of a formula or subformula,
and they usually are subscripted; ^{binary operations} usually occur between
two symbols or subformulas, and they rarely are subscripted. For example,
\begindisplay
|$\bigcup_{n=1}^m(x_n\cup y_n)$|\qquad yields \qquad
$\bigcup_{n=1}^m(x_n\cup y_n)$
\enddisplay
The large operators |\sum|, |\prod|, |\coprod|, and |\int| should also be
distinguished from smaller symbols called |\Sigma|~($\Sigma$),
|\Pi|~($\Pi$), |\amalg|~($\amalg$), and |\smallint|~($\smallint$),
respectively; the ^|\smallint| operator is rarely used.

%\subsection Binary operations. Besides $+$ and $-$, you can type
\subsection 二元运算符. Besides $+$ and $-$, you can type
^^|\pm|^^|\cap|^^|\vee|
^^|\mp|^^|\cup|^^|\wedge|
^^|\setminus|^^|\uplus|^^|\oplus|
^^|\cdot|^^|\sqcap|^^|\ominus|
^^|\times|^^|\sqcup|^^|\otimes|
^^|\ast|^^|\triangleleft|^^|\oslash|
^^|\star|^^|\triangleright|^^|\odot|
^^|\diamond|^^|\wr|^^|\dagger|
^^|\circ|^^|\bigcirc|^^|\ddagger|
^^|\bullet|^^|\bigtriangleup|^^|\amalg|
^^|\div|^^|\bigtriangledown|
\beginsymbols
\pm&|\pm|&\cap&|\cap|&\vee&|\vee|\cr
\mp&|\mp|&\cup&|\cup|&\wedge&|\wedge|\cr
\setminus&|\setminus|&\uplus&|\uplus|&\oplus&|\oplus|\cr
\cdot&|\cdot|&\sqcap&|\sqcap|&\ominus&|\ominus|\cr
\times&|\times|&\sqcup&|\sqcup|&\otimes&|\otimes|\cr
\ast&|\ast|&\triangleleft&|\triangleleft|&\oslash&|\oslash|\cr
\star&|\star|&\triangleright&|\triangleright|&\odot&|\odot|\cr
\diamond&|\diamond|&\wr&|\wr|&\dagger&|\dagger|\cr
\circ&|\circ|&\bigcirc&|\bigcirc|&\ddagger&|\ddagger|\cr
\bullet&|\bullet|&\bigtriangleup&|\bigtriangleup|&\amalg&|\amalg|\cr
\div&|\div|&\bigtriangledown&|\bigtriangledown|\cr
\endsymbols
It's customary to say |$G\backslash| |H$| to denote double cosets
of $G$ by~$H$~($G\backslash H$), and |$p\backslash n$| to mean that
$p$ divides~$n$~($p\backslash n$); but |$X\setminus Y$| denotes
the elements of set~$X$ minus those of set~$Y$~($X\setminus Y$).
Both operations use the same symbol, but |\backslash| is
type Ord, while |\setminus| is type Bin (so \TeX\ puts
more space around it).

%\subsection Relations. Besides $<$, $>$, and $=$, you can type
\subsection 二元关系符. Besides $<$, $>$, and $=$, you can type
^^|\leq|^^|\geq|^^|\equiv|
^^|\prec|^^|\succ|^^|\approx|
^^|\preceq|^^|\succeq|^^|\propto|
^^|\ll|^^|\gg|^^|\asymp|
^^|\subset|^^|\supset|^^|\sim|
^^|\subseteq|^^|\supseteq|^^|\simeq|
^^|\sqsubseteq|^^|\sqsupseteq|^^|\cong|
^^|\in|^^|\ni|^^|\bowtie|
^^|\vdash|^^|\dashv|^^|\models|
^^|\smile|^^|\mid|^^|\doteq|
^^|\frown|^^|\parallel|^^|\perp|
\beginsymbols
\leq&|\leq|&\geq&|\geq|&\equiv&|\equiv|\cr
\prec&|\prec|&\succ&|\succ|&\sim&|\sim|\cr
\preceq&|\preceq|&\succeq&|\succeq|&\simeq&|\simeq|\cr
\ll&|\ll|&\gg&|\gg|&\asymp&|\asymp|\cr
\subset&|\subset|&\supset&|\supset|&\approx&|\approx|\cr
\subseteq&|\subseteq|&\supseteq&|\supseteq|&\cong&|\cong|\cr
\sqsubseteq&|\sqsubseteq|&\sqsupseteq&|\sqsupseteq|&\bowtie&|\bowtie|\cr
\in&|\in|&\ni&|\ni|&\propto&|\propto|\cr
\vdash&|\vdash|&\dashv&|\dashv|&\models&|\models|\cr
\smile&|\smile|&\mid&|\mid|&\doteq&|\doteq|\cr
\frown&|\frown|&\parallel&|\parallel|&\perp&|\perp|\cr
\endsymbols
The symbols |\mid| and |\parallel| define relations that use the same
characters as you get from \| and |\|\|; \TeX\ puts space around
them when they are relations.

%\subsection Negated relations. Many of the relations just listed can be
%negated or ``crossed out'' by prefixing them with ^|\not|, as follows:
\subsection 否定关系符. Many of the relations just listed can be
negated or ``crossed out'' by prefixing them with ^|\not|, as follows:
\beginsymbols
\not<&|\not<|&\not>&|\not>|&\not=&|\not=|\cr
\not\leq&|\not\leq|&\not\geq&|\not\geq|&
  \not\equiv&|\not\equiv|\cr
\not\prec&|\not\prec|&\not\succ&|\not\succ|&
  \not\sim&|\not\sim|\cr
\not\preceq&|\not\preceq|&\not\succeq&|\not\succeq|&
  \not\simeq&|\not\simeq|\cr
\not\subset&|\not\subset|&\not\supset&|\not\supset|&
  \not\approx&|\not\approx|\cr
\not\subseteq&|\not\subseteq|&\not\supseteq&|\not\supseteq|&
  \not\cong&|\not\cong|\cr
\not\sqsubseteq&|\not\sqsubseteq|&\not\sqsupseteq&|\not\sqsupseteq|&
  \not\asymp&|\not\asymp|\cr
\endsymbols
The symbol |\not| is a relation character of width zero, so it will
overlap a relation that comes immediately after it. The positioning
isn't always ideal, because some relation symbols are wider than others;
for example, |\not\in| gives `$\not\in$', but it is preferable to
have a steeper cancellation, `$\notin$'. The latter symbol is
available as a special control sequence called ^|\notin|. The definition
of\/ |\notin| in Appendix~B indicates how similar symbols can be constructed.

%\subsection Arrows. There's also another big class of relations, namely
\subsection 箭头. There's also another big class of relations, namely
^^{arrows}
^^|\leftarrow| ^^|\longleftarrow| ^^|\uparrow| ^^|\Leftarrow|
^^|\Longleftarrow| ^^|\Uparrow| ^^|\rightarrow| ^^|\longrightarrow|
^^|\downarrow| ^^|\Rightarrow| ^^|\Longrightarrow| ^^|\Downarrow|
^^|\leftrightarrow| ^^|\longleftrightarrow| ^^|\updownarrow|
^^|\Leftrightarrow| ^^|\Longleftrightarrow| ^^|\Updownarrow| ^^|\mapsto|
^^|\longmapsto| ^^|\nearrow| ^^|\hookleftarrow| ^^|\hookrightarrow|
^^|\searrow| ^^|\leftharpoonup| ^^|\rightharpoonup| ^^|\swarrow|
^^|\leftharpoondown| ^^|\rightharpoondown| ^^|\nwarrow|
^^|\rightleftharpoons|
those that point:
$$\displayindent=16pt
  \halign\bgroup
  \qquad\hbox to10pt{\hss$#$\hss}\enspace&\hbox to75pt{#\hss}&
  \qquad\hbox to10pt{\hss$#$\hss}\enspace&\hbox to85pt{#\hss}&
  \qquad\hbox to10pt{\hss$#$\hss}\enspace&\hbox to80pt{#\hss}\cr
\leftarrow&|\leftarrow|&\longleftarrow&|\longleftarrow|&
  \uparrow&|\uparrow|\cr
\Leftarrow&|\Leftarrow|&\Longleftarrow&|\Longleftarrow|&
  \Uparrow&|\Uparrow|\cr
\rightarrow&|\rightarrow|&\longrightarrow&|\longrightarrow|&
  \downarrow&|\downarrow|\cr
\Rightarrow&|\Rightarrow|&\Longrightarrow&|\Longrightarrow|&
  \Downarrow&|\Downarrow|\cr
\leftrightarrow&|\leftrightarrow|&\longleftrightarrow&|\longleftrightarrow|&
  \updownarrow&|\updownarrow|\cr
\Leftrightarrow&|\Leftrightarrow|&\Longleftrightarrow&|\Longleftrightarrow|&
  \Updownarrow&|\Updownarrow|\cr
\mapsto&|\mapsto|&\longmapsto&|\longmapsto|&
  \nearrow&|\nearrow|\cr
\hookleftarrow&|\hookleftarrow|&\hookrightarrow&|\hookrightarrow|&
  \searrow&|\searrow|\cr
\leftharpoonup&|\leftharpoonup|&\rightharpoonup&|\rightharpoonup|&
  \swarrow&|\swarrow|\cr
\leftharpoondown&|\leftharpoondown|&\rightharpoondown&|\rightharpoondown|&
  \nwarrow&|\nwarrow|\cr
\rightleftharpoons&|\rightleftharpoons|\cr
\enddisplay
Up and down arrows will grow larger, like delimiters (see Chapter~17).
To put symbols over left and right arrows, plain \TeX\ provides
a ^|\buildrel| macro: You type |\buildrel|\<superscript>^|\over|\<relation>,
and the superscript is placed on top of the relation just as limits are
placed over large operators. For example,
\begindisplay
\hbox to 30pt{$\buildrel \alpha\beta \over \longrightarrow$\hfil}%
|\buildrel \alpha\beta \over \longrightarrow|\cr
\hbox to 30pt{$\buildrel \rm def \over =$\hfil}%
|\buildrel \rm def \over =|\cr
\enddisplay
\ (In this context, `|\over|' does not define a fraction.)

%\subsection Openings. The following ^{left delimiters} are available,
%besides `(':
\subsection 开符号. The following ^{left delimiters} are available,
besides `(':
^^|\lbrack| ^^|\lbrace| ^^|\langle| ^^|\lfloor| ^^|\lceil|
\beginsymbols
\lbrack&|\lbrack|&\lfloor&|\lfloor|&\lceil&|\lceil|\cr
\lbrace&|\lbrace|&\langle&|\langle|\cr
\endsymbols
You can also type simply `|[|' to get |\lbrack|. All of these will grow if
you prefix them by ^|\bigl|, ^|\Bigl|, ^|\biggl|, ^|\Biggl|, or ^|\left|.
Chapter~17 also mentions ^|\lgroup| and ^|\lmoustache|, which are
available in sizes greater than |\big|. If you need more delimiters,
the following combinations work reasonably well in the normal text size:
\beginsymbols
\lbrack\!\lbrack&|\lbrack\!\lbrack|&
\langle\!\langle&|\langle\!\langle|&
(\!(&|(\!(|\cr
\endsymbols

%\subsection Closings. The corresponding right delimiters are present too:
\subsection 闭符号. The corresponding right delimiters are present too:
^^|\rbrack| ^^|\rbrace| ^^|\rangle| ^^|\rfloor| ^^|\rceil|
\beginsymbols
\rbrack&|\rbrack|&\rfloor&|\rfloor|&\rceil&|\rceil|\cr
\rbrace&|\rbrace|&\rangle&|\rangle|\cr
\endsymbols
Everything that works for openings works also for closings, but reversed.
\goodbreak

%\subsection Punctuation. \TeX\ puts a thin space after commas and semicolons
%that appear in mathematical formulas, and it does the same for a colon
%that is called ^|\colon|. \ (Otherwise a ^{colon} is considered to be a
%relation, as in `$x:=y$' and `$a:b::c:d$', which you type by saying
%`|$x:=y$|' and `|$a:b::c:d$|'.) \ Examples of\/ |\colon| are
%\begindisplay
%\hbox to 115pt{$f\colon A\rightarrow B$\hfil}\enspace
%|$f\colon A\rightarrow B$|\cr
%\hbox to 115pt{$L(a,b;c\colon x,y;z)$\hfil}\enspace
%|$L(a,b;c\colon x,y;z)$|\cr
%\enddisplay
%Plain \TeX\ also defines ^|\ldotp| and ^|\cdotp| to be `.' and `$\cdot$'
%with the spacing of commas and semicolons. These symbols don't
%occur directly in formulas, but they are useful in the definition of\/
%^|\ldots| and ^|\cdots|.
\subsection 标点符号. \TeX\ puts a thin space after commas and semicolons
that appear in mathematical formulas, and it does the same for a colon
that is called ^|\colon|. \ (Otherwise a ^{colon} is considered to be a
relation, as in `$x:=y$' and `$a:b::c:d$', which you type by saying
`|$x:=y$|' and `|$a:b::c:d$|'.) \ Examples of\/ |\colon| are
\begindisplay
\hbox to 115pt{$f\colon A\rightarrow B$\hfil}\enspace
|$f\colon A\rightarrow B$|\cr
\hbox to 115pt{$L(a,b;c\colon x,y;z)$\hfil}\enspace
|$L(a,b;c\colon x,y;z)$|\cr
\enddisplay
Plain \TeX\ also defines ^|\ldotp| and ^|\cdotp| to be `.' and `$\cdot$'
with the spacing of commas and semicolons. These symbols don't
occur directly in formulas, but they are useful in the definition of\/
^|\ldots| and ^|\cdots|.

%\subsection Alternate names. If you don't like plain \TeX's name for
%some math symbol---for example, if there's another name that looks
%better or that you can remember more easily---the remedy is simple:
%You just say, e.g., `|\let\cupcap=\asymp|'. Then you can type
%`|f(n)\cupcap n|' instead of `|f(n)\asymp n|'.
\subsection 别名. If you don't like plain \TeX's name for
some math symbol---for example, if there's another name that looks
better or that you can remember more easily---the remedy is simple:
You just say, e.g., `|\let\cupcap=\asymp|'. Then you can type
`|f(n)\cupcap n|' instead of `|f(n)\asymp n|'.

Some symbols have alternate names that are so commonly used that plain
\TeX\ provides two or more equivalent control sequences:
^^|\ne| ^^|\neq| ^^|\le| ^^|\ge| ^^|\to| ^^|\gets| ^^|\owns| ^^|\land|
^^|\lor| ^^|\lnot| ^^|\vert|
$$\halign{\indent$\hfil#\hfil$\enspace&#\hfil\enspace&
  (same as #)\hfil\cr
\ne&|\ne| or |\neq|&|\not=|\cr
\le&|\le|&|\leq|\cr
\ge&|\ge|&|\geq|\cr
\{&|\{|&|\lbrace|\cr
\}&|\}|&|\rbrace|\cr
\to&|\to|&|\rightarrow|\cr
\gets&|\gets|&|\leftarrow|\cr
\owns&|\owns|&|\ni|\cr
\land&|\land|&|\wedge|\cr
\lor&|\lor|&|\vee|\cr
\lnot&|\lnot|&|\neg|\cr
\vert&|\vert|&\|\cr
\Vert&|\Vert|&|\|\|\cr
}$$
There's also ^|\iff| ($\iff$), which is just like |\Longleftrightarrow|
except that it puts an extra thick space at each side.

%\subsection Non-math symbols. Plain \TeX\ makes four special symbols
%available outside of math mode, although the characters themselves
%are actually typeset from the math symbols font:
%$$\halign{\indent$\hfil#\hfil$\enspace&#\hfil\enspace\cr
%\S&|\S|\cr
%\P&|\P|\cr
%\dag&|\dag|\cr
%\ddag&|\ddag|\cr
%\phantom{\gets}\cr\noalign{\kern-\baselineskip}
%}$$
%These control sequences do not act like ordinary math symbols; they don't
%change their size when they appear in subscripts or superscripts, and you
%must say, e.g., |$x^{\P}$| instead of |$x^\P$| when you use them in
%formulas. However, the ^|\dag| and ^|\ddag| symbols are available in
%math mode under the names ^|\dagger| and ^|\ddagger|. It would be easy
%to define mathematical equivalents of\/ ^|\S| and ^|\P|, if~these symbols
%suddenly caught a mathematician's fancy.
\subsection 非数学符号. Plain \TeX\ makes four special symbols
available outside of math mode, although the characters themselves
are actually typeset from the math symbols font:
$$\halign{\indent$\hfil#\hfil$\enspace&#\hfil\enspace\cr
\S&|\S|\cr
\P&|\P|\cr
\dag&|\dag|\cr
\ddag&|\ddag|\cr
\phantom{\gets}\cr\noalign{\kern-\baselineskip}
}$$
These control sequences do not act like ordinary math symbols; they don't
change their size when they appear in subscripts or superscripts, and you
must say, e.g., |$x^{\P}$| instead of |$x^\P$| when you use them in
formulas. However, the ^|\dag| and ^|\ddag| symbols are available in
math mode under the names ^|\dagger| and ^|\ddagger|. It would be easy
to define mathematical equivalents of\/ ^|\S| and ^|\P|, if~these symbols
suddenly caught a mathematician's fancy.

\endchapter

Seek not for fresher founts afar,
Just drop your bucket where you are.
\author SAM WALTER ^{FOSS}, {\sl Back Country Poems\/} (1892)

\bigskip

No one compositor will have all the signs and symbols available.
The number of special signs and symbols is almost limitless,
with new ones being introduced all the time.
\author UNIVERSITY OF ^{CHICAGO} PRESS, {\sl Manual of Style\/} (1969) % p296

%\eject
\eject\byebye
