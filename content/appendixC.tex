% -*- coding: utf-8 -*-

\input macros

%\beginchapter Appendix C. Character\\Codes
\beginchapter Appendix C. 字符编码

%\ninepoint
%Different computers tend to have different ways of representing the
%characters in files of text, but \TeX\ gives the same results on
%all machines, because it converts everything to a standard internal
%code when it reads a file. \TeX\ also converts back from its internal
%representation to the appropriate external code, when it writes
%a file of text; therefore most users need not be aware of the fact
%that the ^{codes} have actually switched back and forth inside the machine.
\ninepoint
不同的计算机往往用不同的方式表示文本文件的字符,
但 \TeX\ 在所有机器上都给出相同的结果,
因为它在读取文件时将所有东西都转换为标准的内部编码。
而 \TeX\ 在写入文本文件时也将内部编码转回恰当的外部编码;
因此大多数用户不会意识到实际上字符^{编码}在机器中来回转换过。

%The purpose of this appendix is to define \TeX's internal code,
%which has the same characteristics on all implementations of \TeX.
%The existence of such a code is important, because it
%makes \TeX\ constructions ``portable.'' For example, \TeX\ allows
%^{alphabetic constants} like |`b| to be used as numbers; the fact
%that |`b| always denotes the integer~98 means that we can
%write machine-independent macros that decide, for instance, whether
%a given character is a digit between |0| and |9|.
%Furthermore the internal code of \TeX\ also survives in its ^|dvi|
%output files, which can be printed by software that knows nothing
%about where the |dvi| data originated; essentially the same
%output will be obtained from all implementations of \TeX, regardless
%of the host computer, because the |dvi| data is expressed in a
%machine-independent code.
这个附录的目的是定义 \TeX\ 的内部编码,它在所有 \TeX\ 实现中都是一致的。
这种编码的存在性是重要的,因为它使得 \TeX\ 是``可移植的''。
例如,\TeX\ 允许将类似 |`b| 的^{字母表常数}作为整数使用;
|`b| 永远表示整数~98 意味着我们可以编写机器无关的宏,
用于确定给定的字符是否为一个在 |0| 和 |9| 之间的数字。
此外,\TeX\ 的内部编码还用于它的 ^|dvi| 输出文件;
打印 |dvi| 文件的软件不知道其数据来自哪里,
但在任何主机的 \TeX\ 实现中将得到完全相同的输出,
因为 |dvi| 数据是用机器无关的编码表示的。

%\smallskip
%\TeX's internal code is based on the American Standard Code for
%Information Interchange, known popularly as ``^{ASCII}.'' There are
%128 codes, numbered 0~to~127; we conventionally express the numbers
%in octal notation, from \oct{000} to \oct{177}, or in
%hexadecimal notation, from \hex{00} to \hex{7F}. Thus, the value of
%|`b| is normally called \oct{142} or \hex{62}, not 98. In the
%ASCII scheme, codes \oct{000} through \oct{040} and
%code \oct{177} are~assigned to special functions; for example,
%code \oct{007} is called |BEL|, and it means ``Ring the bell.''
%The other 94 codes are assigned to visible symbols. Here is a
%chart that shows ASCII codes in such a way that octal and hexadecimal
%equivalents can easily be read off:
\smallskip
\TeX\ 的内部编码基于``^{ASCII}''(American Standard Code for
Information Interchange)。其中一共有 128 个代码,以 0~到~127 编号;
习惯上对这些数字我们用 \oct{000} 到 \oct{177} 的八进制表示,
或者用 \hex{00} 到 \hex{7F} 的十六进制表示。
因此,|`b| 的值通常被称为 \oct{142} 或 \hex{62},而不是 98。
在 ASCII 体制中,从 \oct{000} 到 \oct{040} 以及 \oct{177}
的代码分配给特殊功能;例如,编码 \oct{007} 称为 |BEL|,
它的意思为 ``Ring the bell''。另外 94 个代码分配给可见符号。
这里有个图表显示了 ASCII 编码,在其中很容易看出八进制和十六进制的对应:
\beginchart{\global\count@='41\tentt
  \def\chartstrut{\lower4.3pt\vbox to13.6pt{}}}
&\oct{00x}&&NUL&&SOH&&STX&&ETX&&EOT&&ENQ&&ACK&&BEL&&\oddline0
&\oct{01x}&&BS&&HT&&LF&&VT&&FF&&CR&&SO&&SI&\evenline
&\oct{02x}&&DLE&&DC1&&DC2&&DC3&&DC4&&NAK&&SYN&&ETB&&\oddline1
&\oct{03x}&&CAN&&EM&&SUB&&ESC&&FS&&GS&&RS&&US&\evenline
&\oct{04x}&&SP&&\:&&\:&&\:&&\:&&\:&&\:&&\:&&\oddline2
&\oct{05x}&&\:&&\:&&\:&&\:&&\:&&\:&&\:&&\:&\evenline
&\oct{06x}&&\:&&\:&&\:&&\:&&\:&&\:&&\:&&\:&&\oddline3
&\oct{07x}&&\:&&\:&&\:&&\:&&\:&&\:&&\:&&\:&\evenline
&\oct{10x}&&\:&&\:&&\:&&\:&&\:&&\:&&\:&&\:&&\oddline4
&\oct{11x}&&\:&&\:&&\:&&\:&&\:&&\:&&\:&&\:&\evenline
&\oct{12x}&&\:&&\:&&\:&&\:&&\:&&\:&&\:&&\:&&\oddline5
&\oct{13x}&&\:&&\:&&\:&&\:&&\:&&\:&&\:&&\:&\evenline
&\oct{14x}&&\:&&\:&&\:&&\:&&\:&&\:&&\:&&\:&&\oddline6
&\oct{15x}&&\:&&\:&&\:&&\:&&\:&&\:&&\:&&\:&\evenline
&\oct{16x}&&\:&&\:&&\:&&\:&&\:&&\:&&\:&&\:&&\oddline7
&\oct{17x}&&\:&&\:&&\:&&\:&&\:&&\:&&\:&&DEL&\evenline
\endchart

%Ever since ASCII was established in the early 1960s, people have had
%different ideas about what to do with positions \oct{000}--\oct{037} and
%\oct{177}, because most of the functions assigned to those codes are
%appropriate only for special purposes like file transmission, not for
%applications to printing or to interactive computing. It turned out that
%manufacturers soon started producing line printers that were capable of
%generating 128 characters, 33~of~which were tailored to the special needs
%of particular customers; part of the advantage of a standard code was
%therefore lost. On the other hand, the remaining 95~codes (including
%\oct{40}=|SP|, a blank space) have become widely adopted, and they are now
%implanted within most of today's computer terminals. When an ASCII
%keyboard is available, you can specify each of the 128 codes to \TeX\ in
%terms of the 95 standard characters, as follows:
自从 1960 年代 ASCII 建立之后,
人们对如何对待位置 \oct{000}--\oct{037} 和 \oct{177} 有不同的想法,
因为这些编码对应的功能仅在文件传输等特殊目的中才用得到,在打印或者交互计算中用不到。
结果是制造商很快就开始生产可以生成 128 个字符的行式打印机,
其中的 33 个字符被按照特定客户的特别需求而定制;标准编码因此丢失了部分优势。
另一方面,剩下的 95 个编码(包括 \oct{40}=|SP|,即空格)被广泛采用,
并在今天的大多数计算机终端中扎根。在有 ASCII 键盘可用时,
你可以用 95 个标准字符给 \TeX\ 指定 128 个编码中任何一个,方法如下:
\beginchart{\def\?{\char`^\char`^\:}\global\count@='100
  \postdisplaypenalty=0 \tentt}
&\oct{00x}&&\?&&\?&&\?&&\?&&\?&&\?&&\?&&\?&&\oddline0
&\oct{01x}&&\?&&\?&&\?&&\?&&\?&&\?&&\?&&\?&\evenline
&\oct{02x}&&\?&&\?&&\?&&\?&&\?&&\?&&\?&&\?&&\oddline1
&\oct{03x}&&\?&&\?&&\?&&\?&&\?&&\?&&\?&&\?&\evenline
&\global\count@='40
\oct{04x}&&\:&&\:&&\:&&\:&&\:&&\:&&\:&&\:&&\oddline2
&\oct{05x}&&\:&&\:&&\:&&\:&&\:&&\:&&\:&&\:&\evenline
&\oct{06x}&&\:&&\:&&\:&&\:&&\:&&\:&&\:&&\:&&\oddline3
&\oct{07x}&&\:&&\:&&\:&&\:&&\:&&\:&&\:&&\:&\evenline
&\oct{10x}&&\:&&\:&&\:&&\:&&\:&&\:&&\:&&\:&&\oddline4
&\oct{11x}&&\:&&\:&&\:&&\:&&\:&&\:&&\:&&\:&\evenline
&\oct{12x}&&\:&&\:&&\:&&\:&&\:&&\:&&\:&&\:&&\oddline5
&\oct{13x}&&\:&&\:&&\:&&\:&&\:&&\:&&\:&&\:&\evenline
&\oct{14x}&&\:&&\:&&\:&&\:&&\:&&\:&&\:&&\:&&\oddline6
&\oct{15x}&&\:&&\:&&\:&&\:&&\:&&\:&&\:&&\:&\evenline
&\oct{16x}&&\:&&\:&&\:&&\:&&\:&&\:&&\:&&\:&&\oddline7
&\oct{17x}&&\:&&\:&&\:&&\:&&\:&&\:&&\:&&\char`^\char`^?&\evenline
\endchart
%(Here `|^^|' ^^{uparrow uparrow} doesn't necessarily mean two
%circumflex characters; it means two identical characters whose
%current |\catcode| is~7. In such cases \TeX\ simply adds or
%subtracts \oct{100} from the internal code of the character
%that immediately follows. For example, |*|~can also be typed as |^^j|;
%|j|~can also be typed as |^^*|.)
(这里的 `|^^|' ^^{uparrow uparrow} 并不一定非得是两个扬抑符;
两个当前 |\catcode| 为~7 的字符也可以。
在这种用法中,\TeX\ 就将后面跟随的字符的内部编码加上或减去 \oct{100}。
例如,输入 |^^j| 可以得到 |*|;而输入 |^^*| 可以得到|j|。)

%An extended ASCII code intended for text editing and interactive computing
% was developed at several universities about 1965, and
%for many years there have been terminals in use at Stanford, MIT,
%Carnegie-Mellon, and elsewhere that have 120 or~121 symbols, not just~95.
%Aficionados of these keyboards (like the author of this book)
%are loath to give up their extra characters; it seems that
%such people make heavy use of about 5~of the extra~25, and occasional
%use of the other~20, although
%different people have different groups of five. For example, the
%author developed \TeX\ on a keyboard that includes the symbols
%{\tentex\char'30},~{\tentex\char1}, {\tentex\char'32}, {\tentex\char'34},
%and~{\tentex\char'35}, and he finds that this makes it much more pleasant
%to type class notes,
%technical papers, and computer programs of the kind he likes to write;
%his logician friends make heavy use of the {\tentex\char'24} and
%{\tentex\char'25} keys; and so on. It is recommended that \TeX\
%implementations on systems with large character sets be consistent with the
%following codes:
%\beginchart{\tentex\postdisplaypenalty=0}
%\normalchart
%\endchart
%Of course, designers of \TeX\ macro packages that are intended to be
%widely used should stick to the standard ASCII characters.
在 1965 年左右,有些大学开发了一个面向文本编辑和交互计算的扩展 ASCII 编码,
而在 Stanford、MIT、Carnegie-Mellon 和其他大学使用多年的一种终端有
120 或 121 个符号,不仅仅是 95 个。
这些键盘的爱好者(如本书作者)不愿意放弃他们的附加字符;
看起来这些人经常使用这 25 个补充字符的其中 5 个,而偶尔使用另外 20 个,
虽然常使用的 5 个字符对不同人来说是不同的。
例如,作者在一个包含符号 {\tentex\char'30}、{\tentex\char1}、{\tentex\char'32}、
{\tentex\char'34} 和 {\tentex\char'35} 的键盘上开发 \TeX ,
他发现这可以更加愉快地输入课程笔记、技术论文以及他喜欢编写的那些计算机程序;
而他的逻辑学朋友经常使用 {\tentex\char'24} 和 {\tentex\char'25} 按键;等等。
建议在拥有大字符集的系统上的 \TeX\ 实现与下面的编码保持一致:
\beginchart{\tentex\postdisplaypenalty=0}
\normalchart
\endchart
当然,面向广大用户的 \TeX\ 宏包的设计者应当只使用标准的 ASCII 字符。

%Incidentally, the ASCII character |^| that appears in position \oct{136}
%is sometimes called a ``^{caret},'' but dictionaries of English tell us
%that a caret is a larger symbol, more like character \oct{004} in the
%extended set above. The correct name for |^| is ``^{circumflex},'' but
%this is quite a mouthful, so a shorter name like ``^{hat}'' is preferable.
%It seems desirable to preserve the traditional distinction between
%caret and hat.
Incidentally, the ASCII character |^| that appears in position \oct{136}
is sometimes called a ``^{caret},'' but dictionaries of English tell us
that a caret is a larger symbol, more like character \oct{004} in the
extended set above. The correct name for |^| is ``^{circumflex},'' but
this is quite a mouthful, so a shorter name like ``^{hat}'' is preferable.
It seems desirable to preserve the traditional distinction between
caret and hat.

%The extended code shown above was developed at MIT; it is similar to, but
%slightly better than, the code implemented at Stanford. Seven of the codes
%are conventionally assigned to the standard ASCII control functions
%|NUL| (^\<null>), |HT| (^\<tab>), |LF| (^\<linefeed>), |FF|
%(^\<formfeed>), |CR| (^\<return>), |ESC| (^\<escape>), and
%|DEL| (^\<delete>), and they appear in the standard ASCII positions; hence the
%corresponding seven characters
%{\tentex\char0~\char'11~\char'12~\char'14~\char'15~\char'33~\char'177}
%do not actually appear on the keyboard. These seven ``hidden''
%characters show up only on certain output devices.
The extended code shown above was developed at MIT; it is similar to, but
slightly better than, the code implemented at Stanford. Seven of the codes
are conventionally assigned to the standard ASCII control functions
|NUL| (^\<null>), |HT| (^\<tab>), |LF| (^\<linefeed>), |FF|
(^\<formfeed>), |CR| (^\<return>), |ESC| (^\<escape>), and
|DEL| (^\<delete>), and they appear in the standard ASCII positions; hence the
corresponding seven characters
{\tentex\char0~\char'11~\char'12~\char'14~\char'15~\char'33~\char'177}
do not actually appear on the keyboard. These seven ``hidden''
characters show up only on certain output devices.

%Modern keyboards allow 256 codes to be input, not just 128; so \TeX\
%represents characters internally as numbers in the range 0--255 (i.e.,
%\oct{000}--\oct{377}, or \hex{00}--\hex{FF}). Implementations of \TeX\
%differ in which characters they will accept in input files and which
%they will transmit to output files; these subsets can be specified
%independently. A completely permissive version of \TeX\ allows full
%256-character input and output; other versions might ignore all
%but the visible characters of ASCII; still other versions might
%distinguish the tab character (code \oct{011}) from a space on input,
%but might output each tab as a sequence of three characters |^^I|.
Modern keyboards allow 256 codes to be input, not just 128; so \TeX\
represents characters internally as numbers in the range 0--255 (i.e.,
\oct{000}--\oct{377}, or \hex{00}--\hex{FF}). Implementations of \TeX\
differ in which characters they will accept in input files and which
they will transmit to output files; these subsets can be specified
independently. A completely permissive version of \TeX\ allows full
256-character input and output; other versions might ignore all
but the visible characters of ASCII; still other versions might
distinguish the tab character (code \oct{011}) from a space on input,
but might output each tab as a sequence of three characters |^^I|.

%Many people, unfortunately, have the opposite problem:\ Instead of the
%95~standard characters and some others, they have fewer than~95 symbols
%actually available. What can be done in such cases? Well, it's possible
%to use \TeX\ with fewer symbols, by invoking more control
%sequences; for example, plain \TeX\ defines ^|\lq|, ^|\rq|, ^|\lbrack|,
%^|\rbrack|, ^|\sp|, and ^|\sb|, so that you need not type |`|,~|'|, |[|, |]|,
%|^|, and~|_|, respectively.
Many people, unfortunately, have the opposite problem:\ Instead of the
95~standard characters and some others, they have fewer than~95 symbols
actually available. What can be done in such cases? Well, it's possible
to use \TeX\ with fewer symbols, by invoking more control
sequences; for example, plain \TeX\ defines ^|\lq|, ^|\rq|, ^|\lbrack|,
^|\rbrack|, ^|\sp|, and ^|\sb|, so that you need not type |`|,~|'|, |[|, |]|,
|^|, and~|_|, respectively.

%A person who implements \TeX\ on computer systems that do not have
%95~externally representable symbols should adhere to the following
%guidelines: \ (a)~Stay as close as possible to the ASCII conventions.
%\ (b)~Make sure that codes \oct{041}--\oct{046}, \oct{060}--\oct{071},
%\oct{141}--\oct{146}, and \oct{160}--\oct{171} are present and that
%each unrepresentable
%internal code $<\null$\oct{200} leads to a representable code when \oct{100} is
%added or subtracted; then all 256 codes can be input and output.
%\ (c)~Cooperate with everyone else who shares
%the same constraints, so that you all adopt the same policy.
%\ (See Appendix~J for information about the \TeX\ Users Group.)
A person who implements \TeX\ on computer systems that do not have
95~externally representable symbols should adhere to the following
guidelines: \ (a)~Stay as close as possible to the ASCII conventions.
\ (b)~Make sure that codes \oct{041}--\oct{046}, \oct{060}--\oct{071},
\oct{141}--\oct{146}, and \oct{160}--\oct{171} are present and that
each unrepresentable
internal code $<\null$\oct{200} leads to a representable code when \oct{100} is
added or subtracted; then all 256 codes can be input and output.
\ (c)~Cooperate with everyone else who shares
the same constraints, so that you all adopt the same policy.
\ (See Appendix~J for information about the \TeX\ Users Group.)

%\def\Russiantt#1{{\tt\hbox to.5em{\hss\eighttt\char#1\hss}}}
%Very few conventions about character codes are hardwired into \TeX:
%Almost everything can be changed by a format package that changes parameters
%like |\escapechar| and sets up the |\catcode|, |\mathcode|, |\uccode|,
%|\lccode|, |\sfcode|, and |\delcode| tables.
%Thus a \TeX\ manuscript that has been written in ^{Denmark}, say,
%can be run in California, and vice versa, even though quite different
%conventions might be used in different countries. The only character codes
%that \TeX\ actually ``knows'' are these: \ (1)~|INITEX| initializes
%the code tables as described in Appendix~B\null; the same initialization is
%done by all implementations of \TeX. \ (2)~\TeX\ uses the character codes
%\]|+-.,`'"<=>0123456789ABCDEF| in its syntax rules (Chapters 20, 24, and
%Appendix~H\null), and it
%uses most of the uppercase and lowercase letters in its ^{keywords}
%|pt|, |to|, |plus|, etc. These same codes and keywords are used in
%all implementations of \TeX. For example, when \TeX\ is implemented
%for ^{Cyrillic} ^^{Russian} keyboards, the letter `\Russiantt5' should be
%assigned to code \oct{160} and `\Russiantt{`T}' to code \oct{164}, so that
%`\Russiantt5\Russiantt{`T}' still means `|pt|'; or else control sequences
%should be defined so that what \TeX\ sees is equivalent to the keywords
%it needs. \ (3)~The operations |\number|,
%|\romannumeral|, |\the|, and |\meaning| can generate letters, digits, spaces,
%decimal points, minus signs, double quotes, colons, and `|>|' signs; these same
%codes are generated in all implementations of \TeX.
%\ (4)~The |\hyphenation| and |\pattern| commands described in Appendix~H
%give special interpretation to the ten digits and to the characters
%`|.|' and~`|-|'. \ (5)~The codes for the four characters |$|~|.|~|{|~|}|
%are inserted when \TeX\ recovers from certain errors, and braces are
%inserted around an ^|\output| routine; appropriate catcodes are attached
%to these tokens, so it doesn't matter if these symbols have their plain
%\TeX\ meanings or not. \ (6)~There is a special convention for
%representing characters 0--255 in the hexadecimal forms
%|^^00|--|^^ff|, explained in Chapter~8. This convention is always
%acceptable as input, when |^| is any character of catcode~7. Text
%output is produced with this convention only when representing
%characters of code $\ge128$ that a \TeX\ installer has chosen not to
%output directly.
\def\Russiantt#1{{\tt\hbox to.5em{\hss\eighttt\char#1\hss}}}
Very few conventions about character codes are hardwired into \TeX:
Almost everything can be changed by a format package that changes parameters
like |\escapechar| and sets up the |\catcode|, |\mathcode|, |\uccode|,
|\lccode|, |\sfcode|, and |\delcode| tables.
Thus a \TeX\ manuscript that has been written in ^{Denmark}, say,
can be run in California, and vice versa, even though quite different
conventions might be used in different countries. The only character codes
that \TeX\ actually ``knows'' are these: \ (1)~|INITEX| initializes
the code tables as described in Appendix~B\null; the same initialization is
done by all implementations of \TeX. \ (2)~\TeX\ uses the character codes
\]|+-.,`'"<=>0123456789ABCDEF| in its syntax rules (Chapters 20, 24, and
Appendix~H\null), and it
uses most of the uppercase and lowercase letters in its ^{keywords}
|pt|, |to|, |plus|, etc. These same codes and keywords are used in
all implementations of \TeX. For example, when \TeX\ is implemented
for ^{Cyrillic} ^^{Russian} keyboards, the letter `\Russiantt5' should be
assigned to code \oct{160} and `\Russiantt{`T}' to code \oct{164}, so that
`\Russiantt5\Russiantt{`T}' still means `|pt|'; or else control sequences
should be defined so that what \TeX\ sees is equivalent to the keywords
it needs. \ (3)~The operations |\number|,
|\romannumeral|, |\the|, and |\meaning| can generate letters, digits, spaces,
decimal points, minus signs, double quotes, colons, and `|>|' signs; these same
codes are generated in all implementations of \TeX.
\ (4)~The |\hyphenation| and |\pattern| commands described in Appendix~H
give special interpretation to the ten digits and to the characters
`|.|' and~`|-|'. \ (5)~The codes for the four characters |$|~|.|~|{|~|}|
are inserted when \TeX\ recovers from certain errors, and braces are
inserted around an ^|\output| routine; appropriate catcodes are attached
to these tokens, so it doesn't matter if these symbols have their plain
\TeX\ meanings or not. \ (6)~There is a special convention for
representing characters 0--255 in the hexadecimal forms
|^^00|--|^^ff|, explained in Chapter~8. This convention is always
acceptable as input, when |^| is any character of catcode~7. Text
output is produced with this convention only when representing
characters of code $\ge128$ that a \TeX\ installer has chosen not to
output directly.

\endchapter

%Code sets obtained by modifying the standard as shown above
%or by other replacements are nonstandard.
%\author ASA SUBCOMMITTEE X3.2, {\sl American Standard\break %
%  Code for Information Interchange\/^^{ASCII}} (1963)
%% in {\sl Communications of the ACM\/}
Code sets obtained by modifying the standard as shown above
or by other replacements are nonstandard.
\author ASA SUBCOMMITTEE X3.2, {\sl American Standard\break %
  Code for Information Interchange\/^^{ASCII}} (1963)
% in {\sl Communications of the ACM\/}

\bigskip

%Both the Stanford and DEC uses of the ASCII control characters
%are in violation of the USA Standard Code,
%but no Federal Marshal is likely to come running out
%and arrest people who type control-T to their computers.
%\author BRIAN ^{REID}, {\sl SCRIBE Introductory User's Manual\/} (1978) % p82
Both the Stanford and DEC uses of the ASCII control characters
are in violation of the USA Standard Code,
but no Federal Marshal is likely to come running out
and arrest people who type control-T to their computers.
\author BRIAN ^{REID}, {\sl SCRIBE Introductory User's Manual\/} (1978) % p82

%\eject
\eject\byebye
