% -*- coding: utf-8 -*-

\input macros

%\beginchapter Appendix B. Basic\\Control\\Sequences
\beginchapter Appendix B. 基本控制系列

%Let's begin this appendix with a chart that summarizes plain \TeX's
%^^{summary of plain TeX} ^^|\+| ^^{tabbing} ^^{accents}
%conventions.
在这个附录开头,我们用一个图表总结 plain \TeX\ 的惯用法。
^^{summary of plain TeX} ^^|\+| ^^{tabbing} ^^{accents}

\def\sep{\medskip\hrule width\hsize\medskip}
\medskip\smallskip
\hrule height .61803pt
\kern 1pt
\hrule
\medskip
\line{\strut Characters that are reserved for special purposes:\hfil
  |\|\hfil|{|\hfil|}|\hfil|$|\hfil|&|\hfil|#|\hfil|%|\hfil|^|\hfil|_|\hfil|~|}
\sep
\halign to\hsize{\strut\hfil#\hfil\tabskip\z@ plus10pt&
\hfil#\hfil&\hfil#\hfil&
\hfil#\hfil\tabskip\z@skip\cr
|\rm roman,|&|{\sl slanted},|&|{\bf boldface},|&|{\it italic\/} type|\cr
roman,&{\sl slanted},&{\bf boldface},&{\it italic\/} type\cr}
\sep
\halign to\hsize{\strut\hfil#\hfil\tabskip\z@ plus10pt&
\hfil#\hfil&\hfil#\hfil&\hfil#\hfil&\hfil#\hfil&\hfil#\hfil&
\hfil#\hfil&\hfil#\hfil&\hfil#\hfil&\hfil#\hfil&\hfil#\hfil&
\hfil#\hfil&\hfil#\hfil&\hfil#\hfil&\hfil#\hfil&\hfil#\hfil&
\hfil#\hfil&\hfil#\hfil&
\hfil#\hfil\tabskip\z@skip\cr
|``|&|''|&|--|&|---|&|?||`|&|!||`|&|\$|&|\#|&|\&|&|\%|&
  |\ae|&|\AE|&|\oe|&|\OE|&|\aa|&|\AA|&|\ss|&|\o|&|\O|\cr
``&''&--&---&?`&!`&\$&\#&\&&\%&\ae&\AE&\oe&\OE&\aa&\AA&\ss&\o&\O\cr}
\sep
\halign to\hsize{\strut\hfil#\hfil\tabskip\z@ plus10pt&
\hfil#\hfil&\hfil#\hfil&\hfil#\hfil&\hfil#\hfil&\hfil#\hfil&
\hfil#\hfil&\hfil#\hfil&\hfil#\hfil&\hfil#\hfil&\hfil#\hfil&
\hfil#\hfil&\hfil#\hfil&
\hfil#\hfil\tabskip\z@skip\cr
|\`a|&|\'e|&|\^o|&|\"u|&|\=y|&|\~n|&|\.p|&|\u\i|&
  |\v s|&|\H\j|&|\t\i u|&|\b k|&|\c c|&|\d h|\cr
\`a&\'e&\^o&\"u&\=y&\~n&\.p&\u\i&\v s&\H\j&\t\i u&\b k&\c c&\d h\cr}
\sep
\halign to\hsize{\strut\hfil#\hfil\tabskip\z@ plus10pt&
\hfil#\hfil&\hfil#\hfil&\hfil#\hfil&\hfil#\hfil&\hfil#\hfil&
\hfil#\hfil&\hfil#\hfil&\hfil#\hfil&\hfil#\hfil&
\hfil#\hfil\tabskip\z@skip\cr
|\l|&|\L|&|\dag|&|\ddag|&|\S|&|\P|&|{\it\$|&
  |\&}|&|\copyright|&|\TeX|&|\dots|\cr
\l&\L&\dag&\ddag&\S&\P&\phantom{\tt////}\it\$&
  \it\&\phantom{\tt/}&\copyright&\TeX&\dots\cr}
\sep
\line{\strut Line break controls:\hfil
  |\break|\hfil|\nobreak|\hfil|\allowbreak|\hfil|\hbox{unbreakable}|}
\line{\strut|dis\-cre\-tion\-ary hy\-phens|\hfil|virgule\slash breakpoint|}
\sep
\settabs 2\columns
\+\strut Breakable horizontal spaces:& Unbreakable horizontal spaces:\cr
\+|\|\] \ normal interword space& |~| \ normal interword space\cr
\+|\enskip| \ this\enskip much& |\enspace     | \ this\enspace much\cr
\+|\quad  | \ this\quad much&   |\thinspace   | \ this\thinspace much\cr
\+|\qquad | \ this\qquad much&  |\negthinspace| \ this\negthinspace much\cr
\+\strut|\hskip| \<arbitrary dimen>& |\kern| \<arbitrary dimen>\cr
\sep
\smallskip
\+\strut Vertical spaces:\hfill
  |\smallskip| $\vcenter{\hrule width2em\smallskip\hrule}$\hfill&
  |\medskip| $\vcenter{\hrule width3em\medskip\hrule}$\hfill
  |\bigskip| $\vcenter{\hrule width4em\bigskip\hrule}$&\cr
\smallskip
\sep
\line{\strut Page break controls:\hfil|\eject|\hfil|\supereject|\hfil
  |\nobreak|\hfil|\goodbreak|\hfil|\filbreak|}
\+\strut Vertical spaces and good breakpoints:&
  |\smallbreak|\hfill|\medbreak|\hfill|\bigbreak|&\cr
\sep
\settabs 4 \columns
\hbox{\strut|\settabs 4 \columns|}
\hbox{|\+Here's an example&of\hfill some &tabbing:&\hrulefill&\cr|}
\+Here's an example&of\hfill some &tabbing:&\hrulefill&\cr
\+\hfill|\hrulefill |&\hrulefill&\hfill|\dotfill |&\dotfill&\cr
\+\hfill|\leftarrowfill |&\leftarrowfill&
  \hfill|\rightarrowfill |&\rightarrowfill&\cr
\+\hfill|\upbracefill |&\upbracefill&
  \hfill|\downbracefill |&\downbracefill&\cr
\smallskip
\line{\strut More general alignments use |\halign|, |\valign|, |\omit|,
  |\span|, and |\multispan|.}
\sep
\line{\strut Examples of the principal conventions for text layout
  appear on the next page.}
\sep\unskip
\kern 1pt
\hrule height .61803pt
\eject
\begingroup\parindent\z@ \obeylines \baselineskip12pt plus 1pt
|% This test file generates the output shown on the opposite page.|
|% It's a bit complex because it tries to illustrate lots of stuff.|
|% TeX ignores commentary (like this) that follows a `%' sign.|
| |
|% First the standard output style is changed slightly:|
^|\hsize||=29pc % The lines in this book are 29 picas wide.|
^|\vsize||=42pc % The page body is 42 picas (not counting footlines).|
^|\footline||={\tenrm Footline\quad\dotfill\quad Page \folio}|
^|\pageno||=1009 % This is the starting page number (don't ask why).|
|% See Chapter 23 for the way to make other page format changes via|
|% \hoffset, \voffset, \nopagenumbers, \headline, or \raggedbottom.|
| |
^|\topglue|| 1in % This makes an inch of blank space (1in=2.54cm).|
^|\centerline||{\bf A Bold, Centered Title}|
^|\smallskip|| % This puts a little extra space after the title line.|
^|\rightline||{\it avec un sous-titre \`a la fran\c caise}|
|% Now we use \beginsection to introduce part 1 of the document.|
^|\beginsection|| 1. Plain \TeX nology % The next line must be blank!|
| |^^{blank line}
|The first paragraph of a new section is not indented.|
^|\TeX||\ recognizes the end of a paragraph when it comes to a blank|
|line in your manuscript file. % or to a `\par': see below.|^^|\par|
| |^^|\footnote|^^|\tt|^^|\char|
|Subsequent paragraphs {\it are\/} indented.\footnote*{The amount|
|    of indentation can be changed by changing a parameter called|
|{\tt\char`\\parindent}. Turn the page for a summary of \TeX's most|
|important parameters.} (See?) The computer breaks a paragraph's|
|text into lines in an interesting way---see reference~[1]---and h%|
|     yphenates words automatically     when necessary.|^^{percent}
| |
^|\midinsert|| % This begins inserted material, e.g., a figure.|
|\narrower\narrower % This brings the margins in (see Chapter 14).|
^|\noindent|| |^|\llap||{``}If there hadn't been room for this material on|
|the present page, it would have been inserted on the next one.''|
|\endinsert % This ends the insertion and the effect of \narrower.|
| |
^|\proclaim|| Theorem T. The typesetting of $math$ is discussed in|
|Chapters 16--19, and math symbols are summarized in Appendix~F.|
| |
|\beginsection 2. Bibliography\par % `\par' acts like a blank line.|
^|\frenchspacing|| % (Chapter 12 recommends this for bibliographies.)|
^|\item||{[1]} D.~E. Knuth and M.~F. Plass, ``Breaking paragraphs|^^{Plass}
|into lines,'' {\sl Softw. pract. exp. \bf11} (1981), 1119--1184.|^^{Knuth}
^|\bye|| % This is the way the file ends, not with a \bang but a \bye.|
\eject\endgroup
{\parindent 20pt
\topglue 1in % This makes an inch of blank space (1in=2.54cm).
\centerline{\bf A Bold, Centered Title}
\smallskip % This puts a little extra space between lines here.
\rightline{\it avec un sous-titre \`a la fran\c caise}
% Now we use \beginsection to introduce part 1 of the document.
\beginsection 1. Plain \TeX nology % The next line must be blank!

The first paragraph of a new section is not indented.
\TeX\ recognizes the end of a paragraph when it comes to a blank
line in your manuscript file. % or to a `\par': see below.

Subsequent paragraphs {\it are\/} indented.*{\spacefactor=3000}
(See?) The computer breaks a paragraph's
text into lines in an interesting way---see reference~[1]---and h%
     yphenates words automatically     when necessary.

\midinsert % This begins inserted material, e.g., a figure.
\narrower\narrower % This brings the margins in (see Chapter 14).
\noindent \llap{``}If there hadn't been room for this material on
the present page, it would have been inserted on the next one.''
\endinsert % This ends the insertion and the effect of \narrower.

\proclaim Theorem T. The typesetting of $math$ is discussed in
Chapters 16--19, and math symbols are summarized in Appendix~F.

\beginsection 2. Bibliography\par% `\par' is just like blank line.
\frenchspacing % (Chapter 12 recommends this for bibliographies.)
\item{[1]} D.~E. Knuth and M.~F. Plass, ``Breaking paragraphs
into lines,'' {\sl Softw. pract. exp. \bf11} (1981), 1119--1184.
\vfill
\hrule width 2in \kern 2.6pt
\textindent{*}\strut
The amount of indentation can be changed by changing a parameter called
{\tt\char`\\parindent}. Turn the page for a summary of \TeX's most
important parameters.\par
\baselineskip 24pt
\line{Footline\quad\dotfill\quad Page 1009}
\eject}

The preceding example illustrates most of the basic things
that you can do directly with plain \TeX, but it does not provide an
exhaustive list. Thus, it uses |\centerline| and |\rightline|,
but not |\leftline| or |\line|; it uses |\midinsert|, but not
|\topinsert| or |\pageinsert|; it uses |\smallskip|, but not
|\medskip| or |\bigskip|; it uses |\llap| but not |\rlap|, |\item|
but not |\itemitem|, |\topglue| but not |\hglue|. It does not
illustrate |\raggedright| setting of paragraphs; it does not use
^|\obeylines| or ^|\obeyspaces| to shut off \TeX's automatic
formatting. ^^{as is, see obeylines, obeyspaces}
All such control sequences are explained later in this appendix, and
further information can be found by looking them up in the index.
The main purpose of the example is to serve as a reminder of the
repertoire of possibilities.

Most of the control sequences used in the example are defined by
macros of plain \TeX\ format, but three of them are ^{primitive}, i.e.,
built~in: `|\par|' (end of paragraph), `|\noindent|' (beginning of
non-indented paragraph), and `|\/|' (italic correction). The example
also assigns values to two of \TeX's primitive parameters, namely
|\hsize| and |\vsize|. \TeX\ has scores of parameters, all of which
are listed in Chapter~24, but only a few of them are of special
concern to the majority of \TeX\ users. Here are examples of how
you might want to give new values to the most important ^{parameters}
other than |\hsize| and |\vsize|:
\par\begingroup\nobreak\medskip\parindent\z@ \obeylines
^|\tolerance||=500|\quad(\TeX\ will tolerate lines whose badness is %
  rated 500 or less.)
^|\looseness||=1|\quad(The next paragraph will be one line longer than %
  usual.)
^|\parindent||=4mm|\quad(Paragraphs will be indented by four millimeters.)
^|\hoffset||=1.5in|\quad(All output will be shifted right by one and a %
  half inches.)
^|\voffset||=24pt|\quad(All output will be shifted down by 24 points.)
^|\baselineskip||=11pt plus.1pt|\quad(Baselines will be $11\pt$ apart, %
  or a bit more.)
^|\parskip||=3pt plus1pt minus.5pt|\quad(Extra space will %
  precede each paragraph.)
\endgroup\medbreak\noindent
Plain \TeX\ uses |\parindent| also to control the amount of
indentation provided by ^|\item|, ^|\itemitem|, and ^|\narrower|.

\danger The remainder of this appendix is devoted to the details of the
plain \TeX\ format, which is a set of macros that come with normal
implementations of \TeX. These macros serve three basic purposes:
\ (1)~They make \TeX\ usable, because \TeX's primitive capabilities
operate at a very low level. A~``virgin'' \TeX\ system that has no
macros is like a newborn baby that has an immense amount to learn about
the real world; but it is capable of learning fast. \ (2)~The plain
\TeX\ macros provide a basis for more elaborate and powerful formats
tailored to individual tastes and applications. You can do a lot with
plain \TeX, but pretty soon you'll want to do even more. \ (3)~The macros
also serve to illustrate how additional formats can be designed.

\ninepoint
Somewhere in your computer system you should be able to find a file
called ^|plain.tex| that contains exactly what has been preloaded into
the running \TeX\ system that you use. Our purpose in the rest of this
appendix will be to discuss the contents of |plain.tex|. However, we
will not include a verbatim description, because some parts of that file
are too boring, and because the actual macros have been ``^{optimized}''
^^{efficiency}
with respect to memory space and running time. Unoptimized versions of the
macros are easier for humans to understand, so we shall deal with those;
|plain.tex| contains equivalent constructions that work better on a machine.

So here's the plan for the rest of Appendix~B\null: We shall go through
the contents of |plain.tex|, interspersing an edited transcription of
that file with comments about noteworthy details. When we come to
macros whose usage has not yet been explained---for example, somehow
|\vglue| and |\beginsection| never made it into Chapters 1 through~27---we
shall consider them from a user's viewpoint; but most of the time we shall
be addressing the issues from the standpoint of a macro designer.

%\subsection The code tables. A format's first duty is to establish
%^|\catcode| values. This is necessary because, for example, a |\def|
%command can't be used until there are characters like |{| and~|}| of
%categories 1 and~2. The ^|INITEX| program (which reads |plain.tex|
%so that \TeX\ can be initialized) begins without knowing any grouping
%characters; hence |plain.tex| starts out as follows:
\subsection 编码表. A format's first duty is to establish
^|\catcode| values. This is necessary because, for example, a |\def|
command can't be used until there are characters like |{| and~|}| of
categories 1 and~2. The ^|INITEX| program (which reads |plain.tex|
so that \TeX\ can be initialized) begins without knowing any grouping
characters; hence |plain.tex| starts out as follows:
\beginlines
|% This is the plain TeX format that's described in The TeXbook.|
|% N.B.: A version number is defined at the very end of this file;|
|%       please change that number whenever the file is modified!|
|% And don't modify the file under any circumstances.|
\smallbreak
|\catcode`\{=1 % left brace is begin-group character|
|\catcode`\}=2 % right brace is end-group character|
|\catcode`\$=3 % dollar sign is math shift|
|\catcode`\&=4 % ampersand is alignment tab|
|\catcode`\#=6 % hash mark is macro parameter character|
|\catcode`\^=7 \catcode`\^^K=7 % circumflex and uparrow for superscripts|
|\catcode`\_=8 \catcode`\^^A=8 % underline and downarrow for subscripts|
|\catcode`\^^I=10 % ASCII tab is treated as a blank space|
|\chardef\active=13 \catcode`\~=\active % tilde is active|
|\catcode`\^^L=\active \outer\def^^L{\par} % ASCII form-feed is \outer\par|
\smallskip
|\message{Preloading the plain format: codes,}|
\endlines
These instructions set up the nonstandard characters
|^^K| and |^^A| for ^{superscripts} and ^{subscripts}, in addition to~|^|
and~|_|, so that people with extended character sets can use
{\tentex\char'13} and {\tentex\char1} as recommended in Appendix~C\null.
^^{uparrow char} ^^{downarrow char}
Furthermore |^^I| (^{ASCII} ^\<tab>) is given category~10 (space);
and |^^L| (ASCII
\<formfeed>) becomes an active character that will detect runaways on
files that have been divided into ``^{file pages}'' by ^\<formfeed>
characters. The control sequence ^|\active| is defined to yield the
constant~13; this is the one category code that seems to deserve a
symbolic name, in view of its frequent use in constructing
special-purpose macros.

When |INITEX| begins, category 12 (other) has been assigned to all~256
possible characters, except that the 52~letters |A...Z| and |a...z| are
category~11 (letter), and a few other assignments equivalent to the
following have been made:
\begintt
\catcode `\\ =0   \catcode`\  =10   \catcode `\% =14
\catcode`\^^@=9   \catcode`\^^M=5   \catcode`\^^?=15
\endtt
Thus `|\|' is already an escape character, ^^{backslash}
`\]' is a space, ^^\<space>
and `|%|' is available for comments on the first line of the file;
ASCII ^\<null> is ignored, ASCII ^\<return> is an end-of-line character,
and ASCII ^\<delete> is invalid.

The ^|\message| command shown above prints a progress report on the terminal
when |plain.tex| is being input by |INITEX|. Later on comes
`|\message{registers,}|' and several other messages, but we won't mention
them specifically. The terminal will eventually display something like
this when initialization is complete:
\begintt
** plain
(plain.tex Preloading the plain format: codes, registers,
parameters, fonts, more fonts, macros, math definitions,
output routines, hyphenation (hyphen.tex))
* \dump
Beginning to dump on file plain.fmt
\endtt
followed by a variety of statistics about what fonts were loaded, etc.
If you want to make a new format |super.tex| that adds more features
to |plain.tex|, it's best not to make a new file containing all the
plain stuff, or even to |\input plain|; just type `|&plain super|' in
response to |INITEX|'s ^|**| prompt, ^^{ampersand} ^^|fmt|
to input |plain.fmt| at high speed.

After the opening |\message|, |plain.tex| goes on to define a control
sequence ^|\dospecials| that lists all the characters whose catcodes should
probably be changed to~12 (other) when copying things verbatim:
\beginlines
|\def\dospecials{\do\ \do\\\do\{\do\}\do\$\do\&%|
|  \do\#\do\^\do\^^K\do\_\do\^^A\do\%\do\~}|
\endlines
(Appendix E illustrates how to use |\dospecials|.) \ The ASCII codes for
\<null>, \<tab>, \<linefeed>, \<formfeed>, \<return>, and \<delete>
have not been included in the list.

At this point |plain.tex| completes its initialization of category codes
by setting |\catcode`\@=11|, thereby making the character `|@|' behave
temporarily like a letter. The command |\catcode`\@=12| will appear later,
hence ^{at-sign characters} will act just like ordinary punctuation
marks when \TeX\ is running. The idea is to make it easy for plain \TeX\
to have ^{private control sequences} that cannot be redefined by ordinary
users; all such control sequences will have at least one `|@|' in
their names.

The next job is to set up the ^|\mathcode| table:
\beginlines
|\mathcode`\^^@="2201 \mathcode`\^^A="3223 \mathcode`\^^B="010B|
|\mathcode`\^^C="010C \mathcode`\^^D="225E \mathcode`\^^E="023A|
|\mathcode`\^^F="3232 \mathcode`\^^G="0119 \mathcode`\^^H="0115|
|\mathcode`\^^I="010D \mathcode`\^^J="010E \mathcode`\^^K="3222|
|\mathcode`\^^L="2206 \mathcode`\^^M="2208 \mathcode`\^^N="0231|
|\mathcode`\^^O="0140 \mathcode`\^^P="321A \mathcode`\^^Q="321B|
|\mathcode`\^^R="225C \mathcode`\^^S="225B \mathcode`\^^T="0238|
|\mathcode`\^^U="0239 \mathcode`\^^V="220A \mathcode`\^^W="3224|
|\mathcode`\^^X="3220 \mathcode`\^^Y="3221 \mathcode`\^^Z="8000|
|\mathcode`\^^[="2205 \mathcode`\^^\="3214 \mathcode`\^^]="3215|
|\mathcode`\^^^="3211 \mathcode`\^^_="225F \mathcode`\^^?="1273|
|\mathcode`\ ="8000   \mathcode`\!="5021   \mathcode`\'="8000|
|\mathcode`\(="4028   \mathcode`\)="5029   \mathcode`\*="2203|
|\mathcode`\+="202B   \mathcode`\,="613B   \mathcode`\-="2200|
|\mathcode`\.="013A   \mathcode`\/="013D   \mathcode`\:="303A|
|\mathcode`\;="603B   \mathcode`\<="313C   \mathcode`\=="303D|
|\mathcode`\>="313E   \mathcode`\?="503F   \mathcode`\[="405B|
|\mathcode`\\="026E   \mathcode`\]="505D   \mathcode`\_="8000|
|\mathcode`\{="4266   \mathcode`\|\||="026A   \mathcode`\}="5267|
\endlines
A mathcode is relevant only when the corresponding category code is
11 or~12; therefore many of these codes will rarely be looked at. For
example, the math code for |^^M| specifies the character |\oplus|,
but it's hard to imagine a user who would want |^^M| (ASCII ^\<return>)
to produce an~$\oplus$ sign in the middle of a math formula, since plain
\TeX\ appends |^^M| to the end of every line of input. The math codes
have been set up here, however, to be entirely consistent with the
extended character set presented in Appendix~C and the Computer Modern
fonts described in Appendix~F\null. |INITEX| has done the rest of the
work, as far as mathcodes are concerned: It has set |\mathcode|$\,x=
x+\hex{7000}$ for each of the ten digits $x={}$|`0| to |`9|;
|\mathcode|$\,x=x+\hex{7100}$ for each of the 52~letters; and
|\mathcode|$\,x=x$ for all other values of~$x$.

There's no need to change the ^|\uccode| and ^|\lccode| tables. |INITEX|
has made |\uccode`X=`X|, |\uccode`x=`X|, |\lccode`X=`x|, |\lccode`x=`x|,
and it has made similar assignments for all other letters. The codes are
zero for all nonletters. These tables are used by \TeX's ^|\uppercase|
and |\lowercase| operations, and the hyphenation algorithm also looks
at |\lccode| (see Appendix~H\null). Changes should be made only in format
packages that set \TeX\ up for languages with more than~26 letters
(see Chapter~8).

Next comes the ^|\sfcode| table, which |INITEX| has initialized entirely
to 1000, except that |\sfcode`X=999| for each of the 26 uppercase letters.
Some characters are made ``transparent'' by setting
\beginlines
|\sfcode`\)=0 \sfcode`\'=0 \sfcode`\]=0 % won't change the space factor|
\endlines
and the |\nonfrenchspacing| macro will be used later to change the sfcodes
of special punctuation marks. \ (Chapter~12 explains what an |\sfcode| does.)

The last code table is called ^|\delcode|, and again it's necessary to
change only a few values. |INITEX| has made all delimiter codes equal to
$-1$, which means that no characters are recognized as delimiters in
formulas. But there's an exception: The value |\delcode`\.=0| has been
prespecified, so that `|.|' ^^{period} stands for a ``^{null delimiter}.''
\ (See Chapter~17.) \ Plain format sets up the following nine values,
based on the delimiters available in Computer Modern:
\beginlines
|\delcode`\(="028300    \delcode`\/="02F30E    \delcode`\)="029301|
|\delcode`\[="05B302    \delcode`\|\||="26A30C    \delcode`\]="05D303|
|\delcode`\<="26830A    \delcode`\\="26E30F    \delcode`\>="26930B|
\endlines
It's important to note that |\delcode`\{| and |\delcode`\}| have been left
equal to~$-1$. If those codes were set to certain values, a user would
^^{left brace} ^^{right brace} be able to type, e.g.,
`|\big{|' to get a big left brace; but it would be a big mistake.
The reason is that ^{braces} are used for grouping, when supplying
arguments to macros; all sorts of strange things can happen if you try
to use them both as math delimiters and group delimiters.

At this point the |plain.tex| file contains several definitions
\beginlines
|\chardef\@ne=1 \chardef\tw@=2 \chardef\thr@@=3 \chardef\sixt@@n=16|
|\chardef\@cclv=255 \mathchardef\@cclvi=256|
|\mathchardef\@m=1000 \mathchardef\@M=10000 \mathchardef\@MM=20000|
\endlines
which allow ``private'' control sequences ^|\@ne|, |\tw@|, etc.,
to be used as abbreviations for commonly used constants 1,~2,~\dots; this
convention makes \TeX\ run a little faster, ^^{optimization} ^^{efficiency}
and it means that the macros will consume slightly less memory space.
The usage of these abbreviations will not, however, be shown below
unless necessary; we shall pretend that `|1|\]' appears instead of\/ |\@ne|,
`|10000|\]' instead of\/ |\@M|, and so on, since that makes the programs
more readable. \ (Notice that the long form of\/ |\@ne| is `|1|\]' including
a space, because \TeX\ looks for and removes a space following a constant.)

%\subsection Allocation of registers. The second major part of the
%|plain.tex| file provides a foundation on which systems of independently
%developed macros can coexist peacefully without interfering in their
%usage of ^{registers}. The idea is that macro writers should abide by the
%^^{macro conventions}
%following ground rules: (1)~Registers numbered 0~to~9 are always free for
%temporary ``^{scratch}'' use, but their values are always assumed to be
%clobbered whenever any other macro might get into control.  \ (This applies
%to registers like |\dimen0|, |\toks0|, |\skip1|, |\box3|, etc.; but \TeX\
%has already reserved |\count0| through |\count9| for page number
%identification.) \ (2)~The registers |\count255|, |\dimen255|, and
%|\skip255| are freely available in the same way.
%\ (3)~All assignments to the scratch registers whose numbers are
%1,~3, 5, 7, and~9 should be ^|\global|; all assignments to the
%other scratch registers (0,~2, 4, 6, 8,~255) should be non-|\global|.
%\ (This prevents the phenomenon of ``^{save stack buildup}'' discussed
%in Chapter~27.)
%\ (4)~Furthermore, it's possible to
%use any register in a group, if you ensure that \TeX's grouping
%mechanism will restore the register when you're done with the group, and
%if you are certain that other macros will not make global assignments
%to that register when you need it. \ (5)~But when a register is used
%by several macros, or over long spans of time, it should be allocated
%by |\newcount|, |\newdimen|, |\newbox|, etc. \ (6)~Similar remarks
%apply to ^{input/output streams} used by ^|\read| and ^|\write|, to math
%^{families} used by ^|\fam|, to sets of hyphenation rules used by
%^|\language|, and to insertions (which require
%^|\box|, ^|\count|, ^|\dimen|, and ^|\skip| registers all having the
%same number).
\subsection 分配寄存器. The second major part of the
|plain.tex| file provides a foundation on which systems of independently
developed macros can coexist peacefully without interfering in their
usage of ^{registers}. The idea is that macro writers should abide by the
^^{macro conventions}
following ground rules: (1)~Registers numbered 0~to~9 are always free for
temporary ``^{scratch}'' use, but their values are always assumed to be
clobbered whenever any other macro might get into control.  \ (This applies
to registers like |\dimen0|, |\toks0|, |\skip1|, |\box3|, etc.; but \TeX\
has already reserved |\count0| through |\count9| for page number
identification.) \ (2)~The registers |\count255|, |\dimen255|, and
|\skip255| are freely available in the same way.
\ (3)~All assignments to the scratch registers whose numbers are
1,~3, 5, 7, and~9 should be ^|\global|; all assignments to the
other scratch registers (0,~2, 4, 6, 8,~255) should be non-|\global|.
\ (This prevents the phenomenon of ``^{save stack buildup}'' discussed
in Chapter~27.)
\ (4)~Furthermore, it's possible to
use any register in a group, if you ensure that \TeX's grouping
mechanism will restore the register when you're done with the group, and
if you are certain that other macros will not make global assignments
to that register when you need it. \ (5)~But when a register is used
by several macros, or over long spans of time, it should be allocated
by |\newcount|, |\newdimen|, |\newbox|, etc. \ (6)~Similar remarks
apply to ^{input/output streams} used by ^|\read| and ^|\write|, to math
^{families} used by ^|\fam|, to sets of hyphenation rules used by
^|\language|, and to insertions (which require
^|\box|, ^|\count|, ^|\dimen|, and ^|\skip| registers all having the
same number).

Some handy abbreviations are introduced at this point so that the macros
below will have easy access to scratch registers:
\beginlines
|\countdef\count@=255  \toksdef\toks@=0       \skipdef\skip@=0|
|\dimendef\dimen@=0    \dimendef\dimen@i=1    \dimendef\dimen@ii=2|
\endlines

Here now are the macros that provide allocation for quantities of more
permanent value. These macros use registers
|\count10| through |\count20| to hold the numbers that were allocated
most recently; for example, if\/ |\newdimen| has just reserved |\dimen15|,
the value of\/ |\count11| will be~15. However, the rest of the world
is not supposed to ``know'' that |\count11| has anything to do with
|\dimen| registers. There's a special counter called ^|\allocationnumber|
that will be equal to the most recently allocated number, after every
|\newcount|, |\newdimen|, \dots, |\newinsert| operation; macro packages
are supposed to refer to |\allocationnumber| if they want to find out what
number was allocated. The inside story of how allocation is actually performed
should be irrelevant when the allocation macros are used at a higher level;
you mustn't assume that |plain.tex| really does allocation in any
particular way.
\beginlines
|\count10=22 % this counter allocates \count registers 23, 24, 25, ...|
|\count11=9  % this counter allocates \dimen registers 10, 11, 12, ...|
|\count12=9  % this counter allocates \skip registers 10, 11, 12, ...|
|\count13=9  % this counter allocates \muskip registers 10, 11, 12, ...|
|\count14=9  % this counter allocates \box registers 10, 11, 12, ...|
|\count15=9  % this counter allocates \toks registers 10, 11, 12, ...|
|\count16=-1 % this counter allocates input streams 0, 1, 2, ...|
|\count17=-1 % this counter allocates output streams 0, 1, 2, ...|
|\count18=3  % this counter allocates math families 4, 5, 6, ...|
|\count19=0  % this counter allocates language codes 1, 2, 3, ...|
|\count20=255 % this counter allocates insertions 254, 253, 252, ...|
|\countdef\insc@unt=20 % nickname for the insertion counter|
|\countdef\allocationnumber=21 % the most recent allocation|
|\countdef|^|\m@ne||=22 \m@ne=-1 % a handy constant|
|\def|^|\wlog||{\immediate\write-1} % this will write on log file (only)|
\smallbreak
|\outer\def|^|\newcount||{\alloc@0\count\countdef\insc@unt}|
|\outer\def|^|\newdimen||{\alloc@1\dimen\dimendef\insc@unt}|
|\outer\def|^|\newskip||{\alloc@2\skip\skipdef\insc@unt}|
|\outer\def|^|\newmuskip||{\alloc@3\muskip\muskipdef\@cclvi}|
|\outer\def|^|\newbox||{\alloc@4\box\chardef\insc@unt}|
|\let\newtoks=\relax % this allows plain.tex to be read in twice|
|\outer\def\newhelp#1#2{\newtoks#1#1=\expandafter{\csname#2\endcsname}}|
|\outer\def|^|\newtoks||{\alloc@5\toks\toksdef\@cclvi}|
|\outer\def|^|\newread||{\alloc@6\read\chardef\sixt@@n}|
|\outer\def|^|\newwrite||{\alloc@7\write\chardef\sixt@@n}|
|\outer\def|^|\newfam||{\alloc@8\fam\chardef\sixt@@n}|
|\outer\def|^|\newlanguage||{\alloc@9\language\chardef\@cclvi}|
\smallbreak
|\def\alloc@#1#2#3#4#5{\global\advance\count1#1 by 1|\parbreak%
|  \ch@ck#1#4#2% make sure there's still room|\parbreak%
|  \allocationnumber=\count1#1   \global#3#5=\allocationnumber|\parbreak%
|  \wlog{\string#5=\string#2\the\allocationnumber}}|
\smallbreak
|\outer\def|^|\newinsert||#1{\global\advance\insc@unt by-1|\parbreak%
|  \ch@ck0\insc@unt\count \ch@ck1\insc@unt\dimen|\parbreak%
|  \ch@ck2\insc@unt\skip  \ch@ck4\insc@unt\box|\parbreak%
|  \allocationnumber=\insc@unt|\parbreak%
|  \global\chardef#1=\allocationnumber|\parbreak%
|  \wlog{\string#1=\string\insert\the\allocationnumber}}|
\smallbreak
|\def\ch@ck#1#2#3{\ifnum\count1#1<#2%|\parbreak%
|  \else\errmessage{No room for a new #3}\fi}|^^|No room|
\endlines
The `|\alloc@|' macro does most of the work of allocation. It puts
a message like `|\maxdimen=\dimen10|' into the log file after
|\newdimen| has allocated a place for the |\dimen| register that
will be called |\maxdimen|; such information might be useful when
difficult macros are being ^{debugged}.

A ^|\newhelp| macro has been provided to aid in creating home-made help
texts: You can say, e.g., |\newhelp\helpout{This is a help message.}|, and
then give the command `^|\errhelp||=\helpout|' just before issuing an
^|\errmessage|. This method of creating help texts makes efficient
use of \TeX's ^{memory}, because it puts the text into a control sequence
name where it doesn't take up space that is needed for tokens.

The |plain| file now goes ahead and allocates registers for important constants:
\beginlines
|\newdimen\maxdimen \maxdimen=16383.99999pt|
|\newskip\hideskip \hideskip=-1000pt plus1fill|
|\newskip\centering \centering=0pt plus 1000pt minus 1000pt|
|\newdimen\p@ \p@=1pt % this saves macro space and time|
|\newdimen\z@ \z@=0pt| % likewise
|\newskip\z@skip \z@skip=0pt plus0pt minus0pt|
|\newbox\voidb@x % permanently void box register|
\endlines
The control sequence ^|\maxdimen| stands for the largest permissible
\<dimen>. Alignment macros that appear below will make use of special glue
values called ^|\hideskip| and ^|\centering|.  {\sl N.B.: These three constants
must not be changed under any circumstances\/}; you should either ignore
them completely or just use them and enjoy them.
In fact, the next four constant registers (^|\p@|, ^|\z@|, ^|\z@skip|,
and ^|\voidb@x|) have been given private names so that they are untouchable.
The control sequence |\p@| is used several dozen times as an abbreviation
for `|pt |', and |\z@| is used quite often to stand for either `|0pt |' or
`|0 |'; the use of such abbreviations saves almost 10\% of the space
needed to store the tokens in plain \TeX's macros.  But we shall stick to
^^{optimization}
the unabbreviated forms below, so that the macros are more readable.

A different sort of allocation comes next:
^^|\string|^^|\escapechar|^^|\uppercase|^^|\iftrue|^^|\iffalse|
\beginlines
|\outer\def\newif#1{\count@=\escapechar \escapechar=-1|\parbreak%
|  |^|\expandafter||\expandafter\expandafter|\parbreak%
|   |^|\edef||\@if#1{true}{\let|^|\noexpand||#1=\noexpand\iftrue}%|\parbreak%
|  \expandafter\expandafter\expandafter|\parbreak%
|   \edef\@if#1{false}{\let\noexpand#1=\noexpand\iffalse}%|\parbreak%
|  \@if#1{false}\escapechar=\count@} % the condition starts out false|\parbreak%
|\def\@if#1#2{|^|\csname||\expandafter\if@\string#1#2\endcsname}|\parbreak%
|{\uccode`1=`i \uccode`2=`f \uppercase{\gdef\if@12{}}} % `if' is required|
\endlines
For example, the command |\newif\ifalpha| creates a trio of control
sequences called |\alphatrue|, |\alphafalse|, and |\ifalpha|
(see Chapter~20).

%\subsection Parameters. |INITEX| sets almost all of the numeric registers
%and parameters equal to zero; it makes all of the token registers and parameters
%empty; and it makes all of the box registers void. But there are a few
%^^{parameters, default values} ^^{default values of parameters}
%exceptions: ^|\mag| is set initially to~1000, ^|\tolerance| to~10000,
%^|\maxdeadcycles| to~25, ^|\hangafter| to~1, ^|\escapechar| to~|`\\|, and
%^|\endlinechar| to~|`\^^M|.  Plain \TeX\ assigns new parameter values as
%follows:
\subsection 参数. |INITEX| sets almost all of the numeric registers
and parameters equal to zero; it makes all of the token registers and parameters
empty; and it makes all of the box registers void. But there are a few
^^{parameters, default values} ^^{default values of parameters}
exceptions: ^|\mag| is set initially to~1000, ^|\tolerance| to~10000,
^|\maxdeadcycles| to~25, ^|\hangafter| to~1, ^|\escapechar| to~|`\\|, and
^|\endlinechar| to~|`\^^M|.  Plain \TeX\ assigns new parameter values as
follows:
\beginlines
^|\pretolerance||=100 |^|\tolerance||=200  |^|\hbadness||=1000 |%
  ^|\vbadness||=1000|
^|\linepenalty||=10 |^|\hyphenpenalty||=50 |^|\exhyphenpenalty||=50|
^|\binoppenalty||=700 |^|\relpenalty||=500|
^|\clubpenalty||=150 |^|\widowpenalty||=150 |^|\displaywidowpenalty||=50|
^|\brokenpenalty||=100 |^|\predisplaypenalty||=10000|
^|\doublehyphendemerits||=10000 |^|\finalhyphendemerits||=5000 |%
  ^|\adjdemerits||=10000|
^|\tracinglostchars||=1 |^|\uchyph||=1 |^|\delimiterfactor||=901|
^|\defaulthyphenchar||=`\- |^|\defaultskewchar||=-1 |%
  ^|\newlinechar||=-1|
^|\showboxbreadth||=5 |^|\showboxdepth||=3 |^|\errorcontextlines||=5|
\smallbreak
^|\hfuzz||=0.1pt |^|\vfuzz||=0.1pt |^|\overfullrule||=5pt|
^|\hsize||=6.5in |^|\vsize||=8.9in |^|\parindent||=20pt|
^|\maxdepth||=4pt |^|\splitmaxdepth||=\maxdimen |^|\boxmaxdepth||=\maxdimen|
^|\delimitershortfall||=5pt |^|\nulldelimiterspace||=1.2pt |%
  ^|\scriptspace||=0.5pt|
\smallbreak
^|\parskip||=0pt plus 1pt|
^|\abovedisplayskip||=12pt plus 3pt minus 9pt|
^|\abovedisplayshortskip||=0pt plus 3pt|
^|\belowdisplayskip||=12pt plus 3pt minus 9pt|
^|\belowdisplayshortskip||=7pt plus 3pt minus 4pt|
^|\topskip||=10pt |^|\splittopskip||=10pt|
^|\parfillskip||=0pt plus 1fil|
\smallbreak
^|\thinmuskip||=3mu|
^|\medmuskip||=4mu plus 2mu minus 4mu|
^|\thickmuskip||=5mu plus 5mu|
\endlines

(Some parameters are set by \TeX\ itself as it runs, so it is inappropriate
to initialize them: ^|\time|, ^|\day|, ^|\month|, and ^|\year| are established
at the beginning of a job; ^|\outputpenalty| is given a value when an
|\output| routine is invoked; ^|\predisplaysize|, ^|\displaywidth|,
and ^|\displayindent| get values just before a display is processed;
and the values ^|\looseness||=0|, ^|\hangindent||=0pt|, ^|\hangafter||=1|,
^|\parshape||=0| are assigned at the end of a paragraph and when \TeX\ enters
internal vertical mode.)

The parameters ^|\baselineskip|, ^|\lineskip|, and ^|\lineskiplimit| have not
been initialized here, but a macro called ^|\normalbaselines| is defined below;
this macro sets |\baselineskip=\normalbaselineskip|,
|\lineskip=\normallineskip|, and |\lineskiplimit=\normallineskiplimit|.
An indirect approach like this has been used so that several different
type sizes may be handled, as illustrated in Appendix~E\null.  Plain \TeX\
deals exclusively with $10\pt$ type, but it supports extension to other styles.

Some ``^{pseudo parameters}'' come next. These quantities behave just like
internal parameters of \TeX, and users are free to change them in the same
way, but they are part of the plain \TeX\ format
rather than primitives of the language. ^^{parameters, pseudo}
\beginlines
|\newskip|^|\smallskipamount|| % the amount of a \smallskip|
|  \smallskipamount=3pt plus1pt minus1pt|
|\newskip|^|\medskipamount|| % the amount of a \medskip|
|  \medskipamount=6pt plus2pt minus2pt|
|\newskip|^|\bigskipamount|| % the amount of a \bigskip|
|  \bigskipamount=12pt plus4pt minus4pt|
|\newskip|^|\normalbaselineskip|| % normal value of \baselineskip|
|  \normalbaselineskip=12pt|
|\newskip|^|\normallineskip|| % normal value of \lineskip|
|  \normallineskip=1pt|
|\newdimen|^|\normallineskiplimit|| % normal value of \lineskiplimit|
|  \normallineskiplimit=0pt|
|\newdimen|^|\jot|| % unit of measure for opening up displays|
|  \jot=3pt|
|\newcount|^|\interdisplaylinepenalty|| % interline penalty in \displaylines|
|  \interdisplaylinepenalty=100|
|\newcount|^|\interfootnotelinepenalty|| % interline penalty in footnotes|
|  \interfootnotelinepenalty=100|
\endlines

%\subsection Font information. Now |plain.tex| brings in the data that \TeX\
%needs to know about how to typeset lots of characters in lots of different
%fonts. First the ^|\magstep| macros are defined, to support font scaling:
\subsection 字体信息. Now |plain.tex| brings in the data that \TeX\
needs to know about how to typeset lots of characters in lots of different
fonts. First the ^|\magstep| macros are defined, to support font scaling:
\beginlines
|\def|^|\magstephalf||{1095 }|
|\def\magstep#1{\ifcase#1 1000\or|
|  1200\or 1440\or 1728\or 2074\or 2488\fi\relax}|
\weakendlines
(Incidentally, |\magstep| doesn't use ^|\multiply| to compute values,
since it is supposed to expand to a ^\<number> enroute to \TeX's
``stomach''; ^^{anatomy} |\multiply| wouldn't work, because
it is an assignment command, performed only in the stomach.)

One of the main things that distinguishes one format from another
is the fact that each format gives \TeX\ the necessary knowledge about
a certain family of typefaces. In this case the ^{Computer Modern} types
described in Appendix~F are taken as a basis, although there is a provision
for incorporating other styles.
\beginlines
|\font|^|\tenrm||=cmr10        \font\preloaded=cmr9     \font\preloaded=cmr8|
\nobreak
|\font\sevenrm=cmr7       \font\preloaded=cmr6     \font\fiverm=cmr5|
\smallskip
|\font\teni=cmmi10        \font\preloaded=cmmi9    \font\preloaded=cmmi8|
\nobreak
|\font\seveni=cmmi7       \font\preloaded=cmmi6    \font\fivei=cmmi5|
\smallbreak
|\font\tensy=cmsy10       \font\preloaded=cmsy9    \font\preloaded=cmsy8|
\nobreak
|\font\sevensy=cmsy7      \font\preloaded=cmsy6    \font\fivesy=cmsy5|
\nobreak\smallskip
|\font\tenex=cmex10|
\smallbreak
|\font\tenbf=cmbx10       \font\preloaded=cmbx9    \font\preloaded=cmbx8|
\nobreak
|\font\sevenbf=cmbx7      \font\preloaded=cmbx6    \font\fivebf=cmbx5|
\smallskip
|\font\tensl=cmsl10       \font\preloaded=cmsl9    \font\preloaded=cmsl8|
\nobreak
|\font\tentt=cmtt10       \font\preloaded=cmtt9    \font\preloaded=cmtt8|
\nobreak
|\font\tenit=cmti10       \font\preloaded=cmti9    \font\preloaded=cmti8|
\nobreak
|\font\preloaded=cmss10   \font\preloaded=cmssq8|
\nobreak
|\font\preloaded=cmssi10  \font\preloaded=cmssqi8|
\smallskip
|\font\preloaded=cmr7 scaled \magstep4 % for titles|
\nobreak
|\font\preloaded=cmtt10 scaled \magstep2|
\nobreak
|\font\preloaded=cmssbx10 scaled \magstep2|
\smallskip
|% Additional \preloaded fonts can be specified here.|
|% (And those that were \preloaded above can be eliminated.)|
|\let\preloaded=\undefined % preloaded fonts must be declared anew later.|
\endlines
Notice that most of the fonts have been called ^|\preloaded|; but the
control sequence |\preloaded| is made undefined at the very end, so those
fonts cannot be used directly. There are two reasons for this strange
approach: First, it is desirable to keep the total number of fonts
of plain \TeX\ relatively small, because plain \TeX\ is a sort of
standard format; it shouldn't cost much for someone to acquire all the
fonts of plain \TeX\ in addition to those he really wants. Second, it
is desirable on many computer systems to preload the information for
most of the fonts that people will actually be using, since this saves
a lot of machine time. The |\preloaded| font
information goes into \TeX's memory, where it will come alive instantly
if the user defines the corresponding |\font| again. For example, the
book format in Appendix~E says `|\font\ninerm=cmr9|'; after that
assignment has been obeyed, the control sequence |\ninerm| will
identify the |cmr9| font, whose information does not have to be loaded again.

The exact number and nature of fonts that are preloaded is unimportant;
the only important thing needed for standardization between machines
is that sixteen basic fonts (|cmr10|, |cmr7|, \dots,~|cmti10|) should
actually be loaded. The |plain.tex| files used on different machines
can be expected to differ widely with respect to preloaded fonts,
since the choice of how many fonts to preload and the selection of the
most important fonts depends on local conditions.
For example, at the author's university it is useful to preload a
font that contains the Stanford seal, but that particular font is not
very popular at Berkeley.

Most of these fonts have the default values of\/ ^|\hyphenchar| and
^|\skewchar|, namely |`-| and |-1|; but the math italic and math symbol fonts
have special |\skewchar| values, which are defined next:
\beginlines
|\skewchar\teni='177 \skewchar\seveni='177 \skewchar\fivei='177|
|\skewchar\tensy='60 \skewchar\sevensy='60 \skewchar\fivesy='60|
\endlines

Once the fonts are loaded, they are also grouped into families for
use in math setting, and shorthand names like ^|\rm| and~^|\it| are defined:
\beginlines
|\textfont0=\tenrm \scriptfont0=\sevenrm |^|\scriptscriptfont||0=\fiverm|
\nobreak
|\def\rm{\fam0 \tenrm}|
^|\textfont||1=\teni  |^|\scriptfont||1=\seveni  \scriptscriptfont1=\fivei|
\nobreak
|\def|^|\mit||{\fam1 } \def|^|\oldstyle||{\fam1 \teni}|
|\textfont2=\tensy \scriptfont2=\sevensy \scriptscriptfont2=\fivesy|
\nobreak
|\def|^|\cal||{\fam2 }|
|\textfont3=\tenex \scriptfont3=\tenex   \scriptscriptfont3=\tenex|
\smallbreak
|\newfam\itfam \def\it{\fam\itfam\tenit} \textfont\itfam=\tenit|
|\newfam\slfam \def|^|\sl||{\fam\slfam\tensl} \textfont\slfam=\tensl|
|\newfam\bffam \def|^|\bf||{\fam\bffam\tenbf} \textfont\bffam=\tenbf|
\nobreak
|  \scriptfont\bffam=\sevenbf   \scriptscriptfont\bffam=\fivebf|
^|\newfam||\ttfam \def|^|\tt||{\fam\ttfam\tentt} \textfont\ttfam=\tentt|
\endlines

%\subsection Macros for text. The fifth part of |plain.tex| introduces
%macros that do basic formatting unrelated to mathematics. First come
%some macros that were promised above:
\subsection 用于文本的宏. The fifth part of |plain.tex| introduces
macros that do basic formatting unrelated to mathematics. First come
some macros that were promised above:
\beginlines
|\def|^|\frenchspacing||{\sfcode`\.=1000 \sfcode`\?=1000 \sfcode`\!=1000|
\nobreak
|  \sfcode`\:=1000 \sfcode`\;=1000 |^|\sfcode||`\,=1000 }|
|\def|^|\nonfrenchspacing||{\sfcode`\.=3000 \sfcode`\?=3000 \sfcode`\!=3000|
\nobreak
|  \sfcode`\:=2000 \sfcode`\;=1500 \sfcode`\,=1250 }|
|\def|^|\normalbaselines||{\lineskip=\normallineskip|
\nobreak
|  \baselineskip=\normalbaselineskip \lineskiplimit=\normallineskiplimit}|
\endlines

The next macros are simple but vital. ^^{control tab}^^{control return}
^^{control space} First |\|\<tab> and |\|\<return> are defined so that they
expand to |\|\<space>; this helps to prevent confusion, since all three
cases look identical when displayed on most computer terminals.
Then the macros ^|\lq|, ^|\rq|, ^|\lbrack|, and ^|\rbrack| are defined,
for people who have difficulty typing quotation marks and/or brackets. The
control sequences ^|\endgraf|
and ^|\endline| are made equivalent to \TeX's primitive ^|\par| and~^|\cr|
operations, since it is often useful to redefine the meanings of\/ |\par|
and~|\cr| themselves. Then come the definitions of\/ ^|\space| (a blank space),
^|\empty| (a~list of no tokens), and ^|\null| (an~empty hbox). Finally,
^|\bgroup| and ^|\egroup| are made to provide ``implicit'' grouping characters
that turn out to be especially useful in macro definitions. \ (See Chapters
24--26 and Appendix~D for information about ^{implicit characters}.)
\beginlines
|\def\^^I{\ } \def\^^M{\ }|
|\def\lq{`} \def\rq{'} \def\lbrack{[} \def\rbrack{]}|
|\let\endgraf=\par \let\endline=\cr|
|\def\space{ } \def\empty{} \def\null{\hbox{}}|
|\let\bgroup={ \let\egroup=}|
\endlines

Something a bit tricky comes up now in the definitions of\/ ^|\obeyspaces|
and ^|\obeylines|, since \TeX\ is only ``half obedient'' while these
definitions are half finished:
\beginlines
|\def\obeyspaces{\catcode`\ =\active}|
|{\obeyspaces\global\let =\space}|
|{\catcode`\^^M=\active % these lines must end with `%'|
\nobreak
|  \gdef\obeylines{\catcode`\^^M=\active \let^^M=\par}%|
\nobreak
|  \global\let^^M=\par} % this is in case ^^M appears in a \write|
\endlines
The |\obeylines| macro says `|\let^^M=\par|' instead of `|\def^^M{\par}|'
because the |\let| technique allows constructions such as
`|\let\par=\cr| |\obeylines| ^|\halign||{...}|' in which ^|\cr|'s need
not be given within the alignment.

The ^|\loop||...|^|\repeat| macro provides for iterative operations
as illustrated at the end of Chapter~20. In this macro and several others,
the control sequence `^|\next|' is given a temporary value that is not
going to be needed later; thus, |\next| acts like a ``^{scratch control
sequence}.''
\beginlines
|\def\loop#1\repeat{\def\body{#1}\iterate}|
|\def\iterate{\body \let\next=\iterate \else\let\next=\relax\fi \next}|
|\let\repeat=\fi % this makes \loop...\if...\repeat skippable|
\endlines

Spacing is the next concern. The macros ^|\enskip|, ^|\quad|, and ^|\qquad|
provide spaces that are legitimate breakpoints within a paragraph;
^|\enspace|, ^|\thinspace|, and ^|\negthinspace| produce space that cannot
cause a break (although the space will disappear if it occurs just
next to certain kinds of breaks). All six of these spaces are relative
to the current font. You can get horizontal space that
never disappears by saying `^|\hglue|\<glue>'; this space is able to
stretch or shrink. Similarly, there's a vertical analog,
`^|\vglue|\<glue>'.
The ^|\nointerlineskip| macro suppresses interline glue that would
ordinarily be inserted before the next box in vertical mode; this is a
``one shot'' macro, but ^|\offinterlineskip| is more drastic---it sets
things up so that future ^{interline glue} will be present, but zero.
There also are macros for potentially breakable vertical spaces:
^|\smallskip|, ^|\medskip|, and ^|\bigskip|.^^|\topglue|
\beginlines
|\def\enskip{\hskip.5em\relax}    \def\enspace{\kern.5em }|
|\def\quad{\hskip1em\relax}       \def\qquad{\hskip2em\relax}|
|\def\thinspace{\kern .16667em }  \def\negthinspace{\kern-.16667em }|
\smallbreak
|\def\hglue{|^|\afterassignment||\hgl@\skip@=}|
|\def\hgl@{\leavevmode \count@=\spacefactor \vrule width0pt|
|  \nobreak\hskip\skip@ \spacefactor=\count@}|
|\def\vglue{\afterassignment\vgl@\skip@=}|
|\def\vgl@{\par \dimen@=\prevdepth \hrule height0pt|
|  \nobreak\vskip\skip@ \prevdepth=\dimen@}|
|\def\topglue{\nointerlineskip \vglue-\topskip \vglue} % for top of page|
\smallbreak
|\def\nointerlineskip{\prevdepth=-1000pt }|
|\def\offinterlineskip{\baselineskip=-1000pt|
|  \lineskip=0pt \lineskiplimit=\maxdimen}|
\smallbreak
|\def\smallskip{\vskip\smallskipamount}|
|\def\medskip{\vskip\medskipamount}|
|\def\bigskip{\vskip\bigskipamount}|
\endlines

Speaking of breakpoints, the following macros introduce ^{penalty} markers
that make breaking less, or more, desirable. The ^|\break|, ^|\nobreak|,
and ^|\allowbreak| macros are intended for use in any mode; the |~|~(tie)
and ^|\slash| (hyphen-like~`/') macros are intended for horizontal mode.
The others are intended only for vertical mode, i.e., between paragraphs,
so they begin with |\par|.
\beginlines
|\def\break{\penalty-10000 } \def\nobreak{\penalty10000 }|
|\def\allowbreak{\penalty0 }|
|\def~{\penalty10000\ }|
|\def\slash{/\penalty\exhyphenpenalty}|
\smallbreak
|\def|^|\filbreak||{\par\vfil\penalty-200\vfilneg}|
|\def|^|\goodbreak||{\par\penalty-500 }|
|\def|^|\eject||{\par\penalty-10000 }|
|\def|^|\supereject||{\par\penalty-20000 }|
\smallbreak
|\def|^|\removelastskip||{\ifdim\lastskip=0pt \else\vskip-\lastskip\fi}|
|\def|^|\smallbreak||{\par \ifdim\lastskip<\smallskipamount|\parbreak%
|  \removelastskip \penalty-50 \smallskip \fi}|
|\def|^|\medbreak||{\par \ifdim\lastskip<\medskipamount|\parbreak%
|  \removelastskip \penalty-100 \medskip \fi}|
|\def|^|\bigbreak||{\par \ifdim\lastskip<\bigskipamount|\parbreak%
|  \removelastskip \penalty-200 \bigskip \fi}|
\endlines

Boxes are next: ^|\line|, ^|\leftline|, ^|\rightline|, and ^|\centerline|
produce boxes of the full line width, while ^|\llap| and ^|\rlap| make
boxes whose effective width is zero. The ^|\underbar| macro puts its
argument into an hbox with a straight line at a fixed distance under it.
\beginlines
|\def\line{\hbox to\hsize}|
|\def\leftline#1{\line{#1\hss}} \def\rightline#1{\line{\hss#1}}|
|\def\centerline#1{\line{\hss#1\hss}}|
\smallskip
|\def\llap#1{\hbox to 0pt{\hss#1}} \def\rlap#1{\hbox to 0pt{#1\hss}}|
\smallskip
|\def\m@th{\mathsurround=0pt }|
|\def\underbar#1{$\setbox0=\hbox{#1} \dp0=0pt|\parbreak%
|  \m@th \underline{\box0}$}|
\endlines
(Notice that |\underbar| uses math mode to do its job, although the
operation is essentially non-mathematical in nature. A few of the other
macros below use math mode in similar ways; thus, \TeX's mathematical
abilities prove to be useful even when no mathematical typesetting
is actually being done. A special control sequence ^|\m@th| is
used to ``turn off'' ^|\mathsurround| when such constructions are
being performed.)

\smallbreak
A ^|\strut| is implemented here as a rule of width zero, since this
takes minimum space and time in applications where numerous struts are
present.
\beginlines
|\newbox\strutbox|
|\setbox\strutbox=\hbox{\vrule height8.5pt depth3.5pt width0pt}|
|\def\strut{\relax\ifmmode\copy\strutbox\else\unhcopy\strutbox\fi}|
\weakendlines
The `^|\relax|' in this macro and in others below is necessary in case
|\strut| appears first in an alignment entry, because \TeX\ is in a
somewhat unpredictable mode at such times (see Chapter~22).

The ^|\ialign| macro provides for ^{alignments} when it is necessary to
be sure that ^|\tabskip| is initially zero. The ^|\hidewidth| macro can be
used essentially as |\hfill| in alignment entries that are permitted to
``stick out'' of their column. There's also ^|\multispan|, which permits
alignment entries to span one or more columns.
\beginlines
|\def\ialign{\everycr={}\tabskip=0pt \halign} % initialized \halign|
|\def\hidewidth{\hskip\hideskip}|
\smallbreak
|\newcount\mscount|
|\def\multispan#1{\omit \mscount=#1 \loop\ifnum\mscount>1 \sp@n\repeat}|
|\def\sp@n{\span\omit \advance\mscount by -1 }|
\endlines

Now we get to the ``^{tabbing}'' macros, which are more complicated than
anything else in plain \TeX. They keep track of the tab positions by
maintaining boxes full of empty boxes having the specified widths.
\ (The best way to understand these macros is probably to watch
them in action on simple examples, using |\tracingall|.)
\beginlines
|\newif\ifus@ \newif\if@cr|
|\newbox\tabs \newbox\tabsyet \newbox\tabsdone|
\smallbreak
|\def|^|\cleartabs||{\global\setbox\tabsyet=\null \setbox\tabs=\null}|
|\def|^|\settabs||{\setbox\tabs=\null \futurelet\next\sett@b}|
|\let\+=\relax % in case this file is being read in twice|
|\def\sett@b{\ifx\next\+ \let\next=\relax % turn off \outerness|\parbreak%
|    \def\next{\afterassignment\s@tt@b\let\next}%|\parbreak%
|  \else\let\next=\s@tcols\fi\next}|
|\def\s@tt@b{\let\next=\relax \us@false\m@ketabbox}|
|\outer\def|^|\+||{\tabalign} \def\tabalign{\us@true \m@ketabbox}|
|\def\s@tcols#1\columns{\count@=#1 \dimen@=\hsize|\parbreak%
|  \loop \ifnum\count@>0 \@nother \repeat}|
|\def\@nother{\dimen@ii=\dimen@ \divide\dimen@ii by\count@|\parbreak%
|  \setbox\tabs=\hbox{\hbox to\dimen@ii{}\unhbox\tabs}%|\parbreak%
|  \advance\dimen@ by-\dimen@ii \advance\count@ by -1 }|
\smallbreak
|\def\m@ketabbox{\begingroup|\parbreak%
|  \global\setbox\tabsyet=\copy\tabs \global\setbox\tabsdone=\null|\parbreak%
|  \def\cr{\@crtrue\crcr\egroup\egroup|\parbreak%
|    \ifus@ \unvbox0 |^|\lastbox||\fi \endgroup|\parbreak%
|    \setbox\tabs=\hbox{\unhbox\tabsyet|^|\unhbox||\tabsdone}}%|\parbreak%
|  \setbox0=\vbox\bgroup\@crfalse \ialign\bgroup&\t@bbox##\t@bb@x\crcr}|
\smallbreak
|\def\t@bbox{\setbox0=\hbox\bgroup}|
|\def\t@bb@x{\if@cr\egroup % now \box0 holds the column|\parbreak%
|  \else\hss\egroup \global\setbox\tabsyet=\hbox{\unhbox\tabsyet|\parbreak%
|      \global\setbox1=\lastbox}% now \box1 holds its size|\parbreak%
|    \ifvoid1 \global\setbox1=\hbox to\wd0{}%|\parbreak%
|    \else\setbox0=\hbox to\wd1{\unhbox0}\fi|\parbreak%
|    \global\setbox\tabsdone=\hbox{\box1\unhbox\tabsdone}\fi|\parbreak%
|  \box0}|
\endlines
The macro |\+| has been declared `^|\outer|' here, so that \TeX\ will
be better able to detect runaway arguments and definitions (see Chapter~20).
A non-|\outer| version, called ^|\tabalign|, has also been provided
in case it is necessary to use |\+| in some ``inner'' place.
You can use |\tabalign| just like |\+|, except after ^|\settabs|.

\textindent{$\bullet$} Paragraph shapes of a limited but important kind
are provided by ^|\item|, ^|\itemitem|, and ^|\narrower|. There are also
two macros that haven't been mentioned before: \ (1)~^|\hang| causes
hanging indentation by the normal amount of\/ ^|\parindent|, after the
first line; thus, the entire paragraph will be indented by the same
amount (unless it began with ^|\noindent|). \ (2)~^|\textindent||{stuff}|
is like ^|\indent|, but it puts the `|stuff|' into the indentation, flush
right except for an en~space; it also removes spaces that might follow
the right brace in `|{stuff}|'. For example, the present paragraph
was typeset by the commands `|\textindent{$\bullet$} Paragraph shapes
...|'; the opening `P' occurs at the normal position for a paragraph's
first letter. ^^|\bullet| ^^|\ignorespaces|
\beginlines
|\def\hang{\hangindent\parindent}|
|\def\item{\par\hang\textindent}|
|\def\itemitem{\par\indent \hangindent2\parindent \textindent}|
|\def\textindent#1{\indent\llap{#1\enspace}\ignorespaces}|
|\def\narrower{\advance\leftskip by\parindent|\parbreak%
|  \advance\rightskip by\parindent}|
\endlines

The ^|\beginsection| macro is intended to mark the beginning of a
new major subdivision in a document; to use it, you say
`|\beginsection|\<section title>' followed by a blank line (or~|\par|).
The macro first emits glue and penalties, designed to start a new page
if the present page is nearly full; then it makes a ^|\bigskip| and
puts the section title flush left on a line by itself, in boldface type.
The section title is also displayed on the terminal.
After a ^|\smallskip|, with page break prohibited, a ^|\noindent|
command is given; this suppresses indentation in the next paragraph,
i.e., in the first paragraph of the new section.
\ (However, the next ``paragraph'' will be empty if vertical mode
material immediately follows the |\beginsection| command.) ^^|\message|
\beginlines
|\outer\def\beginsection#1\par{\vskip0pt plus.3\vsize\penalty-250|\parbreak%
|  \vskip0pt plus-.3\vsize\bigskip\vskip\parskip|\parbreak%
|  \message{#1}\leftline{\bf#1}\nobreak\smallskip\noindent}|
\endlines

Special statements in a mathematical paper are often called ^{theorems},
lemmas, definitions, axioms, postulates, remarks, corollaries, algorithms,
facts, conjectures, or some such things, and they generally are
given special typographic treatment. The ^|\proclaim| macro, which
was illustrated earlier in this appendix and also in Chapter~20,
puts the title of the proclamation in boldface, then sets the rest
of the paragraph in slanted type. The paragraph is followed by
something similar to ^|\medbreak|, except that the amount of penalty
is different so that page breaks are discouraged:
\beginlines
|\outer\def\proclaim #1. #2\par{\medbreak|\parbreak%
|  \noindent{\bf#1.\enspace}{\sl#2\par}%|\parbreak%
|  \ifdim\lastskip<\medskipamount \removelastskip\penalty55\medskip\fi}|
\endlines

^{Ragged-right setting} is initiated by restricting the spaces between
words to have a fixed width, and by putting variable space at the right
of each line. You should not call ^|\raggedright| until your text font
has already been specified; it is assumed that the ragged-right material
will not be in a variety of different sizes. \ (If this assumption is
not valid, a different approach should be used: ^|\fontdimen| parameters
3 and~4 of the fonts you will be using should be set to zero, by saying,
e.g., `|\fontdimen3\tenrm=0pt|'. These parameters specify the stretchability
and shrinkability of ^{interword spaces}.) \ ^^|\spaceskip| ^^|xspaceskip|
A special macro ^|\ttraggedright| should be used for ragged-right setting
in ^{typewriter type}, since the spaces between words are generally
bigger in that style. \ (Spaces are already unstretchable and
unshrinkable in font~|cmtt|.)
\beginlines
|\def\raggedright{\rightskip=0pt plus2em|\parbreak%
|  \spaceskip=.3333em \xspaceskip=.5em\relax}|
|\def\ttraggedright{\tt\rightskip=0pt plus2em\relax}|
\endlines

Now we come to special symbols and accents, which depend primarily on
the characters available in the Computer Modern fonts. Different
constructions will be necessary if other styles of type are used.
When a symbol is built up by forming a box, the ^|\leavevmode| macro
is called first; this starts a new paragraph, if \TeX\ is in vertical
mode, but does nothing if \TeX\ is in horizontal mode or math mode.
\beginlines
|\chardef\%=`\% \chardef\&=`\& \chardef\#=`\# \chardef\$=`\$|%
  ^^{control percent}^^{control ampersand}^^{control hash}^^{control dollar}
|\chardef|^|\ss||="19|
|\chardef|^|\ae||="1A \chardef|^|\oe||="1B \chardef|^|\o||="1C|
|\chardef|^|\AE||="1D \chardef|^|\OE||="1E \chardef|^|\O||="1F|
|\chardef|^|\i||="10 \chardef|^|\j||="11 % dotless letters|
|\def|^|\aa||{\accent'27a} \def|^|\l||{\char'40l}|
\smallbreak
|\def\leavevmode{\unhbox\voidb@x} % begins a paragraph, if necessary|
|\def\_{\leavevmode \kern.06em \vbox{\hrule width0.3em}}|^^{control underline}
|\def|^|\L||{\leavevmode\setbox0=\hbox{L}\hbox to\wd0{\hss\char'40L}}|
|\def|^|\AA||{\leavevmode\setbox0=\hbox{h}\dimen@=\ht0 |%
  |\advance\dimen@ by-1ex|\parbreak%
|  \rlap{\raise.67\dimen@\hbox{\char'27}}A}|
\smallbreak
|\def|^|\mathhexbox||#1#2#3{\leavevmode|\parbreak%
|  \hbox{$\m@th \mathchar"#1#2#3$}}|
|\def|^|\dag||{\mathhexbox279}  \def|^|\ddag||{\mathhexbox27A}|
|\def|^|\S||{\mathhexbox278} \def|^|\P||{\mathhexbox27B}|
\smallbreak
|\def|^|\oalign||#1{\leavevmode\vtop{\baselineskip0pt \lineskip.25ex|\parbreak%
|  \ialign{##\crcr#1\crcr}}} % put characters over each other|
|\def|^|\ooalign||{\lineskiplimit-\maxdimen \oalign}|
|\def|^|\d||#1{\oalign{#1\crcr\hidewidth.\hidewidth}}|
|\def|^|\b||#1{\oalign{#1\crcr\hidewidth|\parbreak%
|    \vbox to.2ex{\hbox{\char'26}\vss}\hidewidth}}|
|\def|^|\c||#1{\setbox0=\hbox{#1}\ifdim\ht0=1ex \accent'30 #1%|\parbreak%
|  \else{\ooalign{\hidewidth\char'30\hidewidth\crcr\unhbox0}}\fi}|
|\def|^|\copyright||{{\ooalign|\parbreak%
|    {\hfil\raise.07ex\hbox{c}\hfil\crcr\mathhexbox20D}}}|
\smallbreak
|\def|^|\dots||{\relax\ifmmode\ldots\else$\m@th \ldots\,$\fi}|
|\def|^|\TeX||{T\kern-.1667em \lower.5ex\hbox{E}\kern-.125em X}|
\smallbreak
|\def|^|\`||#1{{\accent"12 #1}} \def|^|\'||#1{{\accent"13 #1}}|
|\def|^|\v||#1{{\accent"14 #1}} \def|^|\u||#1{{\accent"15 #1}}|
^^{esc hat}|\def|^|\=||#1{{\accent"16 #1}} \def\^#1{{\accent"5E #1}}|
|\def|^|\.||#1{{\accent"5F #1}} \def|^|\H||#1{{\accent"7D #1}}|
^^{esc tilde}|\def\~||#1{{\accent"7E #1}} \def|^|\"||#1{{\accent"7F #1}}|
|\def|^|\t||#1{{\edef\next{\the\font}\the\textfont1\accent"7F\next#1}}|
\endlines

At this point three alternative control-symbol ^{accents} are defined,
suitable for keyboards with extended character sets (cf.~Appendix~C):
\beginlines
|\let\^^_=\v       \let\^^S=\u       \let\^^D=\^|
\endlines

Various ways to fill space with ^{leaders} are provided next.
\beginlines
|\def|^|\hrulefill||{\leaders\hrule\hfill}|
|\def|^|\dotfill||{\cleaders\hbox{$\m@th \mkern1.5mu . \mkern1.5mu$}\hfill}|
|\def|^|\rightarrowfill||{$\m@th \mathord- \mkern-6mu|\parbreak%
|  \cleaders\hbox{$\mkern-2mu \mathord- \mkern-2mu$}\hfill|\parbreak%
|  \mkern-6mu \mathord\rightarrow$}|
|\def|^|\leftarrowfill|%
  |{$\m@th \mathord\leftarrow \mkern-6mu|\parbreak%
|  \cleaders\hbox{$\mkern-2mu \mathord- \mkern-2mu$}\hfill|\parbreak%
|  \mkern-6mu \mathord-$}|
\smallbreak
|\mathchardef\braceld="37A \mathchardef\bracerd="37B|
|\mathchardef\bracelu="37C \mathchardef\braceru="37D|
|\def\upbracefill{$\m@th|\parbreak%
|  \bracelu\leaders\vrule\hfill\bracerd|\parbreak%
|  \braceld\leaders\vrule\hfill\braceru$}|
|\def\downbracefill{$\m@th|\parbreak%
|  \braceld\leaders\vrule\hfill\braceru|\parbreak%
|  \bracelu\leaders\vrule\hfill\bracerd$}|
\endlines
The ^|\upbracefill| and ^|\downbracefill| macros have restricted usage:
they must appear {\sl all by themselves\/} in an hbox or an alignment entry,
except for horizontal spacing.

\smallbreak
Finally, the fifth section of |plain.tex| closes by defining ^|\bye|:
\beginlines
^|\outer||\def\bye{\par\vfill\supereject\end} % the recommended way to stop|
\endlines

%\subsection Macros for math.  The sixth section of |plain.tex| is the
%longest; but it will suffice to give only excerpts here, because most
%of it is simply a tedious listing of special symbols together with their
%font locations, and the same information appears in Appendix~F.
\subsection 用于数学的宏.  The sixth section of |plain.tex| is the
longest; but it will suffice to give only excerpts here, because most
of it is simply a tedious listing of special symbols together with their
font locations, and the same information appears in Appendix~F.

Some rudimentary things come first: The control sequences ^|\sp| and ^|\sb|
are provided for people who can't easily type |^| and |_|; there are
four control symbols that provide spacing corrections; a ``^{discretionary
times sign}'' ^|\*| is defined; and then there's an interesting set of
macros that convert |f'''| into |f^{\prime\prime\prime}|:
\beginlines
|\let\sp=^ \let\sb=_        {\catcode`\_=\active \global\let_=\_}|\parbreak%
|\def|^|\,||{\mskip\thinmuskip}  \def|^|\!||{\mskip-\thinmuskip}|\parbreak%
|\def|^|\>||{\mskip\medmuskip}   \def|^|\;||{\mskip\thickmuskip}|
|\def\*{\discretionary{\thinspace\the\textfont2\char2}{}{}}|
|{\catcode`\^^Z=\active \gdef^^Z{\not=}} % ^^Z is like \ne in math|
\smallbreak
|{\catcode`\'=\active \gdef'{^\bgroup\prim@s}}|\parbreak%
|\def\prim@s{\prime\futurelet\next\pr@m@s}|\parbreak%
|\def\pr@m@s{\ifx'\next\let\nxt\pr@@@s|%
  | \else\ifx^\next\let\nxt\pr@@@t|\parbreak%
|  \else\let\nxt\egroup\fi\fi \nxt}|\parbreak%
|\def\pr@@@s#1{\prim@s} \def\pr@@@t#1#2{#2\egroup}|
\endlines

The next job is to define ^{Greek letters} and other symbols of type ^{Ord}.
Uppercase Greek letters are assigned hexadecimal codes of the form \hex{7xxx},
so that they will change families when ^|\fam| changes.
Three dots `$\,\cdots\,$' are used here and below to indicate that
additional symbols, having similar definitions, are listed in Appendix~F.
\beginlines
|\mathchardef|^|\alpha||="010B|\qquad$\cdots$\qquad|\mathchardef\omega="0121|
|\mathchardef|^|\Gamma||="7000|\qquad$\cdots$\qquad|\mathchardef\Omega="700A|
|\mathchardef|^|\aleph||="0240|\qquad$\cdots$\qquad%
  |\mathchardef\spadesuit="027F|
|\def|^|\hbar||{{\mathchar'26\mkern-9muh}}|
|\def|^|\surd||{{\mathchar"1270}}|
|\def|^|\angle||{{\vbox{\ialign{$\m@th\scriptstyle##$\crcr|\parbreak%
|      \not\mathrel{\mkern14mu}\crcr \noalign{\nointerlineskip}|\parbreak%
|      \mkern2.5mu\leaders\hrule height.34pt\hfill\mkern2.5mu\crcr}}}}|
\endlines

Large operators are assigned hexadecimal codes of the form \hex{1xxx}:
\beginlines
|\mathchardef\smallint="1273|
|\mathchardef|^|\sum||="1350|\qquad$\cdots$\qquad|\mathchardef\biguplus="1355|
|\mathchardef\intop="1352  \def|^|\int||{\intop\nolimits}|
|\mathchardef\ointop="1348 \def\oint{\ointop\nolimits}|
\endlines
Integral signs get special treatment so that their limits won't be set
above and below.

\smallskip
Binary operations are next; nothing exciting here.
\beginlines
|\mathchardef\pm="2206|\qquad$\cdots$\qquad|\mathchardef\amalg="2271|
\endlines

Relations are also fairly straightforward, except for the ones that are
constructed from other characters. The ^|\mapstochar| is a character
`$\mapstochar\mskip5mu$' of width zero that is quite useless by itself,
but it combines with right arrows to make ^|\mapsto| `$\mapsto$' and
^|\longmapsto| `$\longmapsto$'. Similarly, ^|\not| is a relation character of
width zero that puts a slash over the character that follows.  When two
relations are adjacent in a math formula, \TeX\ puts no space between them.
\beginlines
|\mathchardef\leq="3214|\qquad$\cdots$\qquad|\mathchardef\perp="323F|
|\def|^|\joinrel||{\mathrel{\mkern-3mu}}|
|\def|^|\relbar||{\mathrel{\smash-}} \def|^|\Relbar||{\mathrel=}|
|\def|^|\longrightarrow||{\relbar\joinrel\rightarrow}|\parbreak%
|\def|^|\Longrightarrow||{\Relbar\joinrel\Rightarrow}|
|\def|^|\longleftarrow||{\leftarrow\joinrel\relbar}|\parbreak%
|\def|^|\Longleftarrow||{\Leftarrow\joinrel\Relbar}|
|\def|^|\longleftrightarrow||{\leftarrow\joinrel\rightarrow}|\parbreak%
|\def|^|\Longleftrightarrow||{\Leftarrow\joinrel\Rightarrow}|
|\mathchardef\mapstochar="322F \def\mapsto{\mapstochar\rightarrow}|\parbreak%
|\def\longmapsto{\mapstochar\longrightarrow}|
|\mathchardef\lhook="312C \def|^|\hookrightarrow||{\lhook\joinrel\rightarrow}|
|\mathchardef\rhook="312D \def|^|\hookleftarrow||{\leftarrow\joinrel\rhook}|
\smallbreak
|\def|^|\neq||{\not=}   \def|^|\models||{\mathrel|\||\joinrel=}|
|\def|^|\bowtie||{\mathrel\triangleright\joinrel\mathrel\triangleleft}|
\weakendlines

After defining characters ^|\ldotp| and ^|\cdotp| that act as math
punctuation, it is easy to define ^|\ldots| and ^|\cdots| macros that
\vadjust{\penalty-50} % fairly good breakpoint (on July 27, 1983)
give the proper spacing in most circumstances. Vertical and diagonal
dots (^|\vdots| and ^|\ddots|) are also provided here:
\beginlines
|\mathchardef\ldotp="613A\mathchardef\cdotp="6201\mathchardef\colon="603A|
|\def\ldots{\mathinner{\ldotp\ldotp\ldotp}}|
|\def\cdots{\mathinner{\cdotp\cdotp\cdotp}}|
|\def\vdots{\vbox{\baselineskip=4pt \lineskiplimit=0pt|
|    \kern6pt \hbox{.}\hbox{.}\hbox{.}}}|
|\def\ddots{\mathinner{\mkern1mu\raise7pt\vbox{\kern7pt\hbox{.}}\mkern2mu|
|    \raise4pt\hbox{.}\mkern2mu\raise1pt\hbox{.}\mkern1mu}}|
\endlines

Most of the math accents are handled entirely by the ^|\mathaccent|
primitive, but a few of the variable-width ones are constructed the
hard way:
\beginlines
|\def\acute{\mathaccent"7013 }|\qquad$\cdots$\qquad%
  |\def\ddot{\mathaccent"707F }|
|\def|^|\widetilde||{\mathaccent"0365 } \def|^|\widehat||{\mathaccent"0362 }|
|\def|^|\overrightarrow||#1{\vbox{\ialign{##\crcr|\parbreak%
|     \rightarrowfill\crcr\noalign{\kern-1pt\nointerlineskip}|\parbreak%
|     $\hfil\displaystyle{#1}\hfil$\crcr}}}|
|\def|^|\overleftarrow||#1{\vbox{\ialign{##\crcr|\parbreak%
|     \leftarrowfill\crcr\noalign{\kern-1pt\nointerlineskip}|\parbreak%
|     $\hfil\displaystyle{#1}\hfil$\crcr}}}|
|\def|^|\overbrace|%
  |#1{\mathop{\vbox{\ialign{##\crcr\noalign{\kern3pt}|\parbreak%
|     \downbracefill\crcr\noalign{\kern3pt\nointerlineskip}|\parbreak%
|     $\hfil\displaystyle{#1}\hfil$\crcr}}}|^|\limits||}|
|\def|^|\underbrace||#1{\mathop{\vtop{\ialign{##\crcr|\parbreak%
|     $\hfil\displaystyle{#1}\hfil$\crcr|%
  |\noalign{\kern3pt\nointerlineskip}|\parbreak%
|     \upbracefill\crcr\noalign{\kern3pt}}}}\limits}|
|\def|^|\skew||#1#2#3{{#2{#3\mkern#1mu}\mkern-#1mu}{}}|
\endlines

Now we come to 24 delimiters that can change their size: ^^|\delimiter|
\beginlines
|\def\langle{\delimiter"426830A }      \def\rangle{\delimiter"526930B }|
|\def\lbrace{\delimiter"4266308 }      \def\rbrace{\delimiter"5267309 }|
|\def\lceil{\delimiter"4264306 }       \def\rceil{\delimiter"5265307 }|
|\def\lfloor{\delimiter"4262304 }      \def\rfloor{\delimiter"5263305 }|
|\def\lgroup{\delimiter"400033A }      \def\rgroup{\delimiter"500033B }|
|\def\lmoustache{\delimiter"4000340 }  \def\rmoustache{\delimiter"5000341 }|%
  \kern-2pt
|\def\uparrow{\delimiter"3222378 }     \def\Uparrow{\delimiter"322A37E }|
|\def\downarrow{\delimiter"3223379 }   \def\Downarrow{\delimiter"322B37F }|
|\def\updownarrow{\delimiter"326C33F } \def\arrowvert{\delimiter"033C000 }|
|\def\Updownarrow{\delimiter"326D377 } \def\Arrowvert{\delimiter"033D000 }  |
|\def\vert{\delimiter"026A30C }        \def\Vert{\delimiter"026B30D }     |
|\def\backslash{\delimiter"026E30F }   \def\bracevert{\delimiter"033E000 }|
\endlines

The `^|\big||...|^|\Bigg|' macros produce specific sizes: ^^|\Big|^^|\bigg|
\beginlines
|\def\bigl{\mathopen\big} \def\bigm{\mathrel\big} \def\bigr{\mathclose\big}|
|\def\Bigl{\mathopen\Big} \def\Bigm{\mathrel\Big} \def\Bigr{\mathclose\Big}|
|\def\biggl{\mathopen\bigg}            \def\Biggl{\mathopen\Bigg}|
|\def\biggm{\mathrel\bigg}             \def\Biggm{\mathrel\Bigg}|
|\def\biggr{\mathclose\bigg}           \def\Biggr{\mathclose\Bigg}|
|\def\big#1{{\hbox{$\left#1\vbox to 8.5pt{}\right.\n@space$}}}|
|\def\Big#1{{\hbox{$\left#1\vbox to 11.5pt{}\right.\n@space$}}}|
|\def\bigg#1{{\hbox{$\left#1\vbox to 14.5pt{}\right.\n@space$}}}|
|\def\Bigg#1{{\hbox{$\left#1\vbox to 17.5pt{}\right.\n@space$}}}|
|\def\n@space{\nulldelimiterspace=0pt \m@th}|
\endlines

There are a few other simple abbreviations related to delimiters:
\beginlines
|\def|^|\choose||{\atopwithdelims()}|
|\def|^|\brack||{\atopwithdelims[]}|
|\def|^|\brace||{\atopwithdelims\{\}}|
|\def|^|\sqrt||{\radical"270370 }|
\endlines

And now we come to something more interesting. The ^|\mathpalette| operation
constructs a formula in all four styles; it is applied here in the
implementation of\/ ^|\phantom|, ^|\smash|, ^|\root|, and other
operations.
\beginlines
|\def\mathpalette#1#2{\mathchoice{#1\displaystyle{#2}}|\parbreak%
|  {#1\textstyle{#2}}{#1\scriptstyle{#2}}{#1\scriptscriptstyle{#2}}}|
\smallbreak
|\newbox\rootbox|
|\def\root#1\of{\setbox\rootbox=|\parbreak%
|  \hbox{$\m@th \scriptscriptstyle{#1}$}|\parbreak%
|  \mathpalette\r@@t}|
|\def\r@@t#1#2{\setbox0=\hbox{$\m@th #1\sqrt{#2}$}|\parbreak%
|  \dimen@=\ht0 \advance\dimen@ by-\dp0|\parbreak%
|  \mkern5mu \raise.6\dimen@\copy\rootbox \mkern-10mu \box0}|
\smallbreak
|\newif\ifv@ \newif\ifh@|
|\def|^|\vphantom||{\v@true\h@false\ph@nt}|
|\def|^|\hphantom||{\v@false\h@true\ph@nt}|
|\def\phantom{\v@true\h@true\ph@nt}|
|\def\ph@nt{\ifmmode\def\next{\mathpalette\mathph@nt}%|\parbreak%
|  \else\let\next=\makeph@nt\fi \next}|
|\def\makeph@nt#1{\setbox0=\hbox{#1}\finph@nt}|
|\def\mathph@nt#1#2{\setbox0=\hbox{$\m@th#1{#2}$}\finph@nt}|
|\def\finph@nt{\setbox2=\null \ifv@ \ht2=\ht0 \dp2=\dp0 \fi|\parbreak%
|  \ifh@ \wd2=\wd0 \fi \box2 }|
|\def\mathstrut{\vphantom(}|
\smallbreak
|\def\smash{\relax % \relax, in case this comes first in \halign|\parbreak%
|  \ifmmode\def\next{\mathpalette\mathsm@sh}\else\let\next\makesm@sh|\parbreak%
|  \fi \next}|
|\def\makesm@sh#1{\setbox0=\hbox{#1}\finsm@sh}|
|\def\mathsm@sh#1#2{\setbox0=\hbox{$\m@th#1{#2}$}\finsm@sh}|
|\def\finsm@sh{\ht0=0pt \dp0=0pt \box0 }|
\smallbreak
|\def|^|\cong||{\mathrel{\mathpalette\@vereq\sim}} % \sim over =|\parbreak%
|\def\@vereq#1#2{\lower.5pt\vbox{\baselineskip0pt \lineskip-.5pt|\parbreak%
|    \ialign{$\m@th#1\hfil##\hfil$\crcr#2\crcr=\crcr}}}|
|\def|^|\notin||{\mathrel{\mathpalette\c@ncel\in}}|\parbreak%
|\def\c@ncel#1#2{\ooalign{$\hfil#1\mkern1mu/\hfil$\crcr$#1#2$}}|\parbreak%
|\def|^|\rightleftharpoons||{\mathrel{\mathpalette\rlh@{}}}|\parbreak%
|\def\rlh@#1{\vcenter{\hbox{\ooalign{\raise2pt|\parbreak%
|          \hbox{$#1\rightharpoonup$}\crcr $#1\leftharpoondown$}}}}|
|\def|^|\buildrel||#1\over#2{\mathrel{\mathop{\kern0pt #2}\limits^{#1}}}|
|\def|^|\doteq||{\buildrel\textstyle.\over=}|
\endlines
These definitions illustrate how other built-up
symbol combinations could be defined to work in all four styles.

Alternate names are defined now:
\beginlines
|\let|^|\ne||=\neq        \let|^|\le||=\leq         \let|^|\ge||=\geq|
^^{esc vert}^^{esc lbrace}^^{esc rbrace}%
|\let\{=\lbrace      \let\|\||=\Vert         \let\}=\rbrace|
|\let|^|\to||=\rightarrow \let|^|\gets||=\leftarrow \let|^|\owns||=\ni|
|\let|^|\land||=\wedge    \let|^|\lor||=\vee        \let|^|\lnot||=\neg|
|\def|^|\iff||{\;\Longleftrightarrow\;}|
\endlines

The 32 common functions whose names generally appear in roman letters are listed
in Chapter~18. Only a few of the definitions need to be shown here:
^^|\arccos| ^^|\cos| ^^|\csc| ^^|\exp| ^^|\ker| ^^|\limsup| ^^|\min|
^^|\sinh| ^^|\arcsin| ^^|\cosh| ^^|\deg| ^^|\gcd| ^^|\lg| ^^|\ln| ^^|\Pr|
^^|\sup| ^^|\arctan| ^^|\cot| ^^|\det| ^^|\hom| ^^|\lim| ^^|\log| ^^|\sec|
^^|\tan| ^^|\arg| ^^|\coth| ^^|\dim| ^^|\inf| ^^|\liminf| ^^|\max|
^^|\sin| ^^|\tanh|
\beginlines
|\def\arccos{|^|\mathop||{\rm arccos}|^|\nolimits||}|
  \hskip100pt$\cdots$\qquad|\def\tanh{|^|\mathop||{\rm tanh}|^|\nolimits||}|
|\def\det{\mathop{\rm det}}|\qquad$\cdots$\qquad|\def\sup{\mathop{\rm sup}}|
|\def\liminf{\mathop{\rm lim\,inf}} \def\limsup{\mathop{\rm lim\,sup}}|
\smallskip
|\def|^|\bmod||{\mskip-\medmuskip \mkern5mu|\parbreak%
|  |^|\mathbin||{\rm mod} \penalty900 \mkern5mu \mskip-\medmuskip}|
|\def|^|\pmod||#1{\allowbreak \mkern18mu ({\rm mod}\,\,#1)}|
\endlines

The definition of\/ ^|\matrix| goes to some pains to ensure that two
$n$-rowed matrices will have the same height and the same depth, unless
at least one of their rows is unusually big. The definition of
^|\bordermatrix| is even more complicated, but it seems to work
reasonably well; it uses a constant ^|\p@renwd| that represents the
width of a big extensible left parenthesis.
\beginlines
|\def\matrix#1{\null\,|^|\vcenter||{\normalbaselines\m@th|\parbreak%
|    \ialign{\hfil$##$\hfil&&\quad\hfil$##$\hfil\crcr|\parbreak%
|      \mathstrut\crcr\noalign{\kern-\baselineskip}|\parbreak%
|      #1\crcr\mathstrut\crcr\noalign{\kern-\baselineskip}}}\,}|
\smallbreak
|\newdimen\p@renwd \setbox0=\hbox{\tenex B} \p@renwd=\wd0|
|\def\bordermatrix#1{\begingroup \m@th|\parbreak%
|  \setbox0=\vbox{\def\cr|%
  |{\crcr\noalign{\kern2pt\global\let\cr=\endline}}|\parbreak%
|      \ialign{$##$\hfil\kern2pt\kern\p@renwd|%
  |&\thinspace\hfil$##$\hfil|\parbreak%
|        &&\quad\hfil$##$\hfil\crcr|\parbreak%
|        \omit\strut\hfil\crcr\noalign{\kern-\baselineskip}|\parbreak%
|        #1\crcr\omit\strut\cr}}|\parbreak%
|  \setbox2=\vbox{\unvcopy0 \global\setbox1=\lastbox}|\parbreak%
|  \setbox2=\hbox{\unhbox1 \unskip \global\setbox1=\lastbox}|\parbreak%
|  \setbox2=\hbox{$\kern\wd1\kern-\p@renwd \left( \kern-\wd1|\parbreak%
|    \global\setbox1=\vbox{\box1\kern2pt}|\parbreak%
|    \vcenter{\kern-\ht1 \unvbox0 \kern-\baselineskip} \,\right)$}|\parbreak%
|  \null\;\vbox{\kern\ht1\box2}\endgroup}|
\endlines

The next macros are much simpler:
\beginlines
|\def|^|\cases||#1{\left\{\,\vcenter{\normalbaselines\m@th|\parbreak%
|    \ialign{$##\hfil$&\quad##\hfil\crcr#1\crcr}}\right.}|
|\def|^|\pmatrix||#1{\left( \matrix{#1} \right)}|
\endlines

Finally there are macros for displayed equations:
\beginlines
|\def|^|\openup||{\afterassignment\@penup\dimen@=}|\parbreak%
|\def\@penup{\advance\lineskip\dimen@|\parbreak%
|  \advance\baselineskip\dimen@ \advance\lineskiplimit\dimen@}|
|\def|^|\eqalign||#1{\null\,\vcenter{\openup1\jot \m@th|\parbreak%
|  \ialign{\strut\hfil$\displaystyle{##}$&$\displaystyle{{}##}$\hfil|\parbreak%
|      \crcr#1\crcr}}\,}|
\smallbreak
|\newif\ifdt@p|
|\def\displ@y{\global\dt@ptrue \openup1\jot \m@th|\parbreak%
|  |^|\everycr||{\noalign{\ifdt@p \global\dt@pfalse|\parbreak%
|      \vskip-\lineskiplimit \vskip\normallineskiplimit|\parbreak%
|      \else \penalty|^|\interdisplaylinepenalty|| \fi}}}|
|\def\@lign{\tabskip=0pt\everycr={}} % restore inside \displ@y|
|\def|^|\displaylines||#1{\displ@y|\parbreak%
|  \halign{\hbox to\displaywidth{|%
 |$\hfil\@lign\displaystyle##\hfil$}\crcr|\parbreak%
|    #1\crcr}}|
\smallbreak
|\def|^|\eqalignno||#1{\displ@y \tabskip=|^|\centering|\parbreak%
|  \halign to\displaywidth{\hfil$\@lign\displaystyle{##}$\tabskip=0pt|\parbreak%
|    &$\@lign\displaystyle{{}##}$\hfil\tabskip=\centering|\parbreak%
|    &\llap{$\@lign##$}\tabskip=0pt\crcr|\parbreak%
|    #1\crcr}}|
|\def|^|\leqalignno||#1{\displ@y \tabskip=\centering|\parbreak%
|  \halign to\displaywidth{\hfil$\@lign\displaystyle{##}$\tabskip=0pt|\parbreak%
|    &$\@lign\displaystyle{{}##}$\hfil\tabskip=\centering|\parbreak%
|    &\kern-\displaywidth\rlap{$\@lign##$}\tabskip=\displaywidth\crcr|\parbreak%
|    #1\crcr}}|
\endlines
The value of\/ |\lineskiplimit| is assumed to be |\normallineskiplimit|
plus the accumulated amount of ``opening up.'' Thus, the |\vskip|
instructions in |\displ@y| will compensate for the fact that the
first baseline of an alignment is separated by an opened-up baselineskip
from the last line preceding the display.

%\subsection Macros for output. The |plain.tex| file also contains the output
%routine described in Chapters 15 and~23. First there are simple
%facilities related to page numbers, headings, and footings:
\subsection 用于输出的宏. The |plain.tex| file also contains the output
routine described in Chapters 15 and~23. First there are simple
facilities related to page numbers, headings, and footings:
\beginlines
|\countdef|^|\pageno||=0 \pageno=1 % first page is number 1|
|\newtoks|^|\headline|| \headline={\hfil} % headline is normally blank|
|\newtoks|^|\footline|| \footline={\hss\tenrm\folio\hss}|\parbreak%
|  % footline is normally a centered page number in font \tenrm|
|\def|^|\folio||{\ifnum\pageno<0 \romannumeral-\pageno \else\number\pageno \fi}|
|\def|^|\nopagenumbers||{\footline={\hfil}} % blank out the footline|
|\def|^|\advancepageno||{\ifnum\pageno<0 \global\advance\pageno by -1|\parbreak%
|  \else\global\advance\pageno by 1 \fi} % increase |\||pageno|\|
\smallbreak
|\newif\ifr@ggedbottom|
|\def|^|\raggedbottom||{\topskip10pt plus60pt \r@ggedbottomtrue}|
|\def|^|\normalbottom||{\topskip10pt \r@ggedbottomfalse} % undoes \raggedbottom|
\endlines

The ^|\footnote| macro has a few subtle features that can best be appreciated
by someone who reads Chapter~15 very carefully. It also uses some
^|\bgroup| and ^|\futurelet| and ^|\aftergroup| trickery, so that the footnote
text does not need to be a ^{parameter} to |\vfootnote|:
\beginlines
^|\newinsert||\footins|
|\def\footnote#1{\let\@sf=\empty|%
 | % parameter #2 (the text) is read later|\parbreak%
|  \ifhmode\edef\@sf{\spacefactor=\the\spacefactor}\/\fi|\parbreak%
|  #1\@sf\vfootnote{#1}}|
|\def|^|\vfootnote||#1{\insert\footins\bgroup|\parbreak%
|  \interlinepenalty=\interfootnotelinepenalty|\parbreak%
|  \splittopskip=\ht\strutbox % top baseline for broken footnotes|\parbreak%
|  \splitmaxdepth=\dp\strutbox \floatingpenalty=20000|\parbreak%
|  \leftskip=0pt \rightskip=0pt \spaceskip=0pt \xspaceskip=0pt|\parbreak%
|  \textindent{#1}\footstrut\futurelet\next\fo@t}|
|\def\fo@t{\ifcat\bgroup\noexpand\next \let\next\f@@t|\parbreak%
|  \else\let\next\f@t\fi \next}|
|\def\f@@t{\bgroup\aftergroup\@foot\let\next}|
|\def\f@t#1{#1\@foot}|
|\def\@foot{\strut\egroup}|
|\def\footstrut{\vbox to\splittopskip{}}|
|\skip\footins=\bigskipamount % space added when footnote is present|
|\count\footins=1000 % footnote magnification factor (1 to 1)|
|\dimen\footins=8in % maximum footnotes per page|
\endlines

^{Floating insertions} are handled by doing an ^|\insert| whose
vertical list consists of a penalty item followed by a single box:
\beginlines
|\newinsert\topins \newif\ifp@ge \newif\if@mid|
|\def|^|\topinsert||{\@midfalse\p@gefalse\@ins}|
|\def|^|\midinsert||{\@midtrue\@ins}|
|\def|^|\pageinsert||{\@midfalse\p@getrue\@ins}|
|\skip\topins=0pt % no space added when a topinsert is present|
|\count\topins=1000 % magnification factor (1 to 1)|
|\dimen\topins=\maxdimen % no limit per page|
|\def\@ins{\par\begingroup\setbox0=\vbox\bgroup} % start a \vbox|
|\def\endinsert{\egroup % finish the \vbox|\parbreak%
|  \if@mid \dimen@=\ht0 \advance\dimen@ by\dp\z@ |%
 |\advance\dimen@ by12\p@|\parbreak%
|    \advance\dimen@ by\pagetotal \advance\dimen@ by-\pageshrink|\parbreak%
|    \ifdim\dimen@>\pagegoal \@midfalse\p@gefalse\fi\fi|\parbreak%
|  \if@mid \bigskip \box0 \bigbreak|\parbreak%
|  \else\insert\topins{\penalty100 % floating insertion|\parbreak%
|    |^|\splittopskip||=0pt |^|\splitmaxdepth||=\maxdimen |%
  ^|\floatingpenalty||=0|\parbreak%
|    \ifp@ge \dimen@=\dp0|\parbreak%
|     \vbox to\vsize{\unvbox0 \kern-\dimen@} % depth is zero|\parbreak%
|    \else \box0 \nobreak\bigskip\fi}\fi\endgroup}|
\endlines

Most of the ^|\output| routine appears in Chapter 23; it is given here
in full:
\beginlines
|\output={\plainoutput}|
|\def|^|\plainoutput|%
  |{\shipout\vbox{\makeheadline\pagebody\makefootline}%|\parbreak%
|  \advancepageno|\parbreak%
|  \ifnum\outputpenalty>-20000 \else\dosupereject\fi}|
|\def|^|\pagebody||{\vbox to\vsize{\boxmaxdepth=\maxdepth \pagecontents}}|
|\def|^|\makeheadline||{\vbox to 0pt{\vskip-22.5pt|\parbreak%
|    \line{\vbox to8.5pt{}\the\headline}\vss}\nointerlineskip}|
|\def|^|\makefootline||{\baselineskip=24pt \line{\the\footline}}|
|\def|^|\dosupereject|%
  |{\ifnum\insertpenalties>0 % something is being held over|\parbreak%
|  \line{}\kern-\topskip\nobreak\vfill\supereject\fi}|
\smallbreak
|\def|^|\pagecontents||{\ifvoid\topins\else\unvbox\topins\fi|\parbreak%
|  \dimen@=\dp255 \unvbox255|\parbreak%
|  \ifvoid\footins\else % footnote info is present|\parbreak%
|    \vskip\skip\footins \footnoterule \unvbox\footins\fi|\parbreak%
|  \ifr@ggedbottom \kern-\dimen@ \vfil \fi}|
|\def|^|\footnoterule||{\kern-3pt|\parbreak%
|  \hrule width 2truein \kern 2.6pt} % the \hrule is .4pt high|
\endlines

%\subsection Hyphenation and everything else.  The last part of |plain.tex|
%reads the hyphenation patterns and exceptions found on file ^|hyphen.tex|
%(see Appendix~H\null); then it defines a few miscellaneous macros, sets
%up ^|\rm| type, and that's all!
\subsection 连字及其他.  The last part of |plain.tex|
reads the hyphenation patterns and exceptions found on file ^|hyphen.tex|
(see Appendix~H\null); then it defines a few miscellaneous macros, sets
up ^|\rm| type, and that's all!
\beginlines
^|\lefthyphenmin||=2 |^|\righthyphenmin||=3 % disallow x- or -xx breaks|
|\input hyphen % the hyphenation patterns and exceptions|
\nobreak\smallskip
|\def|^|\magnification||{|^|\afterassignment||\m@g\count@}|
|\def\m@g{\mag=\count@|\parbreak%
|  \hsize6.5truein\vsize8.9truein\dimen\footins8truein}|
\smallbreak
|\def|^|\tracingall|%
  |{\tracingonline=1 \tracingcommands=2 \tracingstats=2|\parbreak%
|  \tracingpages=1 \tracingoutput=1 \tracinglostchars=1 |\parbreak%
|  \tracingmacros=2 \tracingparagraphs=1 \tracingrestores=1 |\parbreak%
|  \showboxbreadth=\maxdimen \showboxdepth=\maxdimen \errorstopmode}|
\smallbreak
|\def|^|\showhyphens|%
  |#1{\setbox0=\vbox{\parfillskip0pt \hsize=\maxdimen \tenrm|\parbreak%
|  \pretolerance=-1 \tolerance=-1 \hbadness=0 \showboxdepth=0 \ #1}}|
\smallbreak
|\normalbaselines\rm % select roman font|
|\nonfrenchspacing % punctuation affects the spacing|
|\catcode`@=12 % at signs are no longer letters|
\nobreak\smallskip
|\def|^|\fmtname||{plain}\def\fmtversion{3.0} % identifies the current format|
\endlines
The format name and version number are recorded in control sequences,
in order to help the people who might have to explain why something
doesn't work. Macro files like |plain.tex| should not be changed in any way,
except with respect to preloaded fonts, unless the changes are authorized
by the author of the macros.

\endchapter

The purpose of a programming system is to make a computer easy to use.
To do this, it furnishes languages and various facilities
that are in fact programs invoked and controlled by language features.
But these facilities are bought at a price:
the external description of a programming system is ten to twenty times
as large as the external description of the computer system itself.
The user finds it far easier to specify any particular function,
but there are far more to choose from,
and far more options and formats to remember.
\author FREDERICK P. ^{BROOKS}, JR., {\sl The Mythical Man Month\/} (1975)
  % p 43

\bigskip

When someone says, ``I want a programming language
in which I need only say what I wish done,''
give him a lollipop.
\author ALAN ^{PERLIS}, {\sl Epigrams on Programming\/} (1982)
  % SIGPLAN Notices 17,9 (September 82), 7--13.
  % There are many more, like "Editing is a rewording activity."

%\eject
\eject\byebye
