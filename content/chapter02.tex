% -*- coding: utf-8 -*-

\input macros

%\beginchapter Chapter 2. Book Printing\\versus\\Ordinary Typing
\beginchapter Chapter 2. 书籍排版与普通排版

\origpageno=3

%When you first started using a computer terminal, you probably had to adjust
%to the difference between the digit `1' and the lowercase letter `l'.
%When you take the next step to the level of typography that is common in
%book publishing, a few more adjustments of the same kind need to be made;
%your eyes and your fingers need to learn to make a few more distinctions.
\1当你第一次使用计算机终端时,可能不得不校准数字``1''和小写``l''的差别。%
当你进行到一般书籍出版的排印步骤时,还要做出几个同样的校准;%
你的眼睛和手指需要知道这几个差别。

%In the first place, there are two kinds of ^{quotation marks} in books, but
%only one kind on the typewriter. Even your computer terminal, which has
%more characters than an ordinary typewriter, probably has only a
%non-oriented double-quote mark (|"|), because the standard ^{ASCII} code
%for computers was not invented with book publishing in mind. However, your
%terminal probably does have two flavors of single-quote marks, namely |`|
%and |'|; the second of these is useful also as an ^{apostrophe}.
%American keyboards usually contain a left-quote character that shows up
%as something like {\tt\char'22}, and an apostrophe or right-quote that
%looks like {\tt\char'15} or {\tt\char'23}.
首先,在书中有两种^{引号},但是在打字机上只有一种。
即使在比普通打字机的字符更多的计算机键盘上,也可能只有一种无方向的双引号(|"|),
因为计算机的标准 ^{ASCII} 码不是专门为书籍出版而发明的。
但是,你的计算机可能确实有两种不同的单引号,即 |`| 和 |'| ;
第二个也用作缩写号【译注:比如 |don't| 】。
美国键盘包含一个看起来象 {\tt\char'22} 的左引号,
和一个看起来象 {\tt\char'15} 或 {\tt\char'23} 的缩写号或右引号。

%To produce double-quote marks with \TeX, you simply type two single-quote marks
%of the appropriate kind. For example, to get the phrase
%\begindisplay
%``I understand.''
%\enddisplay
%(including the quotation marks) you should type
%\begintt
%``I understand.''
%\endtt
%to your computer.
为了用 \TeX\ 得到双引号,可直接键入两个相应类型的单引号。%
比如,为了得到习惯用语
\begindisplay
``I understand.''
\enddisplay
(包括双引号), 应该在计算机中键入
\begintt
``I understand.''
\endtt

%A typewriter-like style of type will be used throughout this manual to indicate
%\TeX\ constructions that you might type on your terminal, so that the
%symbols actually typed are readily distinguishable from the output \TeX\ would
%produce and from the comments in the manual itself. Here are the symbols to be
%used in the examples:
%\begintt
%ABCDEFGHIJKLMNOPQRSTUVWXYZ
%abcdefghijklmnopqrstuvwxyz
%0123456789"#$%&@*+-=,.:;?!
%()<>[]{}`'\||/_^~
%\endtt
%If your computer terminal doesn't happen to have all of these, don't
%despair; \TeX\ can make~do with the ones you have. An additional symbol
%\begindisplay
%\]
%\enddisplay
%is used to stand for a {\sl ^{blank space}}, in case it is important
%to emphasize that a blank space is being typed; thus, what you {\sl
%really\/} type in the example above is
%\begintt
%``I|]understand.''
%\endtt
%Without such a symbol you would have
%difficulty seeing the invisible parts of certain constructions. But we
%won't be using `\]' very often, because spaces are usually visible enough.
类似打字机风格将在本手册通篇使用,以显示可在你的终端上键入的 \TeX\ 的用词,%
使得实际键入的符号容易从 \TeX\ 产生的输出和手册自身的注解区分开。%
下面是在例子中要用到的符号:
\begintt
ABCDEFGHIJKLMNOPQRSTUVWXYZ
abcdefghijklmnopqrstuvwxyz
0123456789"#$%&@*+-=,.:;?!
()<>[]{}`'\||/_^~
\endtt
如果你的计算机终端凑巧不能包含所有这些字符,别失望;%
\TeX\ 可设法应付它。%
另外一个符号
\begindisplay
\]
\enddisplay
用来表示一个 {\sl {blank space}}~(空格)——如果要重点强调所键入的空格;%
因此,你{\KT{10}真正}要在上面的例子中键入的是
\begintt
``I|]understand.''
\endtt
如果没有这样的符号,就很难知道某些用词的看不见的部分。%
但是我们不会太频繁使用``\]'', 因为空格一般足以被看见。

%Book printing differs significantly from ordinary typing with respect to
%^{dashes}, ^{hyphens}, and ^{minus signs}. In good math books,
%these symbols are all different; in fact there usually are at least four
%different symbols:
%\begindisplay
%a hyphen (-);\cr
%an en-dash (--);\cr
%an em-dash (---);\cr
%a minus sign ($-$).\cr
%\enddisplay
%Hyphens are used for compound words like `daughter-in-law' and `X-rated'.
%^{En-dash}es are used for number ranges like `pages 13--34', and also in
%contexts like `exercise 1.2.6--52'. ^{Em-dash}es are used for punctuation in
%sentences---they are what we often call simply dashes. And minus signs are
%used in formulas. A conscientious user of \TeX\ will be careful to distinguish
%these four usages, and here is how to do it:
%\begindisplay
%for a hyphen, type a hyphen (|-|);\cr
%for an en-dash, type two hyphens (|--|);\cr
%for an em-dash, type three hyphens (|---|);\cr
%for a minus sign, type a hyphen in mathematics mode (|$-$|).\cr
%\enddisplay
%(Mathematics mode occurs between dollar signs; it is discussed later, so you
%needn't worry about it now.)
\1书籍排版和普通排版在横线号、连字符和减号方面明显不同。
在优秀的数学书籍中,这些符号都是不同的;实际上,一般至少有四种不同的符号:
\begindisplay
连字符 (-);\cr
连接号 (--);\cr
破折号 (---);\cr
减号 ($-$).\cr
\enddisplay
连字符用在像 ``daughter-in-law'' 和 ``X-rated'' 这样的复合单词中。
连接号用在像 ``pages 13--34'' 以及像本书中 ``exercise 1.2.6--52''
这样的数字区间方面。
破折号用作句子中的标点——就是我们通常所称的简单横线号。
而减号用在公式中。
原则性强的 \TeX\ 用户要仔细区分这四种用法,说明如下:
\begindisplay
对连字符,键入一个连字符 (|-|);\cr
对连接号,键入两个连字符 (|--|);\cr
对破折号,键入三个连字符 (|---|);\cr
对减号,把连字符放在数学模式中 (|$-$|)。\cr
\enddisplay
(数学模式出现在美元符号之间;这将在后面讨论,因此你现在不必考虑它。)

%\exercise Explain how to type the following sentence to \TeX: Alice said,
%``I always use an en-dash instead of a hyphen when specifying page numbers
%like `480--491' in a ^{bibliography}.''
%\answer |Alice said, ``I always use an en-dash instead of a hyphen when|\break
%|specifying page numbers like `480--491' in a bibliography.''| \
%(The wrong answer to this question ends with |'480-49l' in a bibliography."|)
\exercise 看看怎样在 \TeX\ 中键入下列句子:
Alice said,
``I always use an en-dash instead of a hyphen when specifying page numbers
like `480--491' in a bibliography.''
\answer |Alice said, ``I always use an en-dash instead of a hyphen when|\break
|specifying page numbers like `480--491' in a bibliography.''|%
(这个问题的容易犯错误的是在结尾处键入了 |'480-49l' in a bibliography."|)

%\exercise What do you think happens when you type four hyphens in a row?
%\answer You get em-dash and hyphen (----), which looks awful.
\exercise 当你在一行中连续键入四个连字符时,会得到什么结果?
\answer 你会得到一个破折号和一个连字符(----),看起来很丑吧。
\smallskip

%If you look closely at most well-printed books, you will find that certain
%combinations of letters are treated as a unit. For example, this is true
%of the `f' and the `i' of `find'. Such combinations are called {\sl
%^{ligatures}}, and professional typesetters have traditionally been
%trained to watch for letter combinations such as |ff|, |fi|, |fl|, |ffi|, and
%|ffl|. \ (The reason is that words like `f{}ind' don't look very good in
%most styles of type unless a ligature is substituted for the letters that
%clash. It's somewhat surprising how often the traditional ligatures appear
%in English; other combinations are important in other languages.)
如果你仔细查看大多数排版精良的书籍,就会发现某些字母组合会当成一个字母来处理。
例如,``find'' 中的 ``f' 和 ``i'' 就是这样。
这样的组合称为{\sl 连写}(ligature),而专业排版工通常会经过培训以留意诸如
|ff|、|fi|、|fl|、|ffi| 和 |ffl| 的字母组合。(原因是像 ``f{}ind''
这样的单词在大多数排版风格上不太好看,除非用连写代替这些冲突的字母。
有点令人惊讶的是,传统的连写在英语中经常出现;其他组合在其他语言中很重要。

%\exercise Think of an English word that contains two ligatures.
%\answer fluffier firefly fisticuffs, flagstaff fireproofing,
%chiffchaff and riffraff.
\exercise 给出一个包含两个连写的英语单词。
\answer fluffier firefly fisticuffs,flagstaff fireproofing,
chiffchaff and riffraff。
\smallskip

%The good news is that you do
%{\sl not\/} have to concern yourself with ligatures: \TeX\ is perfectly
%capable of handling such things by itself, using the
%same mechanism that converts `|--|' into `--'. In fact, \TeX\ will also look
%for combinations of adjacent letters (like `|A|' next to `|V|'\thinspace)
%that ought to be moved closer together for better appearance; this is
%called {\sl ^{kerning}}.
还好,你{\KT{10}不必}管这些连写:\TeX\ 完全有能力自己解决这些问题,
方法同把 ``|--|'' 转换成 ``--'' 一样。实际上,为了得到更好的观感,
\TeX\ 还把应当更接近的相邻字母(象紧跟 ``|V|'' 的 ``|A|'')看作组合;
这称为{\KT{10}字距调整}。

%\medbreak
%To summarize this chapter: When using \TeX\ for straight copy, you type
%the copy as on an ordinary typewriter, except that you need to be careful
%about quotation marks, the number 1, and various kinds of hyphens/dashes.
%\TeX\ will automatically take care of other niceties like ligatures and
%kerning.
\medbreak
\1总结:当用 \TeX\ 直接排版时,就像在普通打字机上键入书稿一样,
但是需要注意引号,数字 1 和各种连字符/破折号。
其它诸如连写和字距调整这样的精细调整由 \TeX\ 自动完成。

%\danger (Are you sure you should be reading this paragraph? The
%``^{dangerous bend}'' sign here is meant to warn you about material that
%ought to be skipped on first reading. And maybe also on second reading.
%The reader-beware paragraphs sometimes refer to concepts that aren't
%explained until later chapters.)
\danger (你准备好看本段了吗?%
这里的``危险''标志就是提醒你,这些材料应当在首次阅读时跳过\hbox{去。}%
可能已经是第二遍了吧。%
提醒读者的这些段落有时用到了后面章节中的概念。)

%\danger If your keyboard does not contain a left-quote symbol, you can
%type ^|\lq|, followed by a space if the next character is a letter, or
%followed by a |\| if the next character is a space. Similarly, ^|\rq|
%yields a right-quote character. Is that clear?
%\begintt
%\lq\lq|]I|]understand.\rq\rq\|]
%\endtt
\danger 如果你的键盘没有左引号符号,那么你可以键入 |\lq|, 如果下一个字符是字母,%
就应该紧接一个空格,如果下一个字符是空格,就应该紧接一个 |\|。%
类似地,|\rq| 得到一个右引号字符。清楚了吗?
\begintt
\lq\lq|]I|]understand.\rq\rq\|]
\endtt

%\danger In case you need to type ^{quotes within quotes}, for example a
%single quote followed by a double quote, you can't simply type
%\thinspace|'''|\thinspace\ because \TeX\ will interpret this as '''
%(namely, double quote followed by single quote).  If you have already read
%Chapter~5, you might expect that the solution will be to use
%grouping---namely, to type something like \thinspace|{'}''|. But it turns
%out that this doesn't produce the desired result, because there is usually
%less space following a single right quote than there is following a double
%right quote: What you get is {'}'', which is indeed a single quote
%followed by a double quote (if you look at it closely enough), but it
%looks almost like three equally spaced single quotes.  On the other hand,
%you certainly won't want to type \thinspace|'|\]|''|, because that space
%is much too large---it's just as large as the space between words---and
%\TeX\ might even start a new line at such a space when making up a
%paragraph! The solution is to type \thinspace|'\thinspace''|, which
%produces '\thinspace'' as desired.^^|\thinspace|
\danger 如果你想键入引号中的引号,比如右单引号后跟一个右双引号,
那么你不能直接键入 |'''|, 因为 \TeX\ 将把它看成 '''(即右双引号后面跟右单引号)。%
如果你已经看过第5章,可能认为解决之道是采用编组——即键入所谓的 |{'}''|。%
但是,却得不到想看到的结果,因为单右引号后面的间距比双右引号后面的要小:
得到的是 {'}'', 这的确是一个单引号后面跟一个双引号(如果你仔细观察的话),
但是看起来却几乎是三个一样的单引号。%
另一方面,当然也不要键入 |'|\]|''|, 否则间距太大——就象单词之间的空格那么大,
并且当 \TeX\ 在生成段落时会在这样的空格处断行!
解决的方法是键入 |'\thinspace''|, 就得到所要的 '\thinspace'' 了。

%\dangerexercise OK, now you know how to produce ''' and '\thinspace'';
%how do you get ``\thinspace` and `{}``\thinspace?
%\answer |``\thinspace`|; and either |`{}``| or |{`}``| or something similar.
%Reason: There's usually less space {\sl preceding\/} a single left quote than
%there is preceding a double left quote. \ (Left and right are opposites.)
\dangerexercise 好了,现在你知道怎样得到 ''' 和 '\thinspace'' 了;
那么你怎样得到 ``\thinspace` 和 `{}``\thinspace 呢?
\answer |``\thinspace`|,以及 |`{}``| 或 |{`}``| 或其他类似写法。
原因是:通常在左单引号{\sl 之前}的空白比在左双引号之前的要小。%
(左引号和和右引号的情形相反。)

%\dangerexercise Why do you think the author introduced the control
%sequence |\thinspace| to solve the adjacent-quotes problem, instead of
%recommending the trickier construction |'$|^|\,||$''| (which also works)?
%\answer Eliminating ^|\thinspace| would mean that a user need not learn
%the term; but it is not advisable to minimize terminology by ``overloading''
%math mode with tricky constructions. For example, a user who wishes to
%take advantage of \TeX's ^|\mathsurround| feature would be thwarted by
%non-mathematical uses of dollar signs. \ (Incidentally, neither |\thinspace|
%nor ^|\,| are built into \TeX; both are defined in terms of more
%primitive features, in Appendix~B.)
\dangerexercise 想一下作者为什么用控制系列|\thinspace|来解决相邻引号问题而%
不是用更巧妙的 |'$|^|\,||$''|~(它也可以胜任)?
\answer 剔除 ^|\thinspace| 意味着用户无需学习该术语;
但是为让术语最少而过多使用数学模式的巧妙构造是不可取的。
举个例子,当用户希望使用 \TeX\ 的 ^|\mathsurround| 特性时,
在非数学情形中使用的 |$| 将给他造成障碍。%
(顺带说一下,|\thinspace| 和 ^|\,| 都不是 \TeX\ 内置的,
这两者都是用更原始的特性定义的,见附录 B 。)

\endchapter

In modern Wit all printed Trash, is
Set off with num'rous\/ {\rm Breaks}\raise.5ex\vbox{\hrule width 2em}%
  and\/ {\rm Dashes}\raise.5ex\vbox{\hrule width 1em}
% no period after the em-dash: stet!
% Sir Walter Scott ruined this quote in his edition of Swift!
\author JONATHAN ^{SWIFT}, {\sl On Poetry: A Rapsody\/} (1733) % lines 93--94
% Rapsody: stet!

\bigskip

Some compositors still object to work
in offices where type-composing machines are introduced.
\author WILLIAM STANLEY ^{JEVONS}, {\sl Political Economy\/} (1878) % sec 55
% "They are all afraid that if the work is done too easily and rapidly,
% they will not be wanted to do it."
% Jevons goes on to say that justifying and page makeup can't be done
% profitably by machines, so the employees needn't fear losing their jobs.

\vfill\eject\byebye
