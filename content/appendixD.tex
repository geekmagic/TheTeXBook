% -*- coding: utf-8 -*-

\input macros

%\beginchapter Appendix D. Dirty Tricks
\beginchapter Appendix D. 诡计多端

\TeX\ was designed to do the ordinary tasks of typesetting: to make paragraphs
and pages. But the underlying mechanisms that facilitate ordinary
typesetting---e.g., boxes, glue, penalties, and macros---are extremely
versatile; hence people have discovered sneaky ways to coerce \TeX\ into
doing tricks quite different from what its author originally had in mind.
Such clever constructions are not generally regarded as examples of
``high \TeX''; but many of them have turned out to be useful and
instructive, worthy of being known (at least by a few wizards). The purpose
of this appendix is to introduce crafty and/or courageous readers
to the nether world of \TeX arcana.

\ninepoint\medskip
\setbox0=\hbox spread-6\fontdimen4\font % that removes all the shrinkability
 {\strut Please don't read this material until you've}
\setbox1=\hbox to\wd0{\strut had plenty of experience with plain \TeX.}
\line{\leaders\hbox{\dbend\kern2pt}\hfil\vtop{\box0\box1}}
\nointerlineskip
\noindent\strut After you have read and understood the secrets below, you'll
know all sorts of devious combinations of \TeX\ commands,
and you will often be tempted to write inscrutable macros. Always remember,
however, that there's usually a simpler and better way to do something
than the first way that pops into your head. You may not have to
resort to any subterfuge at all, since \TeX\ is able to do lots of things
in a straightforward way. Try for simple solutions first.

%\subsection Macro madness. If you need to write complicated ^{macros}, you'll
%need to be familiar with the fine points in Chapter~20. \TeX's control
%sequences are divided into two main categories, ``expandable'' and
%``unexpandable''; the former category includes all macros and |\if...\fi|
%tests, as well as special operations like |\the| and |\input|, while the
%latter category includes the primitive commands listed in Chapter~24.
%The expansion of expandable tokens takes place in \TeX's ``^{mouth},''
%but primitive commands (including assignments) are done in \TeX's
%``^{stomach}.'' One important consequence of this structure is that
%it is impossible to redefine a control sequence or to advance a register
%while \TeX\ is expanding the token list of, say, a |\message| or |\write|
%command; assignment operations are done only when \TeX\ is building a
%vertical or horizontal or math list.
\subsection 疯狂的宏. If you need to write complicated ^{macros}, you'll
need to be familiar with the fine points in Chapter~20. \TeX's control
sequences are divided into two main categories, ``expandable'' and
``unexpandable''; the former category includes all macros and |\if...\fi|
tests, as well as special operations like |\the| and |\input|, while the
latter category includes the primitive commands listed in Chapter~24.
The expansion of expandable tokens takes place in \TeX's ``^{mouth},''
but primitive commands (including assignments) are done in \TeX's
``^{stomach}.'' One important consequence of this structure is that
it is impossible to redefine a control sequence or to advance a register
while \TeX\ is expanding the token list of, say, a |\message| or |\write|
command; assignment operations are done only when \TeX\ is building a
vertical or horizontal or math list.

For example, it's possible to put |\n| asterisks into a paragraph,
by saying simply `|{|^|\loop||\ifnum\n>0 *\advance\n-1 \repeat}|'.
But it's much more difficult to define a
control sequence |\asts| to consist of exactly |\n| consecutive asterisks.
If\/ |\n| were known to be at most~5, say, it would be possible to write
\begintt
\edef\asts{\ifcase\n\or*\or**\or***\or****\or*****\else\bad\fi}
\endtt
since \TeX\ handles |\ifcase| in its mouth. But for general |\n| it would
be impossible to use a construction like
`|\edef\asts{\loop\ifnum\n>0 *\advance\n-1 \repeat}|', since |\n| doesn't change
during an ^|\edef|. A more elaborate program is needed; e.g.,
\begintt
{\xdef\asts{}
  \loop\ifnum\n>0 \xdef\asts{\asts*}\advance\n-1 \repeat}
\endtt
And here's another solution (which is faster, because ^{token list registers}
can be expanded more quickly than macros, using ^|\the|):
\begintt
\newcount\m \newtoks\t \m=\n \t={}
\loop \ifnum\m>0 \t=\expandafter{\the\t *} \advance\m-1 \repeat
\edef\asts{\the\t}
\endtt

However, both of these solutions have a running time proportional to the
square of\/~|\n|. There's a much quicker way to do the job:
\begintt
\begingroup\aftergroup\edef\aftergroup\asts\aftergroup{
\loop \ifnum\n>0 \aftergroup*\advance\n-1 \repeat
\aftergroup}\endgroup
\endtt
Get it? The ^|\aftergroup| commands cause a whole list of other tokens to be
saved up for after the group! This method has only one flaw, namely
that it takes up |\n| cells of space on \TeX's ^{input stack} and |\n|
more on \TeX's ^{save stack}; hence a special version of \TeX\ may be
required when |\n| is larger than 150 or~so.

(Incidentally, there's a completely different way to put |\n| asterisks
into a paragraph, namely to say `|\setbox0=\hbox{*}|%
^|\cleaders|^|\copy||0\hskip\n\wd0|'. This may seem to be the fastest
solution of all; but actually it is not so fast, when all things are
considered, since it generates four bytes of ^|dvi| output per asterisk,
compared to only one byte per asterisk in the other methods. Input/output
time takes longer than computation time, both in \TeX\ itself and in
the later stages of the printing process.)

The problem just solved may seem like a rather special application; after
all, who needs a control sequence that contains a variable number of
asterisks? But the same principles apply in other similar cases,
e.g., when you want to construct a variable-length ^|\parshape|
specification. Similarly, many of the ``toy problems'' solved below are
meant to illustrate paradigms that can be used in real-life situations.

The precise rules for expansion are explained in Chapter~20; and the
best way to get familiar with \TeX's expansion mechanism is to watch it
in action, looking at the log file when |\tracingmacros=2| and
|\tracingcommands=2|. One of the important ways to change the normal
order of expansion is to use ^|\expandafter|; the construction
\begintt
\expandafter\a\b
\endtt
causes |\b| to be expanded first, then~|\a|.
And since |\expandafter| is itself expandable, the construction
\begintt
\expandafter\expandafter\expandafter\a\expandafter\b\c
\endtt
causes |\c| to be expanded first, then |\b|, then |\a|. \ (The next step,
\begintt
\expandafter\expandafter\expandafter\expandafter
 \expandafter\expandafter\expandafter\a
 \expandafter\expandafter\expandafter\b\expandafter\c\d
\endtt
is probably too lengthy to be of any use.)

It's possible to make good use of\/ |\expandafter\a\b| even when |\a|
isn't expandable. For example, the token list assignment
`|\t=\expandafter{\the\t|~|*}|' in the example on the previous page
was able to invade territory where expansion is normally suppressed,
by expanding after a left brace. Similarly,
\begintt
\t=\expandafter{\expandafter*\the\t}
\endtt
would have worked; and ^^|\uppercase|
\begintt
\uppercase\expandafter{\romannumeral\n}
\endtt
yields the value of register |\n| in ^{uppercase roman numerals}.
^^{roman numerals, uppercase}

Here's a more interesting example: Recall that ^|\fontdimen||1| is the
amount of ``^{slant} per point'' of a font; hence, for example,
`^|\the||\fontdimen1\tenit|' expands to `{\tt\the\fontdimen1\tenit}',
where the characters `|pt|' are of category~12. After the macro definitions
\begintt
{\catcode`p=12 \catcode`t=12 \gdef\\#1pt{#1}}
\let\getfactor=\\
\def\kslant#1{\kern\expandafter\getfactor\the\fontdimen1#1\ht0}
\endtt
one can write, e.g., `|\kslant\tenit|' and this will expand to
`|\kern0.25\ht0|'. If the boundary of\/ |\box0| is considered to be
slanted by 0.25 horizontal units per vertical unit, this kern measures
the horizontal distance by which the top edge of the box is skewed with
respect to an edge at the baseline. All of the computation of\/
|\kslant| is done in \TeX's mouth; thus, the mouth can do some rather
complicated things even though it cannot assign new values. \ (Incidentally,
an indirect method was used here to define the control sequence ^|\getfactor|
when the character~|t| had category~12, since control words normally consist
only of letters. The alternative construction
\begintt
{\catcode`p=12 \catcode`t=12
  \csname expandafter\endcsname\gdef
  \csname getfactor\endcsname#1pt{#1}}
\endtt
would also have worked, since ^|\csname| and ^|\endcsname| don't
contain `|p|' or `|t|'!)

The mechanism by which \TeX\ determines the ^{arguments} of a macro
can be applied in unexpected ways. Suppose, for example, that |\t|~is
a token list register that contains some text; we wish to determine if
at least one asterisk (|*|$_{12}$) appears in that text. Here's
one way to do it:
\begintt
\newif\ifresult % for the result of a computed test
\def\atest#1{\expandafter\a\the#1*\atest\a}
\long\def\a#1*#2#3\a{\ifx\atest#2\resultfalse\else\resulttrue\fi}
\endtt
^^|\long|^^|\newif|
Now after `|\atest\t|', the control sequence |\ifresult|
will be |\iftrue| or |\iffalse|, depending on whether or not |\t|~contains
an asterisk. \ (Do you see why?) \ And here's a slightly more elegant way
to do the same thing, using ^|\futurelet| to look ahead:
\begintt
\def\btest#1{\expandafter\b\the#1*\bb}
\long\def\b#1*{\futurelet\next\bb}
\long\def\bb#1\bb{\ifx\bb\next\resultfalse\else\resulttrue\fi}
\endtt
In both cases the solution works if\/ |\t| contains control sequence tokens
as well as character tokens, provided that the special control sequences
|\atest|, |\a|, and |\bb| don't appear. Notice, however, that an asterisk
is ``hidden'' if it appears within a ^{group}~|{...}|; the test is limited
to asterisks at nesting level zero. A token list register is always balanced
with respect to grouping, so there is no danger of the test leading
to error messages concerning missing braces or extra braces.

We can apply the ideas in the preceding paragraph to solve a problem
related to generalized math formatting: ^^{displays, non-centered}
The goal is to set \TeX\ up so that the respective constructions
`\thinspace|$$|$\,\alpha\,$|$$|\thinspace', ^^{dollar dollar}
`\thinspace|$$|$\,\alpha\,$^|\eqno|$\,\beta\,$|$$|\thinspace', and
`\thinspace|$$|$\,\alpha\,$^|\leqno|$\,\beta\,$|$$|\thinspace' will cause
a macro |$$\generaldisplay$$| to be invoked, with |\eq| defined to be $\alpha$;
^^{communication between macros}
furthermore, the test |\ifeqno| should be true when an equation
number~$\beta$ is present, and |\ifleqno| should be true in the case
of\/ |\leqno|. When $\beta$ is present, it should be stored in~|\eqn|.
Here $\alpha$~and~$\beta$ are arbitrary balanced token lists that don't
contain either |\eqno| or |\leqno| at nesting level zero. The
following macros do the required maneuvers:
\begintt
\newif\ifeqno \newif\ifleqno \everydisplay{\displaysetup}
\def\displaysetup#1$${\displaytest#1\eqno\eqno\displaytest}
\def\displaytest#1\eqno#2\eqno#3\displaytest{%
  \if!#3!\ldisplaytest#1\leqno\leqno\ldisplaytest
  \else\eqnotrue\leqnofalse\def\eqn{#2}\def\eq{#1}\fi
  \generaldisplay$$}
\def\ldisplaytest#1\leqno#2\leqno#3\ldisplaytest{\def\eq{#1}%
  \if!#3!\eqnofalse\else\eqnotrue\leqnotrue\def\eqn{#2}\fi}
\endtt
An examination of the three cases
|$$|$\,\alpha\,$|$$|,
|$$|$\,\alpha$|\eqno|$\,\beta\,$|$$|,
|$$|$\,\alpha$|\leqno|$\,\beta\,$|$$| shows that the correct actions will
ensue. Parameter |#3| in the tests `|\if!#3!|' will be either empty or
|\eqno| or |\leqno|; thus, the condition will be false (and the second
`|!|' will be skipped) unless |#3| is empty.

Returning to the problem of |*|'s in |\t|, suppose that it's
necessary to consider |*|'s at all levels of nesting. Then a slower
routine must be used:
\begintt
\def\ctest#1{\resultfalse\expandafter\c\the#1\ctest}
\def\c{\afterassignment\cc\let\next= }
\def\cc{\ifx\next\ctest \let\next\relax
  \else\ifx\next*\resulttrue\fi\let\next\c\fi \next}
\endtt
Here ^|\afterassignment| has been used to retain control after a
non-future ^|\let|; the `|= |' ensures that exactly one token is ^^{equals}
swallowed per use of\/~|\c|. This routine could be modified in an obvious
way to count the total number of |*|'s and/or tokens in~|\t|. Notice the
`|\let\next|' instructions in~|\cc|; it should be clear why the alternative
\begintt
\def\cc{\ifx\next\ctest\else\ifx\next*\resulttrue\fi\c\fi}
\endtt
would not work. \ (The latter |\c| would always swallow a `|\fi|'.)

^{Space tokens} are sometimes anomalous, so they deserve special care. The
following macro ^|\futurenonspacelet| behaves essentially like
|\futurelet| except that it discards any implicit or explicit space tokens
that intervene before a nonspace is scanned:
\begintt
\def\futurenonspacelet#1{\def\cs{#1}%
  \afterassignment\stepone\let\nexttoken= }
\def\\{\let\stoken= } \\ % now \stoken is a space token
\def\stepone{\expandafter\futurelet\cs\steptwo}
\def\steptwo{\expandafter\ifx\cs\stoken\let\next=\stepthree
  \else\let\next=\nexttoken\fi \next}
\def\stepthree{\afterassignment\stepone\let\next= }
\endtt
An operation like |\futurenonspacelet| is useful, for example,
when implementing macros that have a variable number of arguments.

Notice that `|\def\stepthree#1{\stepone}|' would not work here, because of
\TeX's rule that a \]$_{10}$ token is bypassed if it would otherwise be treated
as an ^{undelimited} ^{argument}. Because of this rule it is
difficult to distinguish explicit space tokens from implicit ones.
The situation is surprisingly complex, because it's possible to
use ^|\uppercase| to create ``^{funny space}'' tokens like |*|$_{10}$;
^^|\uccode| for example, the commands
\begintt
\uccode` =`* \uppercase{\uppercase{\def\fspace{ }\let\ftoken= } }
\endtt
make |\fspace| a macro that expands to a funny space, and they make
|\ftoken| an implicit funny space. \ (The tests ^|\if||\fspace*|,
|\if\ftoken*|, ^|\ifcat||\fspace\stoken|, and |\ifcat\ftoken\stoken|
will all be true, assuming that |*| has category~12; but if |*| has
category~10, |\if\fspace*| will be false, because \TeX\ normalizes
all newly created space tokens to \]$_{10}$, as explained in Chapter~8.) \
Since the various forms of space tokens are almost identical in
^^|\noexpand|^^|\escapechar|^^|\string|
behavior, there's no point in dwelling on the details.\footnote\dag{The
following little program is for \TeX\ exegetes who insist on
learning the whole story: Macro |\stest| decides whether or not
the first token of a given token list register
is a \<space token> as defined in Chapter~24. If so, the macro
decides whether or not the token is ``funny,'' i.e., whether or not
the character code is different from an ASCII \<space>; the macro
also decides whether a space token is explicit or implicit.
\begintt|global|displayindent=0pt|global|displaywidth=|hsize
\newif\ifspace \newif\iffunny \newif\ifexplicit
\def\stest#1{\expandafter\s\the#1! \stest}
\def\s{\funnyfalse \global\explicitfalse \futurelet\next\ss}
\def\ss{\ifcat\noexpand\next\stoken \spacetrue
   \ifx\next\stoken \let\next=\sss \else\let\next=\ssss \fi
  \else \let\next=\sssss \fi \next}
\long\def\sss#1 #2\stest{\def\next{#1}%
  \ifx\next\empty \global\explicittrue \fi}
\long\def\ssss#1#2\stest{\funnytrue {\uccode`#1=`~
  \uppercase{\ifcat\noexpand#1}\noexpand~% active funny space
  \else \escapechar=\if*#1`?\else`*\fi
    \if#1\string#1\global\explicittrue\fi \fi}}
\long\def\sssss#1\stest{\spacefalse}
\endtt}

\smallbreak
The argument to ^|\write| is expanded when a |\shipout| occurs, but
sometimes expansion isn't desired. Here's a macro (suggested by Todd
^{Allen}) that suppresses all expansion, by inserting ^|\noexpand| before
each control sequence or ^{active character}. The macro assumes that
|~|~is an active character, and that the tokens being written
do not include implicit spaces or braces. Funny spaces are changed
to ordinary ones.
\begintt
\long\def\unexpandedwrite#1#2{\def\finwrite{\write#1}%
  {\aftergroup\finwrite\aftergroup{\sanitize#2\endsanity}}}
\def\sanitize{\futurelet\next\sanswitch}
\def\sanswitch{\ifx\next\endsanity
  \else\ifcat\noexpand\next\stoken\aftergroup\space\let\next=\eat
   \else\ifcat\noexpand\next\bgroup\aftergroup{\let\next=\eat
    \else\ifcat\noexpand\next\egroup\aftergroup}\let\next=\eat
     \else\let\next=\copytoken\fi\fi\fi\fi \next}
\def\eat{\afterassignment\sanitize \let\next= }
\long\def\copytoken#1{\ifcat\noexpand#1\relax\aftergroup\noexpand
  \else\ifcat\noexpand#1\noexpand~\aftergroup\noexpand\fi\fi
  \aftergroup#1\sanitize}
\def\endsanity\endsanity{}
\endtt
As before, the heavy use of\/ ^|\aftergroup| in |\unexpandedwrite| means
that parameter |#2| should not include more than about 150~tokens.

%\subsection List macros. The next several macros we shall discuss can be
%used to maintain lists of information in the form ^^{backslash backslash}
%\begindisplay
%|\\{|\<item$_1$>|}\\{|\<item$_2$>|}| $\ldots$ |\\{|\<item$_n$>|}|
%\enddisplay
%where each \<item> is a balanced list of tokens. A parameterless control
%sequence whose replacement text has this form may be called a {\sl ^{list
%macro}}. The empty list macro has $n=0$ and it is called ^|\empty|.
\subsection 列表宏. The next several macros we shall discuss can be
used to maintain lists of information in the form ^^{backslash backslash}
\begindisplay
|\\{|\<item$_1$>|}\\{|\<item$_2$>|}| $\ldots$ |\\{|\<item$_n$>|}|
\enddisplay
where each \<item> is a balanced list of tokens. A parameterless control
sequence whose replacement text has this form may be called a {\sl ^{list
macro}}. The empty list macro has $n=0$ and it is called ^|\empty|.

It's easy to add new items at either end of a list macro, and to
concatenate list macros, for example as follows: ^^|\toksdef| ^^|\long|
\begintt
\toksdef\ta=0 \toksdef\tb=2 % token list registers for temp use
\long\def\leftappenditem#1\to#2{\ta={\\{#1}}\tb=\expandafter{#2}%
  \edef#2{\the\ta\the\tb}}
\long\def\rightappenditem#1\to#2{\ta={\\{#1}}\tb=\expandafter{#2}%
  \edef#2{\the\tb\the\ta}}
\def\concatenate#1=#2&#3{\ta=\expandafter{#2}\tb=\expandafter{#3}%
  \edef#1{\the\ta\the\tb}}
\endtt
Conversely, the left item of a list can be removed and placed in a
control sequence by the |\lop| macro defined in the following curious way:
\begintt
\def\lop#1\to#2{\expandafter\lopoff#1\lopoff#1#2}
\long\def\lopoff\\#1#2\lopoff#3#4{\def#4{#1}\def#3{#2}}
\endtt
For example, if\/ |\l| expands to the list `|\\{a\b}\\{c}\\{{d}}|', the
macro invocation |\lop\l\to\z| makes |\l| expand to `|\\{c}\\{{d}}|'
and |\z|~expand to `|a\b|'. The |\lop| operation should be used only when
|\l|~is nonempty, otherwise an error will occur; to test if\/ |\l|~is
empty, one simply says `|\ifx\l\empty|'.

The programming details of the |\lop| macro indicate why individual
items have been enclosed in |{...}| ^{groups}. A simpler kind of list,
in which grouping is omitted and an extra |\\| appears at the end,
suffices for many purposes; one could define, for instance,
\begintt
\long\def\lopoff\\#1\\#2\lopoff#3#4{\def#4{#1}\def#3{\\#2}}
\endtt
and the results would be almost the same as before. In this case an empty
list macro expands to `|\\|'. However, the new |\lop| resulting from this
new |\lopoff| macro also removes a pair
of braces, if the leftmost item happens to be a group; extra braces are
included in our general scheme to prevent such anomalies.

So far the examples we've considered haven't revealed why the |\\|'s
appear in the general scheme; it appears that grouping by itself
should be enough. But in fact, the |\\| separators are enormously
useful, because we can define |\\| to be any desired one-argument macro,
and then we can {\sl execute\/} the list! For example, here's a
way to count the number of items: ^^{hash hash}
\begintt
\def\cardinality#1\to#2{#2=0 \long\def\\##1{\advance#2 by1 }#1}
\endtt
(Parameter |#2| is supposed to be the name of a count register.) \
And here's a way to take a list macro and center all its items
on individual lines within a |\vbox|:
\begintt
\def\centerlist#1{\def\\##1{\relax##1\cr}%
  \vbox{\halign{\hfil##\hfil\cr#1}}}
\endtt
A particular item can be selected by its position number from the left:
\begintt
\def\select#1\of#2\to#3{\def#3{\outofrange}%
  \long\def\\##1{\advance#1-1 \ifnum#1=0 \def#3{##1}\fi}#2}
\endtt
(Here |#1| is a count register, |#2| is a list macro, and |#3| is a
control sequence.) \ And so on; hundreds of other applications can
be imagined.\footnote\dag{The concept of a list macro is strongly related
to the concept of a list procedure in a programming language;
see {\sl Communications of the ACM\/ \bf7} (1964), 280.}

\TeX\ does all of the preceding operations efficiently, in the sense
that the running time will be proportional to the length of the list
macro involved. It's natural to ask if the rightmost item can be
removed with equal efficiency, since the final item of a list is
somewhat hard to isolate. There is apparently no way to delete the
$n$th item of an $n$-item list in order~$n$ steps, maintaining
complete generality, unless the ^|\aftergroup| trick (by which
we created a macro that expands to $n$~asterisks) is used; and the
|\aftergroup| trick is somewhat unattractive in the list application,
because the list might be quite long.\footnote\ddag{The interested reader may
enjoy constructing a macro that removes the $k$th item of an $n$-item list
macro~|\l| in $O(n\log n)$ steps, given $k$ and~|\l|,
without using |\aftergroup|.} However, if we restrict list items to
unexpandable tokens, it turns out to be possible to remove the
^^|\xyzzy| rightmost item quite efficiently:
\begintt
\def\deleterightmost#1{\edef#1{\expandafter\xyzzy#1\xyzzy}}
\long\def\xyzzy\\#1#2{\ifx#2\xyzzy\yzzyx
  \else\noexpand\\{#1}\fi\xyzzy#2}
\long\def\yzzyx#1\xyzzy\xyzzy{\fi}
\endtt
Careful study of this example shows that \TeX's ^{mouth} is capable of
doing ^{recursive} operations, given sufficiently tricky macros.

The contents of a ^|\count| register can easily be converted to decimal
and stored in a control sequence; for example, if\/ |\n| is a register,
`|\edef\csn{\the\n}|' puts its value into~|\csn|. Conversely, a value
from |\csn| can be put back into |\n| by saying simply `|\n=\csn|'.
There's usually no point in doing this transformation just to minimize
the usage of\/ |\count| registers, since \TeX\ has 256 of them; but a
decimal representation like the expansion of\/ |\csn| can be stored in a
list macro, and that might be useful in some applications. Incidentally,
there's a neat way to test if such a control-sequence-number is zero:
`^|\if||0\csn|\<true text>|\else|\<false text>|\fi|' works because
extra digits of a nonzero number will be ignored with the \<true text>.

A technique something like list macros can be used to maintain
unordered sets of control sequences. In this case it's
convenient to leave off the braces; for example,
\begintt
\def\l{\\\alpha\\\beta\\\gamma}
\endtt
defines a ``^{set macro}'' |\l| that represents the control
sequences $\{\,\hbox{|\alpha|},\hbox{|\beta|},\hbox{|\gamma|}\,\}$.
A straightforward construction tests whether a given control sequence
is in the set:
\begintt
\def\ismember#1\of#2{\resultfalse\def\given{#1}%
  \def\\##1{\def\next{##1}\ifx\next\given\resulttrue\fi}#2}
\endtt
And an efficient but not-so-straightforward construction removes all
occurrences of control sequences that are
|\ifx|-equivalent to a given control sequence:
\begintt
\def\remequivalent#1\from#2{\let\given=#1%
  \ifx#2\empty\else\edef#2{\expandafter\plugh#2\plugh}\fi}
\def\plugh\\#1#2{\ifx#1\given\else\noexpand\\\noexpand#1\fi
  \ifx#2\plugh\hgulp\fi\plugh#2}
\def\hgulp\fi\plugh\plugh{\fi}
\endtt

%\subsection Verbatim listing. ^^{Verbatim}
%Plain \TeX\ includes a macro called
%^|\dospecials| that is essentially a set macro, representing the set
%of all characters that have a special category code. \ (The control
%sequence ^|\do| plays the r\^ole of\/ |\\| in the discussion above.) \
%Therefore it's easy to change all of the special characters to
%category~12~(other):
%\begintt
%\def\uncatcodespecials{\def\do##1{\catcode`##1=12 }\dospecials}
%\endtt
%This works even when the set of special characters has been changed,
%provided that |\dospecials| has been updated to represent the current set.
\subsection 逐字呈现. ^^{Verbatim}
Plain \TeX\ includes a macro called
^|\dospecials| that is essentially a set macro, representing the set
of all characters that have a special category code. \ (The control
sequence ^|\do| plays the r\^ole of\/ |\\| in the discussion above.) \
Therefore it's easy to change all of the special characters to
category~12~(other):
\begintt
\def\uncatcodespecials{\def\do##1{\catcode`##1=12 }\dospecials}
\endtt
This works even when the set of special characters has been changed,
provided that |\dospecials| has been updated to represent the current set.

The operation |\uncatcodespecials| just defined is important, of course,
when \TeX's automatic features need to be temporarily disabled. Let's
suppose that we want to create a listing of some computer file,
reproducing the characters and the spacing exactly as they appear in the
file. To make the problem more interesting, let's also print line
numbers in front of each line, as in the listing of |story.tex| on
page~\storypage. To make the problem simpler, let's assume that the file
contains only standard ASCII printing characters: no tab marks or
form feeds or such things. Our goal is to devise a |\listing| macro
such that, e.g., `|\listing{story}|' will insert a listing of
the |story.tex| file into a manuscript, after which \TeX's normal
conventions will be restored. The listing should be in ^|\tt| type.
A macro of the following form meets the desired specifications:
\begintt
\def\listing#1{\par\begingroup\setupverbatim\input#1 \endgroup}
\endtt
Notice that the ^|\endgroup| command here will nicely ``turn off'' all
the weird things that ^|\setupverbatim| turns on. Notice also that the
commands `|\input#1 \endgroup|' will not be listed verbatim, even though
they follow |\setupverbatim|, since they entered \TeX's reading mechanism
when the |\listing| macro was expanded (i.e., before the verbatim business
was actually set up).

But what should |\setupverbatim| do? Well, it ought to include
^|\obeylines|, since this automatically inserts a ^|\par| at the end of
each line that is input; it ought to include ^|\uncatcodespecials|, so
that special characters print as themselves; and it ought to include
^|\obeyspaces|, so that each space counts. But we need to look carefully
at each of these things to see exactly what they do: \
(1)~Plain \TeX's |\obeylines| macro changes the ^|\catcode| of |^^M|
to |\active|, and then it says `|\let^^M=\par|'. Since |^^M| is placed
at the end of each line, this effectively ends each line with |\par|;
however, |\obeylines| doesn't say `|\def^^M{\par}|', so we must make
any desired changes to |\par| before invoking |\obeylines|. \
(2)~The |\uncatcodespecials| operation changes a space to category~12;
but the |\tt| font has the character `\]' in the \<space> position,
so we don't really want \]$_{12}$. \
(3)~The |\obeyspaces| macro in Appendix~B merely changes the \<space>
character to category~13; ^{active character} \]$_{13}$ has been
defined to be the same as ^|\space|, a macro that expands to \]$_{10}$.
This is usually what is desired; for example, it means that spaces in
constructions like `|\hbox to 10 pt {...}|' won't cause any trouble.
But in our application it has an undesirable effect, because it
produces spaces that are affected by the ^{space factor}. To defeat this
feature, it's necessary either to say ^|\frenchspacing| or to
redefine \]$_{13}$ to be the same as |\|\]. ^^{control space}
The latter alternative is better, because the former will discard spaces
at the beginning of each line.

The |\setupverbatim| macro should also take care of putting a line
number into the position of the paragraph indentation. We can take care
of this by introducing a counter variable and using ^|\everypar|, as follows:
\begintt
\newcount\lineno % the number of file lines listed
\def\setupverbatim{\tt \lineno=0
  \obeylines \uncatcodespecials \obeyspaces
  \everypar{\advance\lineno by1 \llap{\sevenrm\the\lineno\ \ }}}
{\obeyspaces\global\let =\ } % let active space = control space
\endtt
^^|\llap|
In theory, this seems like it ought to work; but in practice, it fails
in two ways. One rather obvious failure---at least, it becomes obvious
when the macro is tested---is that all the empty lines of the file
are omitted. The reason is that the |\par| command at the end of an empty
line doesn't start up a new paragraph, because it occurs in vertical mode.
The other failure is not as obvious, because it occurs much less often:
The |\tt| fonts contain ^{ligatures} for Spanish punctuation, so the
sequences |?||`| and~|!||`| will be printed as {\tt?`} and~{\tt!`}
respectively.  Both of these defects can be cured by inserting
\begintt
\def\par{\leavevmode\endgraf} \catcode`\`=\active
\endtt
before |\obeylines| in the |\setupverbatim| macro, and by defining |`|$_{13}$
as follows:
\begintt
{\catcode`\`=\active \gdef`{\relax\lq}}
\endtt
A similar scheme could be used to produce verbatim listings in other fonts;
but more characters would have to be made active, in order to break
ligatures and to compensate for ASCII characters that aren't present.

\def\uncatcodespecials{\def\do##1{\catcode`##1=12 }\dospecials}
\def\verbatim{\begingroup\tt\uncatcodespecials\obeyspaces\doverbatim}
\newcount\balance
{\catcode`<=1 \catcode`>=2 \catcode`\{=12 \catcode`\}=12
  \gdef\doverbatim{<\balance=1\verbatimloop>
  \gdef\verbatimloop#1<\def\next<#1\verbatimloop>%
    \if#1{\advance\balance by1
    \else\if#1}\advance\balance by-1
     \ifnum\balance=0\let\next=\endgroup\fi\fi\fi\next>>
Instead of listing a file verbatim, you might want to define a
|\verbatim| macro such that `|\verbatim{$this$|{\tt\ is }|{\it!}}|'
yields `\verbatim{$this$ is {\it!}}'. It's somewhat dangerous to change
^{category codes}, because \TeX\ stamps the category on each character
when that character is first read from a file. Thus, if\/ |\verbatim|
were defined by a construction of the form
|\long\def\verbatim#1{|\<something>|}|, argument~|#1| would already be
converted to a list of tokens when \<something> starts; |\catcode| changes
would not affect the argument.  The alternative is to change category
codes before scanning the argument to |\verbatim|:
\begintt
\def\verbatim{\begingroup\tt\uncatcodespecials
  \obeyspaces\doverbatim}
\newcount\balance
{\catcode`<=1 \catcode`>=2 \catcode`\{=12 \catcode`\}=12
  \gdef\doverbatim{<\balance=1\verbatimloop>
  \gdef\verbatimloop#1<\def\next<#1\verbatimloop>%
    \if#1{\advance\balance by1
    \else\if#1}\advance\balance by-1
     \ifnum\balance=0\let\next=\endgroup\fi\fi\fi\next>>
\endtt
This works; but it's slow, and it allows verbatim setting only of text
that has balanced braces. It would not be suitable for typesetting
the examples in a book like {\sl The \TeX book}. \ (Appendix~E contains
the verbatim macros that were actually used.) \ Note also that if this
|\verbatim{...}| macro appears in the argument to another macro
like |\centerline|, it will fail because the category codes can no longer be
changed.  The ^|\footnote| macro in Appendix~B is careful to avoid
scanning its argument prematurely; it uses ^|\bgroup| and ^|\egroup| in a
somewhat tricky way, so that category code changes are permitted inside
plain \TeX\ footnotes.

On the other hand, there is a fairly fast way to convert a token list
to an almost-verbatim transcript:
\begintt
\long\def\verbatim#1{\def\next{#1}%
  {\tt\frenchspacing\expandafter\strip\meaning\next}}
\def\strip#1>{}
\endtt
Tokens are stripped off in this construction since, for example,
^|\meaning||\next| might be `\def\next{$this$ is {\it!}}%
\expandafter|\meaning\next|'. Notice that a space will be inserted after
the control word |\it|, but no space might actually have occurred there in
the argument to |\verbatim|; such information has been irretrievably lost.

One of the problems with verbatim mode is that it's hard to stop;
if we turn off all of \TeX's normal control capabilities, we end up
``painting ourselves into a corner'' and reaching a point of no return.
The |\listing| macro was able to solve this problem because the end of
a file brings an old token list back to life. Another solution would be
to specify a certain number of lines, after which verbatim mode should
end. Otherwise it's necessary to put some constraint on the text,
i.e., to make certain texts unprintable in verbatim mode. For example,
here's an approach that typesets everything between |\beginverbatim| and
|\endverbatim|, assuming only that the control sequence |\endverbatim|
does not need to be set:
\begintt
\def\beginverbatim{\par\begingroup\setupverbatim\doverbatim}
{\catcode`\||=0 \catcode`\\=12 % || is temporary escape character
  ||obeylines||gdef||doverbatim^^M#1\endverbatim{#1||endgroup}}
\endtt
This construction assumes that |\beginverbatim| appears at the end of a
line in the manuscript file.  Argument |#1| will be read entirely into
\TeX's memory before anything happens, so the total amount of verbatim
material had better not be too voluminous. Incidentally, it isn't
necessary to say that this macro is ^|\long|, because the |\par|'s
inserted by ^|\obeylines| are really |^^M|'s.

Another approach is to keep one character untouchable. For example,
it's possible to define things so that `|\verbatim|\<char>\<text>\<char>'
will typeset the \<text> verbatim, where the \<text> is not supposed
to contain any occurrences of the repeated delimiter \<char>:
\begintt
\def\verbatim{\begingroup\setupverbatim\doverbatim}
\def\doverbatim#1{\def\next##1#1{##1\endgroup}\next}
\endtt

%\subsection Selective loading of macros. Some interesting problems arise
%when a computer system acquires a large library of macro files. For
%example, suppose that a file |macs.tex| contains the lines
%\begintt
%\let\italcorr=\/
%\def\/{\unskip\italcorr}
%\endtt
%because somebody thought it would be nice to allow an optional space
%before \TeX's primitive ^|\/| command. That's fine, except if |macs.tex|
%is input twice; for example, two other macro files might both say
%^|\input|~|macs|. When those lines are processed the second time,
%|\italcorr| will be |\let| equal to a macro that expands to
%`|\unskip\italcorr|',
%and you can guess what will happen: \TeX\ will get into an infinite loop,
%stoppable only by ^{interrupt}ing the program manually.
%^^{recursion, infinite}
\subsection 可选的宏载入. Some interesting problems arise
when a computer system acquires a large library of macro files. For
example, suppose that a file |macs.tex| contains the lines
\begintt
\let\italcorr=\/
\def\/{\unskip\italcorr}
\endtt
because somebody thought it would be nice to allow an optional space
before \TeX's primitive ^|\/| command. That's fine, except if |macs.tex|
is input twice; for example, two other macro files might both say
^|\input|~|macs|. When those lines are processed the second time,
|\italcorr| will be |\let| equal to a macro that expands to
`|\unskip\italcorr|',
and you can guess what will happen: \TeX\ will get into an infinite loop,
stoppable only by ^{interrupt}ing the program manually.
^^{recursion, infinite}

Fortunately there's an easy way to prevent this problem, by placing
a suitable interlock near the beginning of every macro file that
might introduce such anomalies:
\begintt
\ifx\macsisloaded\relax\endinput\else\let\macsisloaded=\relax\fi
\endtt
Then |\macsisloaded| will be undefined at the time of the first~|\ifx|,
but the file will not be read twice. A different control sequence
should, of course, be used for each file.

Another difficulty with large sets of macros is that they take up space.
It would be nice to preload every macro that every \TeX\ user has ever
dreamed up; but there might not be enough room,
because \TeX's memory capacity is finite. You might find it necessary
to hold back and to load only the macros that are really needed.

How much memory ^{space} ^^{efficiency} does a macro require?
Well, there are four kinds of memory involved: token memory,
name memory, string memory, and character memory. \ (If any of these
becomes too full, it will be necessary to increase what \TeX\ calls the
macro memory size, the hash size, the number of strings, and/or the
pool size, respectively; see Chapter~27.) \ The token memory is
most important; a macro takes one cell of token memory for each token
in its definition, including the `|{|' and the `|}|'. For example,
the comparatively short definition
\begintt
\def\example#1\two{\four}
\endtt
takes five tokens: |#1|, \cstok{two}, |{|$_1$, \cstok{four}, and |}|$_2$.
Each control sequence also takes up one cell of name memory, one cell
of string memory, and as many cells of character memory as there are
characters in the name (seven in the case of\/ |\example|). Character
memory is comparatively cheap; four characters, or in some cases five,
will fit in the same number of bits as a single cell of token memory,
inside the machine. Therefore you don't save much by choosing short
macro names.

\TeX\ will tell you how close you come to exceeding its current
memory capacity if you say ^|\tracingstats||=1|. For example, one of the
runs that the author made while testing galley proofs of this appendix
reported the following statistics:
\begintt
Here is how much of TeX's memory you used:
 209 strings out of 1685
 1659 string characters out of 17636
 27618 words of memory out of 52821
 1172 multiletter control sequences out of 2500
\endtt
Consequently there was plenty of room for more macros: $52821-27618=
25203$ unused cells of main memory, $2500-1172=1328$ of name memory,
$1685-209=1476$ of string memory, and $17636-1659=15977$ of character memory.
But a fairly large \TeX\ was being used, and only the macros of
Appendices B and~E were loaded; in other circumstances it might have
been necessary to conserve space.

One obvious way to keep from loading too many macros is to keep
the macro files short and to |\input| only the ones that you need.
But short files can be a nuisance; sometimes there's a better way.
For example, let's suppose that a file contains five optional classes
of macros called |A|, |B|, |C|, |D|,~|E|, and that a typical user
will probably want only at most two or three of these five; let's
design a |\load| macro so that, for example,
`|\load{macs}{AC}|'
will load file |macs.tex| including options |A| and |C| but not options
|B|, |D|, or |E|. The following |\load| macro converts its second argument
into a set~macro called |\options|:
\begintt
\def\load#1#2{\let\options=\empty \addoptions#2\end \input#1 }
\def\addoptions#1{\ifx#1\end \let\next=\relax
  \else\let\\=\relax\edef\options{\options\\#1}%
   \let\next=\addoptions \fi \next}
\endtt
Inside the file |macs.tex|, a portion of code that should be loaded only
under option~|B|, say, can be enclosed by `|\ifoption B ... \fi|', where
|\ifoption| is defined thus:
\begintt
\def\ifoption#1{\def\\##1{\if##1#1\resulttrue\fi}%
  \resultfalse \options \ifresult}
\endtt
(This is a simple application of ideas presented earlier in this appendix.)

However, the |\ifoption...\fi| scheme isn't very robust, because it requires
all of the macros in the optional part to be well nested with
respect to |\if...| and |\fi|; a macro like |\ifoption| itself couldn't
easily be defined in such a place! There's a better scheme that also
runs faster, based on ^{category code} changes. This idea (due to
Max ^{D\'\i az}) requires that the leftmost nonblank character on each
line be either `|\|' or~`|{|'; it's usually easy to arrange this.
Furthermore, one other symbol, say |~|, is reserved. Then the text
material that is to be loaded only under option~|B| is preceded by
the line `|\beginoption|~|B|' and followed by a line that says
`|~endoptionalcode|'. The ^|\catcode| for~|~| is set to~14 (comment
character), hence the |~endoptionalcode| line will have no effect if
code is not being skipped. The |\beginoption| macro works like this:
\begintt
\def\beginoption#1{\ifoption#1\else\begingroup\swapcategories\fi}
\def\swapcategories{\catcode`\\=14 \catcode`\{=14 \catcode`\~=0 }
\let\endoptionalcode=\endgroup
\catcode`\~=14
\endtt
Once the categories have been swapped, all lines will be skipped at high speed
until the control sequence |~endoptionalcode| is encountered; then
everything will be restored to its former state. Under this scheme,
material that should be loaded only under both options |B| and~|D|
can be prefaced by both `|\beginoption|~|B|' and `|\beginoption|~|D|';
material that should be loaded under either option~|B| or option~|D|
(or both) can be prefaced by
\begintt
\beginoption B
~oroption D
\endtt
if we define |\oroption#1| to be an abbreviation for
`|\ifoption#1\endgroup\fi|'.

Another kind of selective loading is sometimes appropriate, based on
whether or not a particular control sequence is defined. In this
scheme, if the control sequence is undefined, it should remain
undefined and it should take up no space whatever in \TeX's memory.
There's a slick way to do this, namely to say
\begintt
\ifx\cs\undefined ... \fi
\endtt
(assuming that ^|\undefined| has never been defined). \TeX\ does not
put undefined control sequences into its internal tables if they follow
^|\ifx| or if they are encountered while skipping ^{conditional text}.
You can use this idea, for example, to prepare a bibliography for a paper,
by reading a suitably arranged bibliography file; only the entries
that correspond to defined control sequences will be loaded.

%\subsection Brace hacks. Several of \TeX's operations depend on
%^{grouping}, and you'll want to know exactly what this means if
%you try to do certain tricky things. For example, plain \TeX's
%control sequences ^|\bgroup| and ^|\egroup| are ``^{implicit braces}''
%because they have been defined by
%\begintt
%\let\bgroup={  \let\egroup=}
%\endtt
%This means that you can include them in the replacement texts of
%definitions without worrying about how they nest; for example, the
%macros
%\begintt
%\def\beginbox{\setbox0=\hbox\bgroup}
%\def\endbox{\egroup\copy0 }
%\endtt
%allow you to make a box between |\beginbox| and |\endbox|; the behavior
%is almost the same as
%\begintt
%\def\beginbox#1\endbox{\setbox0=\hbox{#1}\copy0 }
%\endtt
%but different in three important ways: (1)~The first alternative allows
%category codes to change inside the box. (2)~The first alternative
%is faster, because it doesn't need to scan the box contents both as an
%argument and as a sequence of actual commands. (3)~The first alternative
%takes less memory space, because no argument needs to be stored.
%Thus, the first alternative is usually superior.
\subsection 花括号技巧. Several of \TeX's operations depend on
^{grouping}, and you'll want to know exactly what this means if
you try to do certain tricky things. For example, plain \TeX's
control sequences ^|\bgroup| and ^|\egroup| are ``^{implicit braces}''
because they have been defined by
\begintt
\let\bgroup={  \let\egroup=}
\endtt
This means that you can include them in the replacement texts of
definitions without worrying about how they nest; for example, the
macros
\begintt
\def\beginbox{\setbox0=\hbox\bgroup}
\def\endbox{\egroup\copy0 }
\endtt
allow you to make a box between |\beginbox| and |\endbox|; the behavior
is almost the same as
\begintt
\def\beginbox#1\endbox{\setbox0=\hbox{#1}\copy0 }
\endtt
but different in three important ways: (1)~The first alternative allows
category codes to change inside the box. (2)~The first alternative
is faster, because it doesn't need to scan the box contents both as an
argument and as a sequence of actual commands. (3)~The first alternative
takes less memory space, because no argument needs to be stored.
Thus, the first alternative is usually superior.

For the purposes of this discussion we shall assume that only `|{|'
has category~1 and that only `|}|' has category~2, although any
characters can actually be used as group delimiters. ^^{braces}
Group nesting is crucial during two of \TeX's main activities:
(a)~when \TeX\ is scanning a ^\<balanced text>, e.g., when \TeX\ is
forming the replacement text of a macro, a parameter, or a token list
variable; (b)~when \TeX\ must determine whether the token
|&|~or ^|\span| ^^{ampersand}
or ^|\cr| or~^|\crcr| is the end of an entry within an ^{alignment}.

\TeX's mouth has two internal counting mechanisms to deal with nesting:
The ``master counter'' goes up by~1 for each |{|$_1$ scanned by
\TeX, and down by~1 for each~|}|$_2$; the ``balance counter'' is
similar, but it is affected only by explicit |{|$_1$ and |}|$_2$ tokens
that are actually contributed to a token list that is being formed.
The master counter decreases by~1 when \TeX\ evaluates the
^{alphabetic constant} |`{|, and it increases by~1 when \TeX\ evaluates
|`}|, hence the net change is zero when such constants are evaluated.
As a consequence of these rules, certain constructions produce
the following effects:
$$\halign{\indent\hfil#\hfil&\quad\hfil#\hfil&
  \quad\hfil#\hfil&\qquad\hfil#\hfil&\quad\hfil#\hfil\cr
&\it Master counter change\span\omit&\it Balance counter change\span\omit\cr
\it Input&\it expanded&\it unexpanded&\it expanded&\it unexpanded\cr
|{|&1&1&1&1\cr
|\bgroup|&0&0&0&0\cr
|\iffalse{\fi|&1&1&0&1\cr
|\ifnum0=`{\fi|&0&1&0&1\cr}$$
The last two cases produce no begin-group tokens when expanded, but they
do affect the master counter as shown. Thus, for example,
\begintt
\def\eegroup{\ifnum0=`{\fi}}
\endtt
makes |\eegroup| behave rather like |\egroup|, but the expansion of\/
|\eegroup| also decreases the master counter.

Alignment processing uses only the master counter, not the balance counter.
An alignment entry ends with the first |&|~or |\span| or |\cr| or~|\crcr|
that appears when the master counter has the value that was present
in the counter at the beginning of the entry. Thus, for example,
the curious construction
\begintt
\halign{\show\par#\relax\cr
  \global\let\par=\cr
  {\global\let\par=\cr}\cr
  \par}
\endtt
causes \TeX\ to perform three |\show| instructions, in which the
respective values of\/ |\par| shown are |\par|, |\relax|, and~|\cr|.
Similarly, each template in the preamble to an alignment ends with
the first |&| or |\cr| or |\crcr| that appears at the master counter level
that was in effect at the beginning of the entry; hence |&|~and |\cr| and
|\crcr| tokens can appear within a template of an alignment, if they
are hidden by braces (e.g., if they appear in a definition).

These facts allow us to draw two somewhat surprising conclusions: (1)~If
an alignment entry has the form `$\,\alpha\,$^|\iffalse||{\fi|$\,\beta\,$%
|\iffalse}\fi|$\,\gamma\,$', it's possible for $\beta$ to include
|&|~and~|\cr| tokens that aren't local to a group.\footnote*{The
token list $\alpha$ should not be empty, however, because \TeX\ expands
the first token of an alignment entry before looking at the template,
in order to see if the entry begins with |\noalign| or |\omit|. The master
counter value that is considered to be present at the beginning of an
entry is the value in the counter just after the ``$u$~part'' of the
template has been entirely read.} (2)~The construction
\begintt
{\span\iffalse}\fi
\endtt
appearing in a preamble contributes `|{|' to the template without
any net change to the master counter; thus, it's very much like
|\bgroup|, except that it produces |{|$_1$ explicitly.
If you understand (1) and~(2), you'll agree that the present appendix
deserves its name.

%\def\pmb#1{\setbox0=\hbox{#1}%
%  \kern-.025em\copy0\kern-\wd0\kern.05em\copy0
%  \kern-\wd0\kern-.025em\raise.0433em\box0 }
%\subsection Box maneuvers. ^^{Box maneuvers}
%Let's turn now from syntax to semantics,
%i.e., from \TeX's mouth to its gastro-intestinal tract. Sometimes an odd
%symbol is needed in boldface type, but it's available only in a normal
%weight. In such cases you can sometimes get by with ``^{poor man's bold},''
%obtained by overprinting the normal weight symbol with slight offsets.
%The following macro typesets its argument three times in three
%slightly different places, equidistant from each other; but the result
%takes up just as much space as if\/ |\pmb| had been simply |\hbox|:
%\begintt
%\def\pmb#1{\setbox0=\hbox{#1}%
%  \kern-.025em\copy0\kern-\wd0
%  \kern.05em\copy0\kern-\wd0
%  \kern-.025em\raise.0433em\box0 }
%\endtt
%For example, `|\pmb{$\infty$}|' yields `\pmb{$\infty$}'. The results
%are somewhat \pmb{fuzzy}, and they certainly are no match for the
%real thing if it's available; but poor man's bold is better than nothing,
%and once in a~while you can get away with it.
\def\pmb#1{\setbox0=\hbox{#1}%
  \kern-.025em\copy0\kern-\wd0\kern.05em\copy0
  \kern-\wd0\kern-.025em\raise.0433em\box0 }
\subsection 操纵盒子. ^^{Box maneuvers}
Let's turn now from syntax to semantics,
i.e., from \TeX's mouth to its gastro-intestinal tract. Sometimes an odd
symbol is needed in boldface type, but it's available only in a normal
weight. In such cases you can sometimes get by with ``^{poor man's bold},''
obtained by overprinting the normal weight symbol with slight offsets.
The following macro typesets its argument three times in three
slightly different places, equidistant from each other; but the result
takes up just as much space as if\/ |\pmb| had been simply |\hbox|:
\begintt
\def\pmb#1{\setbox0=\hbox{#1}%
  \kern-.025em\copy0\kern-\wd0
  \kern.05em\copy0\kern-\wd0
  \kern-.025em\raise.0433em\box0 }
\endtt
For example, `|\pmb{$\infty$}|' yields `\pmb{$\infty$}'. The results
are somewhat \pmb{fuzzy}, and they certainly are no match for the
real thing if it's available; but poor man's bold is better than nothing,
and once in a~while you can get away with it.

When you put something into a box register, you don't need to put
the contents of that register into your document. Thus, you can write
macros that do experiments behind the scenes, trying different
possibilities before making a commitment to a particular decision.
For example, suppose you are typesetting a text in two languages, and
you would like to choose the column widths so that the same number of
lines is obtained in both cases. For example, the following texts
balance perfectly when the first column is $157.1875\pt$ wide and the
second column is $166.8125\pt$ wide; but the second column would be
one line longer than the first if they were both $162\pt$ wide:
^^{Alice} ^^{Bill}
\medskip
\hrule
\smallskip
\begingroup
\eightpoint
\tolerance=9999
\hbadness=2300
\finalhyphendemerits=3000000
\doublehyphendemerits=1000000
\parskip=1pt
\parindent=1.5em
\frenchspacing

\def\firstcol{ % from Surreal Numbers, pp. 46--47
\item{A.}\strut The creative part is really more interesting than the
deductive part. Instead of concentrating just on finding good
answers to questions, it's more important to learn how to find
good questions!

\item{B.} You've got something there. I wish our teachers would
give us problems like, ``Find something interesting about $x$,''
instead of ``Prove $x$.''

\item{A.}Exactly. But teachers are so conservative, they'd be
afraid of scaring off the ``grind'' type of students who
obediently and mechanically do all the homework. Besides, they
wouldn't like the extra work of grading the answers to
nondirected questions.

\item{}The traditional way is to put off all creative aspects
until the last part of graduate school. For seventeen or more
years, a student is taught exams\-man\-ship, then suddenly after
passing enough exams in graduate school he's told to do
something original.
}
\def\\#1{\raise.5pt\hbox{$\scriptscriptstyle
    \ifx#1`\langle\!\langle\else\rangle\!\rangle\fi$}% Spanish quote marks
  \ifx#1`\nobreak\hskip0pt \fi} % allow hyphenation
%\showhyphens{\\`Demuestre\\' \\`apisonadora\\' \\`Encuentren\\'}
\def\secondcol{ % from N\'umeros Surreales, pp. 39--40
\item{A.} \strut La parte creativa es mucho mejor que la deductiva.
En vez de concentrarse en buscar buenas respuestas a ciertas
cuestiones es m\'as importante aprender a proponerse buenas
preguntas.

\item{B.} Me parece una buena ocurrencia. Me gustar\'\i a
que los profesores propusieran problemas del estilo de
\\`Encuentren algo interesante sobre $x$\\' en vez de
\\`Demuestre que $x\ldots\,$\\'.

\item{A.} Exactamente. Pero los profesores son tan conservadores
que temer\'\i an espantar al tipo de estudiante \\`apisonadora\\'
que hace lo que le proponen para casa, obe\-dien\-te\-mente y de forma
mec\'anica. Adem\'as, no creo que les gustase el trabajo adicional
de calificar respuestas a preguntas abiertas.

\item{}La forma tradicional es dejar la parte creativa para los cursos
altos. Durante diecisiete a\~nos o m\'as se ense\~na al es\-tu\-diante a
aprobar, luego de golpe, cerca de la graduaci\'on, se le pide que haga
algo original.
}
\newif\iffail \newdimen\doublewidth \newdimen\delheight
\newcount\n
\newdimen\trialwidth \newdimen\lowwidth \newdimen\highwidth
\wlog{Beginning the binary search for balanced columns:}
\def\balancetwocols{\lowwidth=10em % the lowest feasible \trialwidth
  \highwidth=\doublewidth \advance\highwidth-10em % the highest ditto
  {\n=1 \hbadness=10000 \hfuzz=\maxdimen
    \loop \maketrial
    \wlog{trial \the\n, \the\trialwidth: (\the\ht0,\the\ht2)}
    \testfailure \iffail \preparenewtrial \repeat}
  \maketrial} % now under/overfull boxes will be shown
\def\maketrial{%
  \trialwidth=.5\lowwidth \advance\trialwidth by.5\highwidth
  \setbox0=\vbox{\hsize=\trialwidth \firstcol}
  \setbox2=\vbox{\hsize=\doublewidth
    \advance\hsize-\trialwidth \secondcol}}
\def\testfailure{\dimen0=\ht0 \advance\dimen0-\ht2
  \ifnum\dimen0<0 \dimen0=-\dimen0 \fi
  \ifdim\dimen0>\delheight \ifnum\n=10 \failfalse\else\failtrue\fi
  \else\failfalse\fi}
\def\preparenewtrial{\ifdim\ht0>\ht2 \global\lowwidth=\trialwidth
  \else\global\highwidth=\trialwidth\fi \advance\n by1 }
\doublewidth=27pc \delheight=0pt
\balancetwocols
\line{\box0\hfil\box2}
\endgroup
\smallskip
\hrule
\medskip
Some implementations of \TeX\ display the output as you are running,
so that you can choose column widths interactively until a suitable
balance is obtained. It's fun to play with such systems, but it's also
possible to ask \TeX\ to compute the column widths automatically. The
following code tries up to ten times to find a solution in which the
natural heights of the two columns are different by less than
a given value, |\delheight|. The macros |\firstcol| and
|\secondcol| are supposed to generate the columns, and the sum
of column widths is supposed to be |\doublewidth|.
\beginlines
|\newdimen\doublewidth \newdimen\delheight \newif\iffail \newcount\n|
|\newdimen\trialwidth \newdimen\lowwidth \newdimen\highwidth|
|\def\balancetwocols{\lowwidth=10em % lower bound on \trialwidth|
|  \highwidth=\doublewidth \advance\highwidth-10em % upper bound|
|  {\n=1 \hbadness=10000 \hfuzz=\maxdimen % disable warnings|
|    \loop \maketrial \testfailure \iffail \preparenewtrial \repeat}|
|  \maketrial} % now under/overfull boxes will be shown|
|\def\maketrial{\trialwidth=.5\lowwidth \advance\trialwidth by.5\highwidth|
|  \setbox0=\vbox{\hsize=\trialwidth \firstcol}|
|  \setbox2=\vbox{\hsize=\doublewidth\advance\hsize-\trialwidth\secondcol}}|
|\def\testfailure{\dimen0=\ht0 \advance\dimen0-\ht2|
|  \ifnum\dimen0<0 \dimen0=-\dimen0 \fi|
|  \ifdim\dimen0>\delheight \ifnum\n=10 \failfalse\else\failtrue\fi|
|  \else\failfalse\fi}|
|\def\preparenewtrial{\ifdim\ht0>\ht2 \global\lowwidth=\trialwidth|
|  \else\global\highwidth=\trialwidth\fi \advance\n by1 }|
\endlines
Neither column will be less than 10~ems wide.  This code does a ``^{binary
search},'' assuming that a column will not increase in height when it is
made wider. If no solution is found in 10~trials, there probably is no way
to obtain the desired balance, because a tiny increase in the width of the
taller column will make it shorter than the other one.  The values of\/
^|\hbadness| and ^|\hfuzz| are made infinite during the trial settings,
because warning messages that relate to unused boxes are irrelevant; after
a solution is found, it is computed again, so that any relevant warnings
will be issued.

When a box has been put into a box register, you can change its
height, width, or depth by assigning a new value to the |\ht|, |\wd|,
or |\dp|. Such assignments don't change anything inside the box;
in particular, they don't affect the setting of the glue.

But changes to a box's dimensions can be confusing if you don't understand
exactly how \TeX\ deals with boxes in lists. The rules are stated in
Chapter~12, but it may be helpful to restate them here in a different way.
Given a box and the location of its reference point, \TeX\ assigns
locations to interior boxes as follows:  \ (1)~If the box is an hbox,
\TeX\ starts at the reference point and walks through the horizontal list
inside. When the list contains a box, \TeX\ puts the reference point of
the enclosed box at the current position, and moves right by the width of
that box.  When the list contains glue or kerning, etc., \TeX\ moves right
by the appropriate amount. \ (2)~If the box is a vbox, \TeX\ starts at the
upper left corner (i.e., \TeX\ first moves up from the reference point, by
the height of the box) and walks through the vertical list inside. When
the list contains a box, \TeX\ puts the upper left corner of that
box at the current position; i.e., \TeX\ moves down by the height of that
box, then puts the box's reference point at the current position, then
moves down by the depth of the box. When the list contains glue or
kerning, etc., \TeX\ moves down by the appropriate amount.

As a consequence of these rules, we can work out what happens when
the dimensions of a box are changed. Let |\delta| be a \<dimen>
register, and let |\h| and |\hh| specify horizontal lists that
don't depend on |\box0|. Consider the following macro:
\begintt
\newdimen\temp \newdimen\delta
\def\twohboxes#1{\setbox1=\hbox{\h \copy0 \hh}
  \temp=#10 \advance\temp by \delta #10=\temp
  \setbox2=\hbox{\h \copy0 \hh}}
\endtt
For example, |\twohboxes\wd| makes two hboxes, |\box1| and |\box2|,
that are identical except that the width of\/ |\box0| has been
increased by $\delta$ in |\box2|. What difference does this make?
There are several cases, depending on whether |#1| is |\wd|, |\ht|,
or |\dp|, and depending on whether |\box0| is an hbox or a vbox.
\ {\it Case~1}, |\twohboxes\wd|: The material from |\hh| is
moved right by $\delta$ in |\box2|, compared to its position
in |\box1|. Also |\wd2| is $\delta$ more than~|\wd1|.
\ {\it Case~2}, |\twohboxes\ht|: If\/ |\box0| is an hbox, everything
remains in the same position; but if\/ |\box0| is a vbox, everything
in |\copy0| moves up by $\delta$. Also |\ht2| may differ from |\ht1|.
\ {\it Case~3}, |\twohboxes\dp|: Everything remains in the
same position, but |\dp2| may differ from |\dp1|.

Similarly, we can work out the changes when box dimensions are
changed for boxes within vertical lists. In this case we shall
ignore the influence of interline glue by defining |\twovboxes| as follows:
\begintt
\def\twovboxes#1{
  \setbox1=\vbox{\v\nointerlineskip\copy0\nointerlineskip\vv}
  \temp=#10 \advance\temp by \delta #10=\temp
  \setbox2=\vbox{\v\nointerlineskip\copy0\nointerlineskip\vv}}
\endtt
What is the difference between |\box1| and |\box2| now?
\ {\it Case~1}, |\twovboxes\wd|: Everything remains in the same
position, but |\wd2| may differ from |\wd1|.
\ {\it Case~2}, |\twovboxes\ht|: If\/ |\box0| is an hbox, everything
in |\v| moves up by~$\delta$ in |\box2|, compared to
the corresponding positions in |\box1|, if we make the reference
points of the two boxes identical; but if\/ |\box0| is a vbox,
everything in it moves up by~$\delta$, together with the material
in |\v|. Also, |\ht2| is $\delta$~more than |\ht1|.
\ {\it Case~3}, |\twovboxes\dp|:
If\/ |\vv| is empty, |\dp2| is $\delta$~more than~|\dp1|, and
nothing else changes. Otherwise everything in |\v| and in |\copy0|
moves up by~$\delta$, and |\ht2| is $\delta$ more than~|\ht1|.
%\ (If you don't believe it, try it.)

\newdimen\unit
\def\point#1 #2 {\rlap{\kern#1\unit
    \raise#2\unit\hbox{$
      \scriptstyle\bullet\;(#1,#2)$}}}
\unit=\baselineskip
\setbox0=\vtop{\hrule
  \hbox{\vrule height10\unit depth9.4\unit \kern2\unit
    \hbox{\unit=\baselineskip
      \point 0 0 % Alioth (Epsilon Urs\ae\ Majoris), magnitude 1.79
      \point 0 8 % Dubhe (Alpha Urs\ae\ Majoris),    magnitude 1.81
      \point 0 -8 % Alkaid(Eta Urs\ae\ Majoris),     magnitude 1.87
      \point -1 -2.5 % Mizar (Zeta Urs\ae\ Majoris), magnitude 2.26
      \point 4 7 % Merak (Beta Urs\ae\ Majoris),     magnitude 2.37
      \point 4 2 % Phekda (Gamma Urs\ae\ Majoris),   magnitude 2.44
      \point 1 1.5 % Megrez (Delta Urs\ae\ Majoris), magnitude 3.30
      }%        Sources: Atlas of the Universe; Astronomy Data Book
    \kern7\unit \vrule}\hrule}
\dp0=0pt
\dimen@=240pt
\parshape=21
\z@\hsize \z@\hsize
\z@\dimen@ \z@\dimen@ \z@\dimen@ \z@\dimen@ \z@\dimen@ \z@\dimen@
\z@\dimen@ \z@\dimen@ \z@\dimen@ \z@\dimen@ \z@\dimen@ \z@\dimen@
\z@\dimen@ \z@\dimen@ \z@\dimen@ \z@\dimen@ \z@\dimen@ \z@\dimen@
\z@\hsize
\TeX\ is designed to put boxes together either horizontally
or vertically, not diagonally. But that's not a serious
limitation, because the use of negative spacing makes it
possible to put things anywhere on a page. ^^{Ursa Major}
^^{points with arbitrary coordinates} ^^{coordinates}
\vadjust{\kern-\ht0\kern-2pt\rightline{\box0}\kern2pt}%
For example, the seven points in the diagram at the right of this
paragraph were typeset by saying simply
\begintt|hsize=|displaywidth
\hbox{\unit=\baselineskip
  \point 0 0
  \point 0 8
  \point 0 -8
  \point -1 -2.5
  \point 4 7
  \point 4 2
  \point 1 1.5
  }
\endtt
{\count0=\prevgraf \advance\count0 by7 \prevgraf=\count0}%
The |\point| macro makes a box of width zero; hence the individual |\point|
specifications can be given in any order, and there's no restriction on
the coordinates:
\begintt|hsize=|displaywidth
\newdimen\unit
\def\point#1 #2 {\rlap{\kern#1\unit
    \raise#2\unit\hbox{$
      \scriptstyle\bullet\;(#1,#2)$}}}
\endtt
^^|\kern|{\count0=\prevgraf \advance\count0 by3 \prevgraf=\count0}%
If the |\point| specifications are not enclosed in an ^|\hbox|---i.e., if
they occur in vertical mode---a similar construction can be used. In
this case |\point| should create a box whose height and depth are zero:
\begintt
\def\point#1 #2 {\vbox to0pt{\kern-#2\unit
    \hbox{\kern#1\unit$\scriptstyle\bullet\;(#1,#2)$}\vss}
  \nointerlineskip}
\endtt
(The ^|\nointerlineskip| is necessary to prevent interline glue from
messing things up.)

\begingroup
\let\qc=\manual
\catcode`\ =9 \endlinechar=-1 % ignore all spaces (temporarily)
\newcount\dir \newdimen\y \newdimen\w
\newif\ifvisible \let\B=\visibletrue \let\W=\visiblefalse
\newbox\NE \newbox\NW \newbox\SE \newbox\SW \newbox\NS \newbox\EW
\setbox\SW=\hbox{\qc a} \setbox\NW=\hbox{\qc b}
\setbox\NE=\hbox{\qc c} \setbox\SE=\hbox{\qc d}
\w=\wd\SW \dimen0=\fontdimen8\qc
\setbox\EW=\hbox{\kern-\dp\SW \vrule height\dimen0 width\wd\SW} \wd\EW=\w
\setbox\NS=\hbox{\vrule height\ht\SW depth\dp\SW width\dimen0}  \wd\NS=\w
\def\L{\ifcase\dir \dy+\NW \or\dx-\SW \or\dy-\SE \or\dx+\NE\dd-4\fi \dd+1}
\def\S{\ifcase\dir \dx+\EW \or \dy+\NS \or \dx-\EW \or \dy-\NS \fi}
\def\R{\ifcase\dir \dy-\SW\dd+4 \or\dx+\SE \or\dy+\NE \or\dx-\NW\fi \dd-1}
\def\T{\ifcase\dir\kern-\w\dd+2\or\ey-\dd+2\or\kern\w\dd-2\or\ey+\dd-2\fi}
\edef\dd#1#2{\global\advance\dir#1#2\space}
\def\dx#1#2{\ifvisible\raise\y\copy#2 \if#1-\kern-2\w\fi\else\kern#1\w\fi}
\def\dy#1#2{\ifvisible\raise\y\copy#2 \kern-\w \fi \global\advance\y#1\w}
\def\ey#1{\global\advance\y#1\w}
\def\path#1{\hbox{\B \dir=0 \y=0pt #1}}
\catcode`\ =10 \endlinechar='15 % resume normal spacing conventions

If you enjoy fooling around making pictures, instead of typesetting
ordinary text, \TeX\ will be a source of endless frustration/amusement
for you, because almost anything is possible if you have suitable fonts.
For example, suppose you have a font |\qc| that contains four ^{quarter
circles}:
\begindisplay
\def\\#1{\kern\dp#1\copy#1\kern-\dp#1}
\tt a = \\\SW \qquad b = \\\NW \qquad c = \\\NE \qquad d = \\\SE
\enddisplay
Each of these characters has the same height, the same width, and the same
depth; the width and the height-plus-depth are equal to the diameter of
the corresponding full circle. Furthermore, the reference point of each
character is in a somewhat peculiar place: Each quarter arc has a
horizontal endpoint such that the lower edge of the curve is at the
baseline, and a vertical endpoint such that the left edge is directly
above or below the reference point. This convention makes it possible to
guarantee perfect alignment between these characters and rules that meet
them at the endpoints; the thickness of such rules should be
^|\fontdimen||8\qc|.

Given those characters, it's possible to devise macros |\path|, |\L|,
^^{turtle commands}
|\R|, |\S|, and~|\T| such that ^|\path||{|$\langle$any string of\/ |\L|'s, |\R|'s,
|\S|'s, and |\T|'s$\rangle$|}| produces a path that starts traveling East,
but it turns left for each~|\L|, right for each~|\R|, goes straight
for each~|\S|, and turns backward for each~|\T|. Thus, for example,
%|\path{\S\T\R\R\R\L\S\T\R\R\R\L}| yields `\kern\dp\SW
%\smash{\path{\S\T\R\R\R\L\S\T\R\R\R\L}}\kern-\dp\SW',
|\path{\L\T\S\T\R\L\T\S\T\R}| yields `\kern\dp\SW
\smash{\path{\L\T\S\T\R\L\T\S\T\R}}\kern-\dp\SW', and you
can also get the following effects:
\def\sm#1{\smash{$\vcenter{#1}$}}%
\begindisplay
\noalign{\kern5pt}
|\path{\L\R\S\R\S\R\S\S\R\R}|\kern30pt
  \sm{\path{\L\R\S\R\S\R\S\S\R\R}}\cr
\noalign{\kern16pt}
|\path{\R\R\R\R\T\S\S\L\L\L\L\L\S\S}|\kern47pt
  \sm{\path{\R\R\R\R\T\S\S\L\L\L\L\L\S\S}}\cr
\noalign{\kern16pt}
|\def\X{\L\T\L\L\T\L\L\T} \path{\X\X\X\X}|\kern20pt
  \sm{\def\X{\L\T\L\L\T\L\L\T} \path{\X\X\X\X}}\cr
\noalign{\kern10pt}
%|\path{\S\S\S\S\L\R\R\L\S\R\R\L\R\R\L\S\S\S\L\R\R\L\R\R}|\kern20pt
%  \sm{\path{\S\S\S\S\L\R\R\L\S\R\R\L\R\R\L\S\S\S\L\R\R\L\R\R}}\cr
\enddisplay
Furthermore, there are operations |\B| and |\W| that make the path
black (visible) and white (invisible), respectively:
\begindisplay \openup1pt
\noalign{\kern14pt}
|\path{\R\R\S|\cr
|  \W\S\S\S\R\R|\cr
|  \B\R\R\S\R\S\R\S\S\S\R\S\S\S\S\S\R\S\R|
 \kern100pt\raise4pt\hbox{\sm{\path{\R\R\S
  \W\S\S\S\R\R
  \B\R\R\S\R\S\R\S\S\S\R\S\S\S\S\S\R\S\R
  \W\R\R\R\S\L\S
  \B\L\S\S\S\S}}}\cr
|  \W\R\R\R\S\L\S|\cr
|  \B\L\S\S\S\S}|\cr
\noalign{\kern15pt}
\enddisplay
(It may be necessary to put kerns before and after the path, since the
box produced by |\path| may not be as wide as the actual path itself.)

The |\path| macros work differently from |\point|, since the boxes need
not have zero width in this application:
\beginlines
^|\catcode||`\ =9 |^|\endlinechar||=-1 % ignore all spaces (temporarily)|
|\newcount\dir \newdimen\y \newdimen\w|
|\newif\ifvisible \let\B=\visibletrue \let\W=\visiblefalse|
|\newbox\NE \newbox\NW \newbox\SE \newbox\SW \newbox\NS \newbox\EW|
|\setbox\SW=\hbox{\qc a} \setbox\NW=\hbox{\qc b}|
|\setbox\NE=\hbox{\qc c} \setbox\SE=\hbox{\qc d}|
|\w=\wd\SW \dimen0=\fontdimen8\qc|
|\setbox\EW=\hbox{\kern-\dp\SW \vrule height\dimen0 width\wd\SW} \wd\EW=\w|
|\setbox\NS=\hbox{\vrule height\ht\SW depth\dp\SW width\dimen0}  \wd\NS=\w|
|\def\L{\ifcase|%
  |\dir \dy+\NW \or\dx-\SW \or\dy-\SE \or\dx+\NE\dd-4\fi \dd+1}|\kern-2pt
|\def\S{\ifcase\dir \dx+\EW \or \dy+\NS \or \dx-\EW \or \dy-\NS \fi}|
|\def\R{\ifcase|%
  |\dir \dy-\SW\dd+4 \or\dx+\SE \or\dy+\NE \or\dx-\NW\fi \dd-1}|\kern-2pt
|\def\T{|^|\ifcase|%
  |\dir\kern-\w\dd+2\or\ey-\dd+2\or\kern\w\dd-2\or\ey+\dd-2\fi}|\kern-2pt
\goodbreak
|\edef\dd#1#2{\global\advance\dir#1#2\space}|
|\def\dx#1#2|%
  |{\ifvisible\raise\y\copy#2 \if#1-\kern-2\w\fi\else\kern#1\w\fi}|\kern-2pt
|\def\dy#1#2{\ifvisible\raise\y\copy#2 \kern-\w \fi \global\advance\y#1\w}|
|\def\ey#1{\global\advance\y#1\w}|
|\def\path#1{\hbox{\B \dir=0 \y=0pt #1}}|
|\catcode`\ =10 |^|\endlinechar||=`\^^M % resume normal spacing conventions|
\smallskip
|\newcount\n % the current order in the \dragon and \nogard macros|
|\def\dragon{\ifnum\n>0{\advance\n-1 \dragon\L\nogard}\fi}|
|\def\nogard{\ifnum\n>0{\advance\n-1 \dragon\R\nogard}\fi}|
\endlines
(The last three lines are not part of the |\path| macros, but they can
be used as an interesting test case. To get the famous ``^{dragon curve}''
of order~9, all you have to say is `|\path{\dir=3 \n=9 \dragon}|'.)^^{recursion}
\endgroup

\smallbreak
Let's turn now to another box-oriented problem. The ^|\listing| macro
discussed earlier in this appendix was restricted to listing files
that contain only visible ASCII characters. Sometimes it's desirable to
deal with ASCII ^\<tab> marks too, where a \<tab> is equivalent to
1~or~2~or $\cdots$ or~8 spaces (whatever is necessary to make the
current line length a multiple of~8). How can this be done?

We shall assume that files can contain a special symbol that \TeX\ will
input as character number~9, the ASCII \<tab> code; some implementations
can't actually do this. If a file contains the three symbols |^^I|, plain
\TeX\ will normally input them as a single character, number~9; but in a
verbatim listing of the file we naturally want such symbols to print as
themselves, i.e., as |^^I|.

The following construction redefines |\setupverbatim| so that the previous
|\listing| macro will work with \<tab> characters. The idea is to keep the
line-so-far in an hbox, which can be ``measured'' in order to find out how
many characters have appeared since the beginning of the line or since the
most recent \<tab>.
\begintt
\def\setupverbatim{\tt \lineno=0
  \def\par{\leavevmode\egroup\box0\endgraf}
  \obeylines \uncatcodespecials \obeyspaces
  \catcode`\`=\active \catcode`\^^I=\active
  \everypar{\advance\lineno by1
    \llap{\sevenrm\the\lineno\ \ }\startbox}}
\newdimen\w \setbox0=\hbox{\tt\space} \w=8\wd0 % tab amount
\def\startbox{\setbox0=\hbox\bgroup}
{\catcode`\^^I=\active
  \gdef^^I{\leavevmode\egroup
    \dimen0=\wd0 % the width so far, or since the previous tab
    \divide\dimen0 by\w
    \multiply\dimen0 by\w % compute previous multiple of \w
    \advance\dimen0 by\w  % advance to next multiple of \w
    \wd0=\dimen0 \box0 \startbox}}
\endtt
^^|\divide|^^|\multiply|
(The new things in |\setupverbatim| are the `|\egroup\box0|' in the
redefinition of\/ |\par|; the `|\catcode`\^^I=\active|'; and the
`|\startbox|' in |\everypar|.) \ The |\settabs| and~|\+| macros of
Appendix~B provide another example of how tab operations can be simulated
by boxing and unboxing.

\smallbreak
\def\beginvrulealign{\setbox0=\vbox\bgroup}
\def\endvrulealign{\egroup % now \box0 holds the alignment}
  \setbox0=\vbox{\setbox2=\hbox{\vrule height\ht0 depth\dp0 width0pt}
    \unvbox0 \setbox0=\lastbox % now \box0 is the bottom row
    \nointerlineskip \copy0 % put it back
    \global\setbox1=\hbox{}
    \setbox4=\hbox{\unhbox0
      \loop \skip0=\lastskip \unskip % remove tabskip glue
       \advance\skip0 by-.4pt % rules are .4pt wide
       \divide\skip0 by 2
       \global\setbox1=\hbox{\hskip\skip0\vrule\hskip\skip0
         \unhbox2\unhbox1}%
       \setbox2=\lastbox % remove alignment entry
       \ifhbox2 \setbox2=\hbox{\kern\wd2}\repeat}}%
  \hbox{\rlap{\box0}\box1}}
\setbox8=\vbox{
\beginvrulealign
\tabskip=10pt
\halign{&\strut#\hfil\cr
These&    after\cr
vertical& the\cr
rules&    alignment\cr
were&     was\cr
inserted& completed!\cr}
\endvrulealign
}
Chapter 22 explains how to put vertical ^{rules in tables} by considering the
rules to be separate columns. There's also another way, provided that
the rules extend all the way from the top of the table to the bottom.
For example,
\begindisplay
|\beginvrulealign|\cr
|\tabskip=10pt|\cr
|\halign{&\strut#\hfil\cr|\cr
|These&    after\cr|\cr
|vertical& the\cr|\kern70pt yields\qquad\smash{$\vcenter{\box8}$}\cr
|rules&    alignment\cr|\cr
|were&     was\cr|\cr
|inserted& completed!\cr}|\cr
|\endvrulealign|\cr
\enddisplay
The magic macros in this case examine the bottom row of the
^{alignment}, which consists of alternating ^{tabskip glue} and boxes;
each item of tabskip glue in that bottom row will be bisected by
a vertical rule. Here's how:
\beginlines
|\def\beginvrulealign{\setbox0=\vbox\bgroup}|
|\def\endvrulealign{\egroup % now \box0 holds the entire alignment|
|  \setbox0=\vbox{\setbox2=\hbox{\vrule height\ht0 depth\dp0 width0pt}|
|    |^|\unvbox||0 \setbox0=\lastbox % now \box0 is the bottom row|
|    \nointerlineskip \copy0 % put it back|
|    \global\setbox1=\hbox{} % initialize box that will contain rules|
|    \setbox4=\hbox{\unhbox0 % now open up the bottom row|
|      \loop \skip0=|^|\lastskip|| |^|\unskip|| % remove tabskip glue|
|       \advance\skip0 by-.4pt % rules are .4pt wide|
|       \divide\skip0 by 2|
|       \global\setbox1=\hbox{\hskip\skip0\vrule\hskip\skip0|
|         \unhbox2\unhbox1}%|
|       \setbox2=|^|\lastbox|| % remove alignment entry|
|       \ifhbox2 \setbox2=\hbox{\kern\wd2}\repeat}}%|
|  \hbox{\rlap{\box0}\box1}} % superimpose the alignment on the rules|
\endlines
This method works with all alignments created by ^|\halign||{...}|. For
alignments created by, say, |\halign to100pt{...}|, the method works
only if the bottom row of the alignment contains all of the columns,
and only if `|\box1|' is replaced by `|\hbox to100pt{\unhbox1}|'
at the end of\/ |\endvrulealign|.

%\begingroup
%\eightpoint
%\hyphenpenalty10000 \exhyphenpenalty10000 \pretolerance10000 % no hyphens
%\newbox\dbox \setbox\dbox=\hbox to .4em{\hss.\hss} % dot box for leaders
%\newskip\rrskipb \rrskipb=.5em plus3em % ragged right space before break
%\newskip\rrskipa \rrskipa=-.17em plus-3em minus.11em % ditto, after
%\newskip\rlskipa \rlskipa=0pt plus3em % ragged left space after break
%\newskip\rlskipb \rlskipb=.33em plus-3em minus .11em % ditto, before
%\newskip\lskip \lskip=3.3\wd\dbox plus1fil minus.3\wd\dbox % for leaders
%\newskip\lskipa \lskipa=-2.67em plus-3em minus.11em % after leaders
%\mathchardef\rlpen=1000 \mathchardef\leadpen=600 % constants used
%\def\rrspace{\nobreak\hskip\rrskipb\penalty0\hskip\rrskipa}
%\def\rlspace{\penalty\rlpen\hskip\rlskipb\vadjust{}\nobreak\hskip\rlskipa}
%\uccode`~=` \uppercase{
%  \def\:{\nobreak\hskip\rrskipb \penalty\leadpen \hskip\rrskipa
%    \vadjust{}\nobreak\leaders\copy\dbox\hskip\lskip
%    \kern3em \penalty\leadpen \hskip\lskipa
%    \vadjust{}\nobreak\hskip\rlskipa \let~=\rlspace}
%  \everypar{\hangindent=1.5em \hangafter=1 \let~=\rrspace}}
%\parindent=0pt
%\parfillskip=0pt
%\obeyspaces
%\def\text{ACM Symposium on Theory of Computing, Eighth Annual (Hershey, %
%Pa., 1976)\:1879, 4813, 5414, 6918, 6936, 6937, 6946, 6951, %
%6970, 7619, 9605, 10148, 11676, 11687, 11692, 11710, 13869}
%\catcode`\ =10
%\def\test#1 {\vbox{\hsize=#1em\hrule\strut\text\strut\par\hrule}}
%\global\setbox5=\test 12.5
%\global\setbox7=\test 17.5
%%\global\setbox9=\test 22.5
%\global\setbox9=\test 30
%\endgroup
%\subsection Paragraph maneuvers. Chapter 14 promised that Appendix D would
%present an example where ^{ragged right} and ^{ragged left} setting
%occur in the same paragraph. The following interesting example was
%suggested by the ``^{Key Index}'' in {\sl ^{Mathematical Reviews}},
%^^{index macros} where the entries consist of a possibly long title
%followed by dot ^{leaders} followed by a possibly long list of
%review numbers. If the title doesn't fit on one line, it should be set
%ragged right, with hanging indentation on all lines after the first;
%if the references don't all fit on one line, they should be set
%ragged left. For example, given the input
%\begintt
%ACM Symposium on Theory of Computing, Eighth Annual (Hershey, %
%Pa., 1976)\:1879, 4813, 5414, 6918, 6936, 6937, 6946, 6951, %
%6970, 7619, 9605, 10148, 11676, 11687, 11692, 11710, 13869
%\endtt
%the following three types of output are desired, depending on the
%column width:
%\begindisplay
%\box5&\box7\cr
%\noalign{\vskip6pt}
%\box9\span\cr
%\enddisplay
%Notice that the dot leaders are treated in three different ways, depending
%on which works out best: They may occur at the left of the first line after
%the title, or they may appear at the end of the last line of the title
%(in which case they stop well before the right margin), or they may occur
%in the middle of a line. Furthermore, the ragged-right lines are supposed
%to end at least $0.5\em$ from the right margin. Our goal is to achieve
%all this as a special case of \TeX's general paragraphing method. The
%simple approach of Appendix~B won't work, because |\raggedright| is
%achieved there by adjusting ^|\rightskip|; \TeX\ uses the same |\rightskip|
%value in all lines of a paragraph.
\begingroup
\eightpoint
\hyphenpenalty10000 \exhyphenpenalty10000 \pretolerance10000 % no hyphens
\newbox\dbox \setbox\dbox=\hbox to .4em{\hss.\hss} % dot box for leaders
\newskip\rrskipb \rrskipb=.5em plus3em % ragged right space before break
\newskip\rrskipa \rrskipa=-.17em plus-3em minus.11em % ditto, after
\newskip\rlskipa \rlskipa=0pt plus3em % ragged left space after break
\newskip\rlskipb \rlskipb=.33em plus-3em minus .11em % ditto, before
\newskip\lskip \lskip=3.3\wd\dbox plus1fil minus.3\wd\dbox % for leaders
\newskip\lskipa \lskipa=-2.67em plus-3em minus.11em % after leaders
\mathchardef\rlpen=1000 \mathchardef\leadpen=600 % constants used
\def\rrspace{\nobreak\hskip\rrskipb\penalty0\hskip\rrskipa}
\def\rlspace{\penalty\rlpen\hskip\rlskipb\vadjust{}\nobreak\hskip\rlskipa}
\uccode`~=` \uppercase{
  \def\:{\nobreak\hskip\rrskipb \penalty\leadpen \hskip\rrskipa
    \vadjust{}\nobreak\leaders\copy\dbox\hskip\lskip
    \kern3em \penalty\leadpen \hskip\lskipa
    \vadjust{}\nobreak\hskip\rlskipa \let~=\rlspace}
  \everypar{\hangindent=1.5em \hangafter=1 \let~=\rrspace}}
\parindent=0pt
\parfillskip=0pt
\obeyspaces
\def\text{ACM Symposium on Theory of Computing, Eighth Annual (Hershey, %
Pa., 1976)\:1879, 4813, 5414, 6918, 6936, 6937, 6946, 6951, %
6970, 7619, 9605, 10148, 11676, 11687, 11692, 11710, 13869}
\catcode`\ =10
\def\test#1 {\vbox{\hsize=#1em\hrule\strut\text\strut\par\hrule}}
\global\setbox5=\test 12.5
\global\setbox7=\test 17.5
%\global\setbox9=\test 22.5
\global\setbox9=\test 30
\endgroup
\subsection 操纵段落. Chapter 14 promised that Appendix D would
present an example where ^{ragged right} and ^{ragged left} setting
occur in the same paragraph. The following interesting example was
suggested by the ``^{Key Index}'' in {\sl ^{Mathematical Reviews}},
^^{index macros} where the entries consist of a possibly long title
followed by dot ^{leaders} followed by a possibly long list of
review numbers. If the title doesn't fit on one line, it should be set
ragged right, with hanging indentation on all lines after the first;
if the references don't all fit on one line, they should be set
ragged left. For example, given the input
\begintt
ACM Symposium on Theory of Computing, Eighth Annual (Hershey, %
Pa., 1976)\:1879, 4813, 5414, 6918, 6936, 6937, 6946, 6951, %
6970, 7619, 9605, 10148, 11676, 11687, 11692, 11710, 13869
\endtt
the following three types of output are desired, depending on the
column width:
\begindisplay
\box5&\box7\cr
\noalign{\vskip6pt}
\box9\span\cr
\enddisplay
Notice that the dot leaders are treated in three different ways, depending
on which works out best: They may occur at the left of the first line after
the title, or they may appear at the end of the last line of the title
(in which case they stop well before the right margin), or they may occur
in the middle of a line. Furthermore, the ragged-right lines are supposed
to end at least $0.5\em$ from the right margin. Our goal is to achieve
all this as a special case of \TeX's general paragraphing method. The
simple approach of Appendix~B won't work, because |\raggedright| is
achieved there by adjusting ^|\rightskip|; \TeX\ uses the same |\rightskip|
value in all lines of a paragraph.

The solution to this problem requires an understanding of the ^{line-breaking}
algorithm; it depends on how demerits are calculated, and on how items
are removed at the breakpoints, so the reader should review Chapter~14
until those concepts are firmly understood. Basically, we need to
specify a sequence of box/glue/penalty items for the spaces in the
title portion, another sequence for the spaces in the reference portion,
and another sequence for the dot leaders. In the title portion of each
index entry, interword spaces can be represented by the sequence
\begintt
\penalty10000 \hskip.5em plus3em \penalty0
\hskip-.17em plus-3em minus.11em
\endtt
Thus, there is a stretchability of $3\em$ if a line break occurs
at the |\penalty0|; otherwise the net interword space will be $.33\em$,
shrinkable to $.22\em$. This gives ragged right margins. The interword
spaces in the reference portion are designed to produce ragged left margins
and to minimize the number of lines devoted to references:
\begintt
\penalty1000 \hskip.33em plus-3em minus.11em
\vadjust{}\penalty10000 \hskip0pt plus3em
\endtt
The ^|\vadjust||{}| does nothing, but it doesn't disappear at a line break.
Thus, if a break occurs at the |\penalty1000|, the following line will
begin with stretchability $3\em$; but if no break occurs, the net space
will be $.33\em$ minus $.11\em$. Finally, the transition between title
and references can be specified by
\begintt
\penalty10000 \hskip.5em plus3em \penalty600
\hskip-.17em plus-3em minus.11em
\vadjust{}\penalty10000
\leaders\copy\dbox\hskip3.3\wd\dbox plus1fil minus.3\wd\dbox
\kern3em \penalty600 \hskip-2.67em plus-3em minus.11em
\vadjust{}\penalty10000 \hskip0pt plus3em
\endtt
(Quite a mouthful.) \
This long sequence of penalty and glue items begins rather like the interword
spaces in the first part, and it ends rather like the interword spaces in
the last part. It has two permissible breakpoints, namely at the
`|\penalty600|' items. The first breakpoint causes the leaders to
appear at the beginning of a line; the second causes them to appear
at the end, but 3~ems away. The leader width will always be at least
three times the width of\/ |\dbox|, so at least two copies of\/ |\dbox|
will always appear. Here is the actual \TeX\ code that can be used
to set up the desired behavior:
\beginlines
|\hyphenpenalty10000 \exhyphenpenalty10000 |^|\pretolerance||10000 % no hyphens|
^|\newbox||\dbox \setbox\dbox=\hbox to .4em{\hss.\hss} % dot box for leaders|
^|\newskip||\rrskipb \rrskipb=.5em plus3em % ragged right space before break|
|\newskip\rrskipa \rrskipa=-.17em plus-3em minus.11em % ditto, after|
|\newskip\rlskipa \rlskipa=0pt plus3em % ragged left space after break|
|\newskip\rlskipb \rlskipb=.33em plus-3em minus .11em % ditto, before|
|\newskip\lskip \lskip=3.3\wd\dbox plus1fil minus.3\wd\dbox % for leaders|
|\newskip\lskipa \lskipa=-2.67em plus-3em minus.11em % after leaders|
^|\mathchardef||\rlpen=1000 \mathchardef\leadpen=600 % constants used|
|\def\rrspace{|^|\nobreak||\hskip\rrskipb\penalty0\hskip\rrskipa}|
|\def\rlspace{|%
  |\penalty\rlpen\hskip\rlskipb\vadjust{}\nobreak\hskip\rlskipa}|\kern-2pt
|\uccode`~=` \uppercase{|
|  \def\:{\nobreak\hskip\rrskipb \penalty\leadpen \hskip\rrskipa|
|    \vadjust{}\nobreak\leaders\copy\dbox\hskip\lskip|
|    \kern3em \penalty\leadpen \hskip\lskipa|
|    \vadjust{}\nobreak\hskip\rlskipa \let~=\rlspace}|
|  \everypar{\hangindent=1.5em \hangafter=1 \let~=\rrspace}}|
|\uccode`~=0 |^|\parindent||=0pt |^|\parfillskip||=0pt |^|\obeyspaces|
\endlines
Putting the interword glue into |\skip| registers saves a great deal
of time and memory space when \TeX\ works with such paragraphs;
`|\hskip|\<explicit glue>' occupies six cells of \TeX's ^{box memory},
but `|\hskip|\<skip register>' occupies only two. ^^{efficiency}
Notice the tricky use of\/ ^|\uppercase| here to convert |~|$_{13}$ into
\]$_{13}$; ^^{active spaces} ``random'' ^{active characters} can be
obtained in a similar way.

\medbreak
Let's turn now to a much simpler problem: {\sl^{hanging punctuation}}.
\medskip
\setbox0=\vbox{\vfil
\hyphenpenalty=10000
\eightpoint
\hsize=14pc
\newdimen\commahang \setbox0=\hbox{,} \commahang=\wd0
\newdimen\periodhang \setbox0=\hbox{.} \periodhang=\wd0
\newdimen\quotehang \setbox0=\hbox{`} \quotehang=\wd0
\newdimen\qquotehang \setbox0=\hbox{``} \qquotehang=\wd0
\newskip\zzz \def\allowhyphens{\nobreak\hskip\zzz}
\def\lqq{``} \def\rqq{''} \def\pnt{.}
\def\comma{,\kern-\commahang\kern\commahang}
\def\period{.\kern-\periodhang\kern\periodhang}
\def\rquote{'\kern-\quotehang\kern\quotehang}
\def\lquote{\ifhmode\kern\quotehang\vadjust{}\else\leavevmode\fi
  \kern-\quotehang`\allowhyphens}
\catcode`,=\active \let,=\comma
\catcode`.=\active \let.=\period
\catcode`'=\active \def'{\futurelet\next\rqtest}
\catcode``=\active \def`{\futurelet\next\lqtest}
\def\rqtest{\ifx\next'\let\next=\rquotes\else\let\next=\rquote\fi\next}
\def\lqtest{\ifx\next`\let\next=\lquotes\else\let\next=\lquote\fi\next}
\def\rquotes'{\rqq\kern-\qquotehang\kern\qquotehang}
\def\lquotes`{\ifhmode\kern\qquotehang\vadjust{}\else\leavevmode\fi
  \kern-\qquotehang\lqq\allowhyphens}
\parindent=0pt
``What is hanging punctuation?'' asked Alice, with a puzzled frown.
`Well, y'know, actually,' answered Bill, `I'd rather demonstrate it than
explain it.' ``Oh, now I see.  Commas, periods, and quotes are allowed to
stick out into the margins, if they occur next to a line break.'' `Yeah, I
guess.' ``Really! But why do all your remarks have single quotes, while
mine are double?'' `I haven't the foggiest; it's weird. Ask the author of
this crazy book.'
}\setbox2=\vsplit0 to.5\ht0 \line{\box2\hfil\box0}
\medskip\noindent
Each comma in Alice and Bill's demonstration paragraph
was represented inside of \TeX\ by
the sequence of three items `|,\kern-\commahang|^|\kern||\commahang|',
and there were similar replacements for periods and for closing quotes;
opening quotes were represented by the longer sequence
\begintt
\kern\qquotehang\vadjust{}\kern-\qquotehang``\allowhyphens
\endtt
where ^|\allowhyphens| allows the following word to be ^{hyphenate}d.
This construction works because kerns disappear into ^{line breaks} in the
proper way; the relevant rules from Chapter~14 are: (1)~A line break can
occur at a kern that is immediately followed by glue. (2)~Consecutive
glue, kern, and penalty items disappear at a break.

To set \TeX\ up for hanging punctuation, you can say
\beginlines
|\newdimen\commahang      \setbox0=\hbox{,}      \commahang=\wd0|
|\newdimen\periodhang     \setbox0=\hbox{.}      \periodhang=\wd0|
|\newdimen\quotehang      \setbox0=\hbox{`}      \quotehang=\wd0|
|\newdimen\qquotehang     \setbox0=\hbox{``}     \qquotehang=\wd0|
|\newskip\zzz \def\allowhyphens{\nobreak\hskip\zzz}|
|\def\lqq{``} \def\rqq{''} \def\pnt{.}|
|\def\comma{,\kern-\commahang\kern\commahang}|
|\def\period{.\kern-\periodhang\kern\periodhang}|
|\def\rquote{'\kern-\quotehang\kern\quotehang}|
|\def\lquote{\ifhmode\kern\quotehang\vadjust{}\else\leavevmode\fi|
|  \kern-\quotehang`\allowhyphens}|
|\catcode`,=\active \let,=\comma \catcode`.=\active \let.=\period|
|\catcode`'=\active \def'{\futurelet\next\rqtest}|
|\catcode``=\active \def`{\futurelet\next\lqtest}|
|\def\rqtest{\ifx\next'\let\next=\rquotes\else\let\next=\rquote\fi\next}|
|\def\lqtest{\ifx\next`\let\next=\lquotes\else\let\next=\lquote\fi\next}|
|\def\rquotes'{\rqq\kern-\qquotehang\kern\qquotehang}|
|\def\lquotes`{\ifhmode\kern\qquotehang\vadjust{}\else\leavevmode\fi|
|  \kern-\qquotehang\lqq\allowhyphens}|
\endlines
Notice that the macros need to do their own checking for ligatures, and
they also take appropriate actions when a paragraph begins with an opening
quote.  Since |\kern| does not affect the ^{space factor}, hanging
punctuation doesn't affect \TeX's spacing conventions within a line.
Partially hanging punctuation can be obtained by decreasing the amounts of\/
|\commahang|, etc.  The macros ^|\pnt|, ^|\lq|, and ^|\rq| should be used
in constants; for example, a dimension of $6.5\,$in must be written
`|6\pnt5in|' when hanging punctuation is in effect, and
`|\catcode\lq,=12|' makes commas inactive again. A special font with
zero-width ^|\hyphenchar| should be used for ``hanging hyphenation.''

%\medbreak
%And now for our next trick, let's consider an application to short footnotes.
%^^{footnotes, short} The footnotes at the bottom of this page%
%\footnote{$^1$}{First footnote.}$^,$%
%\footnote{$^2$}{Second footnote. (Every once in a~while a long
%  footnote might occur, just to make things difficult.)}$^,$%
%\footnote{$^3$}{Third footnote.}$^,$%
%\footnote{$^4$}{Fourth footnote.}$^,$%
%\footnote{$^5$}{Fifth footnote. (This is incredibly boring,
%  but it's just an example.)}$^,$%
%\footnote{$^6$}{Another.}$^,$%
%\footnote{$^7$}{And another.}$^,$%
%\footnote{$^8$}{Ho hum.}$^,$%
%\footnote{$^9$}{Umpteenth footnote.}$^,$%
%\footnote{$^{10}$}{Oodles of them.}
%look funny, because most of them are quite short. When a document has lots
%of footnotes, and when most of them take up only a small part of a line,
%the output routine ought to reformat them in some more appropriate way.
\medbreak
And now for our next trick, let's consider an application to short footnotes.
^^{footnotes, short} The footnotes at the bottom of this page%
\footnote{$^1$}{First footnote.}$^,$%
\footnote{$^2$}{Second footnote. (Every once in a~while a long
  footnote might occur, just to make things difficult.)}$^,$%
\footnote{$^3$}{Third footnote.}$^,$%
\footnote{$^4$}{Fourth footnote.}$^,$%
\footnote{$^5$}{Fifth footnote. (This is incredibly boring,
  but it's just an example.)}$^,$%
\footnote{$^6$}{Another.}$^,$%
\footnote{$^7$}{And another.}$^,$%
\footnote{$^8$}{Ho hum.}$^,$%
\footnote{$^9$}{Umpteenth footnote.}$^,$%
\footnote{$^{10}$}{Oodles of them.}
look funny, because most of them are quite short. When a document has lots
of footnotes, and when most of them take up only a small part of a line,
the output routine ought to reformat them in some more appropriate way.
\vfill\supereject

\newcount\footno
For example, one approach would be to typeset the footnotes in narrow
columns and to put, say, three columns of footnotes at the bottom of
each page. The ten example footnotes might then look like this:
\medskip
\begingroup
\def\vfootnote#1{\insert\footins{
    \eightpoint \hsize=9pc \parindent=1pc
    \leftskip=0pt \raggedright \pretolerance=10000
    \interlinepenalty=\interfootnotelinepenalty
    \floatingpenalty=20000
    \splittopskip=\ht\strutbox \splitmaxdepth=\dp\strutbox
    \global\advance\footno by 1
    \item{$^{\the\footno}$}\strut#1\strut\par\allowbreak}}
\output={\showboxbreadth=9999 \showboxdepth=1
  %\nonstopmode \showbox\footins \errorstopmode
  \rigidbalance\footins 3 7pt
  \global\setbox1=\lastbox
  \unvbox255}
\footno=0
\vfootnote{First footnote.}
\vfootnote{Second footnote. (Every once in a~while a long
  footnote might occur, just to make things difficult.)}
\vfootnote{Third footnote.}
\vfootnote{Fourth footnote.}
\vfootnote{Fifth footnote.
  (This is incredibly boring, but it's just an example.)}
\vfootnote{Another.}
\vfootnote{And another.}
\vfootnote{Ho hum.}
\vfootnote{Umpteenth footnote.}
\vfootnote{Oodles of them.}
\newcount\k \newdimen\h
\def\rigidbalance#1#2 #3 {\setbox0=\box#1 \k=#2 \h=#3
  \line{\splittopskip=\h \vbadness=10000 \hfilneg
  \valign{##\vfil\cr\dosplits}}}
\def\dosplits{\ifnum\k>0 \noalign{\hfil}\splitoff
  \global\advance\k-1\cr\dosplits\fi}
\def\splitoff{\dimen0=\ht0 \divide\dimen0 by\k \advance\dimen0 by\h
  \vsplit0 to \dimen0 }
\eject
\box1\endgroup
\medskip\noindent
In this case, the footnotes could be generated by
\begindisplay
|\insert\footins{\eightpoint \hsize=9pc \parindent=1pc|\cr
|  \leftskip=0pt |^|\raggedright|| \pretolerance=10000|\cr
|  \hyphenpenalty=10000 \exhyphenpenalty=10000|\cr
|  \interlinepenalty=\interfootnotelinepenalty|\cr
|  \floatingpenalty=20000|\cr
|  \splittopskip=\ht\strutbox \splitmaxdepth=\dp\strutbox|\cr
|  \item{$^{\the\footno}$}\strut|\<text of footnote>^|\strut|\cr
|  \par|^|\allowbreak||}|\cr
\enddisplay
and |\count\footins| would be set to 333 so that each footnote line
would be considered to occupy about one third of a line on the page.
The output routine would then see a |\box\footins| that looks like
this:
\begindisplay
|\vbox(142.0+2.0)x108.0|\cr
|.\hbox(7.0+2.0)x108.0, glue set 42.23425fil []|\cr
|.\penalty 0|\cr
|.\hbox(7.0+2.0)x108.0, glue set 0.29266 []|\cr
|.\penalty 250|\cr
|.\glue(\baselineskip) 1.44444|\cr
|.\hbox(5.55556+1.55556)x96.0, glue set 0.8693, shifted 12.0 []|\cr
|.\penalty 100|\cr
|.\glue(\baselineskip) 1.88889|\cr
|.\hbox(5.55556+1.55556)x96.0, glue set 0.92438, shifted 12.0 []|\cr
\qquad\vdots\cr
|.\hbox(7.0+2.0)x108.0, glue set 18.56308fil []|\cr
|.\penalty 0|\cr
|.\hbox(7.0+2.0)x108.0, glue set 36.92476fil []|\cr
|.\penalty 0|\cr
\enddisplay
The individual footnotes each end with `|\penalty 0|'; footnotes that
take up more than one line have larger penalties between the lines,
and interline glue appears there too.

How should the output routine break such a box up into three roughly
equal pieces? Notice that the contents of the box are completely rigid,
i.e, there is no glue that can stretch or shrink. Furthermore, we can
assume that the contents of the box are regular, i.e., that the
inter-baseline distances are all the same. In such circumstances
a fairly simple ^{balancing} routine can be used to trisect the box.

Let's consider a more general problem: Suppose that a rigid vbox is given,
$n$~lines tall, where adjacent baselines are $b$~units apart. Suppose
also that the top baseline is $h$~units from the top of the
vbox, where $0<h<b$. \ (In our footnote example, $b=9\pt$ and $h=7\pt$;
in the standard settings of plain \TeX, $b=12\pt$ and $h=10\pt$. We might
as well work the problem for general $b$ and $h$.) \ It follows that
the height of the vbox is $H=h+b(n-1)=bn+h-b$.

If $n$ lines are to be distributed evenly into $k$ columns, the first
column should contain $\lceil n/k\rceil$ lines. \ (This denotes the
smallest integer greater than or equal to $n/k$.) \ For example,
our application to footnotes has $n=16$ and $k=3$, hence the first
column should contain 6~lines. After forming the first column,
we have reduced the problem to $n=10$ and $k=2$, so two 5-line
columns will complete the operation. \ (Notice that it is better
to divide 16 into $6+5+5$ instead of $6+6+4$.) \ Once we have
found the first column, it's always possible to reduce the $k$-column
problem to a $(k-1)$-column problem, so we need only concentrate
on finding the first column.

Let $m=\lceil n/k\rceil$. The height of the given box is $bn+h-b$,
and the height of the first column should be $bm+h-b$; hence
we want to do a ^|\vsplit| to that height. However, it isn't
necessary to calculate $bm+h-b$ exactly, since a bit of
arithmetic proves that
\begindisplay
$\displaystyle bm+h-b\;<\;{bn+h-b\over k}+h\;<\;b(m+1)+h-b.$
\enddisplay
Therefore it suffices to |\vsplit| to height $H'=H/k+h$; under the
assumptions of rigidity, and assuming that a valid break is possible
after each line, |\vsplit| to $H'$ will split after the maximum
number of lines that yield a box of height $\le H'$. \ (We have
observed that $m$~lines produce a box of height~$<H'$ while
$m+1$~lines produce a box of height~$>H'$.) \ The following \TeX\ code
does this:
\beginlines
|\newcount\k \newdimen\h % registers used by the \rigidbalance routine|
|\def\rigidbalance#1#2 #3 {\setbox0=\box#1 \k=#2 \h=#3|
|  \line{\splittopskip=\h \vbadness=10000 \hfilneg|
|    \valign{##\vfil\cr\dosplits}}}|
|\def\dosplits{\ifnum\k>0 \noalign{\hfil}\splitoff|
|  \global\advance\k-1\cr\dosplits\fi}|
|\def\splitoff{\dimen0=\ht0|
|  |^|\divide||\dimen0 by\k \advance\dimen0 by\h|
|  \vsplit0 to \dimen0 }|
\endlines
This code is interesting on a number of counts. First, notice that
the calculation does not depend on~$b$, only on~$h$ and the height of
the given box; hence |\rigidbalance| has three parameters: a box register
number, the number of columns~$k$, and the top baseline height~$h$.
The routine splits the given vbox into $k$~nearly equal pieces and
justifies the result in a |\line|. The value of\/ ^|\splittopskip| is
set to~$h$ so that subsequent vboxes will satisfy the ground rules of the
original vbox, as the problem is reduced from~$k$ to~$k-1$. Each column will
be preceded by ^|\hfil|, hence ^|\hfilneg| is used to cancel the |\hfil|
before the first column. A ^|\valign| is used to align all of the columns
at the top. Notice that the preamble to this |\valign| is
quite simple; and the body of the |\valign| is generated by a ^{recursive
macro} |\dosplits| that produces the $k$~columns. The value of\/
^|\vbadness| is set to~10000 because each |\vsplit| operation
will produce an ^{underfull} vbox whose badness is 10000.

In our application to footnotes, the |\output| routine can reformat
the contents of\/ |\box\footins| by saying, for example,
\begintt
\rigidbalance\footins 3 7pt
\setbox\footins=\lastbox
\endtt
since ^|\lastbox| will be the result of\/ |\rigidbalance|.

%This solution to the problem of short footnotes might result in
%^{club lines} or ^{widow lines}, since the balancing routine we have
%described simply trisects the total number of lines. For example,
%if the tenth footnote of our example had not been present,
%the fifteen remaining lines would have been split $5+5+5$;
%the second column would have been headed by the lonely
%word `{\eightrm difficult.)}', and the third column would have
%started with `{\eightrm just an example.)}'. The rigid balancing
%procedure could be replaced by one that allows ^{ragged-bottom}
%columns, but there's also another approach: The entire set of footnotes
%could be combined into a single paragraph, with generous spacing
%between the individual items. For example, the ten footnotes we
%have been considering might appear as follows:
This solution to the problem of short footnotes might result in
^{club lines} or ^{widow lines}, since the balancing routine we have
described simply trisects the total number of lines. For example,
if the tenth footnote of our example had not been present,
the fifteen remaining lines would have been split $5+5+5$;
the second column would have been headed by the lonely
word `{\eightrm difficult.)}', and the third column would have
started with `{\eightrm just an example.)}'. The rigid balancing
procedure could be replaced by one that allows ^{ragged-bottom}
columns, but there's also another approach: The entire set of footnotes
could be combined into a single paragraph, with generous spacing
between the individual items. For example, the ten footnotes we
have been considering might appear as follows:
%\medskip
%\begingroup
%{\catcode`p=12 \catcode`t=12 \gdef\\#1pt{#1}}
%\let\getfactor=\\
%\eightpoint
%\newskip\footglue \footglue=1.5em plus.3em minus.3em
%\newdimen\footnotebaselineskip \footnotebaselineskip=10pt
%\dimen0=\footnotebaselineskip \multiply\dimen0 by 1024
%\divide \dimen0 by \hsize \multiply\dimen0 by 64
%\xdef\fudgefactor{\expandafter\getfactor\the\dimen0 }
%\def\vfootnote#1{\insert\footins{\floatingpenalty=20000
%  \global\advance\footno by 1
%  \eightpoint \setbox0=\hbox{%
%    $^{\the\footno}$#1\penalty-10\hskip\footglue}
%  \dp0=0pt \ht0=\fudgefactor\wd0 \box0}}
%\output={\showboxbreadth=9999 \showboxdepth=1
%  \nonstopmode \showbox\footins \errorstopmode
%  \global\setbox1=\vbox{\makefootnoteparagraph}
%  \unvbox255}
%\footno=0
%\vfootnote{First footnote.}
%\vfootnote{Second footnote. (Every once in a~while a long
%  footnote might occur, just to make things difficult.)}
%\vfootnote{Third footnote.}
%\vfootnote{Fourth footnote.}
%\vfootnote{Fifth footnote.
%  (This is incredibly boring, but it's just an example.)}
%\vfootnote{Another.}
%\vfootnote{And another.}
%\vfootnote{Ho hum.}
%\vfootnote{Umpteenth footnote.}
%\vfootnote{Oodles of them.}
%%\def\makefootnoteparagraph{\unvbox\footins \makehboxofhboxes
%% \showboxbreadth=9999 \showboxdepth=1
%% \nonstopmode \showbox0 \errorstopmode
%%  \setbox0=\hbox{\unhbox0 \removehboxes}
%% \showboxbreadth=30
%% \nonstopmode \showbox0 \errorstopmode
%%  \baselineskip=10pt\noindent\unhbox0 }
%%\def\makehboxofhboxes{\setbox0=\hbox{}
%%  \loop\setbox2=\lastbox
%%  \ifhbox2 \setbox0=\hbox{\box2\unhbox0}\repeat}
%%\def\removehboxes{\setbox0=\lastbox
%%  \ifhbox0{\removehboxes}\unhbox0 \fi}
%\def\makefootnoteparagraph{\unvbox\footins
% \baselineskip=\footnotebaselineskip \removehboxes}
%\def\removehboxes{\unskip\setbox0=\lastbox
% \ifhbox0{\removehboxes}\unhbox0 \else\noindent \fi}
%\eject
%\box1\endgroup
%\medskip
\medskip
\begingroup
{\catcode`p=12 \catcode`t=12 \gdef\\#1pt{#1}}
\let\getfactor=\\
\eightpoint
\newskip\footglue \footglue=1.5em plus.3em minus.3em
\newdimen\footnotebaselineskip \footnotebaselineskip=10pt
\dimen0=\footnotebaselineskip \multiply\dimen0 by 1024
\divide \dimen0 by \hsize \multiply\dimen0 by 64
\xdef\fudgefactor{\expandafter\getfactor\the\dimen0 }
\def\vfootnote#1{\insert\footins{\floatingpenalty=20000
  \global\advance\footno by 1
  \eightpoint \setbox0=\hbox{%
    $^{\the\footno}$#1\penalty-10\hskip\footglue}
  \dp0=0pt \ht0=\fudgefactor\wd0 \box0}}
\output={\showboxbreadth=9999 \showboxdepth=1
  %\nonstopmode \showbox\footins \errorstopmode
  \global\setbox1=\vbox{\makefootnoteparagraph}
  \unvbox255}
\footno=0
\vfootnote{First footnote.}
\vfootnote{Second footnote. (Every once in a~while a long
  footnote might occur, just to make things difficult.)}
\vfootnote{Third footnote.}
\vfootnote{Fourth footnote.}
\vfootnote{Fifth footnote.
  (This is incredibly boring, but it's just an example.)}
\vfootnote{Another.}
\vfootnote{And another.}
\vfootnote{Ho hum.}
\vfootnote{Umpteenth footnote.}
\vfootnote{Oodles of them.}
\def\makefootnoteparagraph{\unvbox\footins \makehboxofhboxes
 \showboxbreadth=9999 \showboxdepth=1
 %\nonstopmode \showbox0 \errorstopmode
  \setbox0=\hbox{\unhbox0 \removehboxes}
 \showboxbreadth=30
 %\nonstopmode \showbox0 \errorstopmode
  \baselineskip=10pt\noindent\unhbox0 }
\def\makehboxofhboxes{\setbox0=\hbox{}
  \loop\setbox2=\lastbox
  \ifhbox2 \setbox0=\hbox{\box2\unhbox0}\repeat}
\def\removehboxes{\setbox0=\lastbox
  \ifhbox0{\removehboxes}\unhbox0 \fi}
\eject
\box1\endgroup
\medskip

It would be possible to take the contents of\/ |\box\footins| shown
previously and to reformat everything into a paragraph, but such an operation
would be needlessly complicated. If footnotes are to be paragraphed by the
output routine, it's better simply to prepare them in unjustified
hboxes. Each of these hboxes will be unboxed later, so we are free to play
with their heights, widths, and depths. It's convenient to set the
depth to zero and the height to an estimate of how much a particular
footnote will contribute to the final paragraph. For example, if
a footnote takes up exactly half of the |\hsize|, and if the final
footnote is going to be set with |\baselineskip=10pt|, then the
height of the footnote hbox should be set to $5\pt$. By letting
|\count\footins=1000|, we'll have a pretty good estimate of the size
of the final footnote paragraph. In other words, the following
insertion scheme is suggested:
\begindisplay
|\insert\footins{\floatingpenalty=20000|\cr
|  \eightpoint \setbox0=\hbox{%|\cr
|    $^{\the\footno}$|\<text of footnote>|\penalty-10\hskip\footglue}|\cr
|  \dp0=0pt \ht0=\fudgefactor\wd0 \box0}|\cr
\enddisplay
The penalty of $-10$ tends to favor line breaks between footnotes;
|\footglue| is the amount of glue between footnotes in the final
footnote paragraph; and |\fudgefactor| is the ratio of\/ |\baselineskip|
to |\hsize| in that paragraph. The author defined the necessary quantities
as follows in his experiments:
\begintt
\eightpoint \newskip\footglue \footglue=1.5em plus.3em minus.3em
\newdimen\footnotebaselineskip \footnotebaselineskip=10pt
\dimen0=\footnotebaselineskip \multiply\dimen0 by 1024
\divide \dimen0 by \hsize \multiply\dimen0 by 64
\xdef\fudgefactor{\expandafter\getfactor\the\dimen0 }
\endtt
^^|\multiply| ^^|\divide| ^^|sp|
(The computation of\/ |\fudgefactor| uses the fact that $1\pt=1024\times64\,$sp,
and it assumes that the |\footnotebaselineskip| is less than $16\pt$.)

Inside the output routine, |\box\footins| will now be a vbox of hboxes, and the
height of this vbox will be an estimate of the height of the final paragraph.
For example, our ten footnotes produce
\begintt
\vbox(34.48158+0.0)x386.4221
.\hbox(2.00175+0.0)x70.68285 []
.\hbox(10.94359+0.0)x386.4221 []
.\hbox(2.09749+0.0)x74.06345 []
.\hbox(2.2077+0.0)x77.95517 []
.\hbox(7.6296+0.0)x269.40376 []
.\hbox(1.40851+0.0)x49.73532 []
.\hbox(1.87659+0.0)x66.26334 []
.\hbox(1.38826+0.0)x49.02003 []
.\hbox(2.67213+0.0)x94.35402 []
.\hbox(2.25597+0.0)x79.65926 []
\endtt
and the height of $34.48158\pt$ corresponds to an estimate of about
three and a~half lines. \ (\TeX's page builder has also added |\skip\footins|
when estimating the total contribution due to footnotes.)

The reformatting of\/ |\box\footins| takes
place in three stages. First the vbox of hboxes is changed to an hbox
of hboxes, so that we obtain, e.g.,
\begindisplay
|\hbox(10.94359+0.0)x1217.5593|\cr
|.\hbox(2.00175+0.0)x70.68285 []|\cr
\qquad\vdots\cr
|.\hbox(2.25597+0.0)x79.65926 []|\cr
\enddisplay
(the same contents as before, but strung in a horizontal row instead of
a vertical column). Then the inner hboxes are unboxed, and we obtain
\begintt
\hbox(6.68999+2.0)x1217.5593
.\mathon
.\hbox(3.86665+0.0)x4.16661, shifted -2.82333 []
.\mathoff
.\eightrm F
.etc.
\endtt
Finally the outer hbox is unboxed, and the horizontal list inside it
is converted into a paragraph. Here is the actual \TeX\ code:
\beginlines
|\def\makefootnoteparagraph{\unvbox\footins \makehboxofhboxes|
|  \setbox0=\hbox{\unhbox0 \removehboxes}|
|  \baselineskip=\footnotebaselineskip\noindent\unhbox0\par}|
|\def\makehboxofhboxes{\setbox0=\hbox{}|
|  \loop\setbox2=\lastbox \ifhbox2 \setbox0=\hbox{\box2\unhbox0}\repeat}|
|\def\removehboxes{\setbox0=\lastbox|
|  \ifhbox0{\removehboxes}\unhbox0 \fi}|
\endlines
^^|\unhbox|^^|\unvbox|
The |\removehboxes| operation is especially noteworthy, because it uses
\TeX's ^{save stack} to hold all of the hboxes before unboxing them. Each
level of recursion in this routine uses one cell of input stack space and
three cells of save stack space; thus, it is generally safe to do more than
100 footnotes without exceeding \TeX's capacity. The |\makehboxofhboxes|
routine is not as efficient; \TeX\ doesn't allow a vbox to be unboxed
in horizontal mode, or vice versa, hence the trick of\/ |\removehboxes|
cannot be used. This means that the running time is proportional to~$n^2$,
if there are $n$~footnotes, because the time to make or unmake a box is
proportional to the number of items in the top-level list inside.
However, the constant of proportionality is small, so there is no
need to resort to a more complicated scheme that would be asymptotically faster.
Indeed, the |\lastbox| operation itself has a running time approximately
equal to $a+mb$, where $m$~is the number of items ^^{efficiency} on the
list preceding the box that is removed; hence |\removehboxes| has a
running time of order $n^2$ as well. But the constant~$b$ is so small that
for practical purposes it's possible to think of\/ |\lastbox| as almost
instantaneous. Note, however, that it would be a mistake to bypass the
|\removehboxes| operation by saying `|\setbox0=\hbox{\unhbox2\unhbox0}|'
in |\makehboxofhboxes|; that would make the top-level list inside |\box0| too
long for efficient unboxing.

%\subsection Communication with output routines. It would be possible to
%write an entire book about \TeX\ output routines; but the present
%appendix is already too long, so it will suffice to mention only one
%or two sneaky tricks that a person might not readily think~of. \ (Appendix~E
%gives some less sneaky examples.)
\subsection 与输出例程交流. It would be possible to
write an entire book about \TeX\ output routines; but the present
appendix is already too long, so it will suffice to mention only one
or two sneaky tricks that a person might not readily think~of. \ (Appendix~E
gives some less sneaky examples.)

Sometimes an output routine needs to know why it was invoked, so there's
a problem of communicating information from the rest of the program.
\TeX\ provides general |\mark| operations, but marks don't always yield
the right sorts of clues. Then there's ^|\outputpenalty|, which can be
tested to see what penalty occurred at a breakpoint; any penalty of
$-10000$, $-10001$, $-10002$, or less,
forces the output routine to act, hence different penalty values
can be used to pass different messages. \ (When the output routine puts
material back on the list of contributions, it need not restore the
penalty at the breakpoint.) \ If output has been forced by a
highly negative value of\/ |\outputpenalty|, the output routine can
use |\vbox{\unvcopy255}| to discover how full the page-so-far actually is.
Underfull and overfull boxes are not reported when |\box255| is
packaged for use by the output routine, so there's no harm in ejecting
a page prematurely if you want to pass a signal. \ (Set
^|\holdinginserts| positive to pass a signal when the contents of\/
|\box255| will be sent back through the page builder again, if any
insertions are present.)

Perhaps the dirtiest trick of all is to communicate with the output
routine via the {\sl depth\/} of\/ |\box255|. For example, suppose that
you want to know whether or not the current page ends with the last
line of a paragraph. If each paragraph ends with `|\specialstrut|', where
|\specialstrut| is like ^|\strut| but $1\,$sp deeper, then |\dp255|
will have a recognizable value if a page ends simultaneously
with a paragraph. \ (Of course, ^|\maxdepth| must be suitably large; plain
\TeX\ takes |\maxdepth=4pt|, while struts are normally $3.5\pt$ deep, so
there's no problem.) \ A distance of~$1000\,$sp is invisible to the
naked eye, so a variety of messages can be passed in this way.

If the value of\/ ^|\vsize| is very small, \TeX\ will construct paragraphs
as usual but it will send them to the output routine one line at a time.
In this way the output routine could attach marginal notes, etc., based
on what occurs in the line. Paragraphs that have been rebuilt in this
way can also be sent back from the output routine to the page builder;
normal page breaks will then be found, if\/ |\vsize| has been restored.

An output routine can also write notes on a file, based on what occurs in
a manuscript. A two-pass system can be devised where \TeX\ simply gathers
information during the first pass; the actual typesetting can be done during
the second pass, using |\read| to recover information that was written
during the first.

%\subsection Syntax checking. Suppose you want to run a manuscript through
%\TeX\ just to check for errors, without getting any output. Is there a
%way to make \TeX\ run significantly faster while doing this? Yes; here's how:
%(1)~Say `|\font|^|\dummy||=dummy|'\kern1pt; your system should include a
%file |dummy.tfm| that defines a font with no characters (but with
%enough |\fontdimen| parameters to qualify as a math symbol font).
%(2)~Set all the font identifiers you are using equal to |\dummy|. For
%example, |\let\tenrm=\dummy|, |\let\tenbf=\dummy|, \dots, |\textfont0=\dummy|,
%etc. (3)~Say `|\dummy|' to select the dummy font (since plain \TeX\ may
%have selected the real |\tenrm|).  (4)~Set ^|\tracinglostchars||=0|, so that
%\TeX\ won't complain when characters aren't present in the dummy font. (5)~Set
%\begintt
%\output={\setbox0=\box255\deadcycles=0}
%\endtt
%^^|\deadcycles|
%so that nothing will be shipped out, yet \TeX\ will not think that your
%output routine is flaky. (6)~Say ^|\newtoks||\output|, so that no other
%output routine will be defined. (7)~Say ^|\frenchspacing| so that \TeX\
%will not have to do space factor calculations. (8)~Say
%^|\hbadness||=10000| so that underfull boxes will not be reported.
%(9)~And if you want to disable |\write| commands, use the following
%trick due to Frank ^{Yellin}:
%\begintt
%\let\immediate=\relax \def\write#1#{{\afterassignment}\toks0=}
%\endtt
%These changes usually make \TeX\ run more than four times as fast.
%^^|\afterassignment|
\subsection 语法检查. Suppose you want to run a manuscript through
\TeX\ just to check for errors, without getting any output. Is there a
way to make \TeX\ run significantly faster while doing this? Yes; here's how:
(1)~Say `|\font|^|\dummy||=dummy|'\kern1pt; your system should include a
file |dummy.tfm| that defines a font with no characters (but with
enough |\fontdimen| parameters to qualify as a math symbol font).
(2)~Set all the font identifiers you are using equal to |\dummy|. For
example, |\let\tenrm=\dummy|, |\let\tenbf=\dummy|, \dots, |\textfont0=\dummy|,
etc. (3)~Say `|\dummy|' to select the dummy font (since plain \TeX\ may
have selected the real |\tenrm|).  (4)~Set ^|\tracinglostchars||=0|, so that
\TeX\ won't complain when characters aren't present in the dummy font. (5)~Set
\begintt
\output={\setbox0=\box255\deadcycles=0}
\endtt
^^|\deadcycles|
so that nothing will be shipped out, yet \TeX\ will not think that your
output routine is flaky. (6)~Say ^|\newtoks||\output|, so that no other
output routine will be defined. (7)~Say ^|\frenchspacing| so that \TeX\
will not have to do space factor calculations. (8)~Say
^|\hbadness||=10000| so that underfull boxes will not be reported.
(9)~And if you want to disable |\write| commands, use the following
trick due to Frank ^{Yellin}:
\begintt
\let\immediate=\relax \def\write#1#{{\afterassignment}\toks0=}
\endtt
These changes usually make \TeX\ run more than four times as fast.
^^|\afterassignment|

\endchapter

^{Wolfe}, who had moved around the desk and into his chair,
^^{Hombert, Humbert} put up a palm at him: ``Please, Mr.~Hombert.
I think it is always advisable to take a short-cut when it is feasible.\rlap{''}
\author REX ^{STOUT}, {\sl The Rubber Band\/} (1936)

\bigskip

``My dear ^{Watson}, try a little analysis yourself,\rlap{''}
said he, ^^{Holmes} with a touch of impatience.
``You know my methods. Apply them,
and it will be instructive to compare results.\rlap{''}
\author CONAN ^{DOYLE}, {\sl The Sign of the Four\/} (1890)

%\eject
\eject\byebye
