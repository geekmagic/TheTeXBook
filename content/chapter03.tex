% -*- coding: utf-8 -*-

\input macros

%\beginchapter Chapter 3. Controlling\\\TeX
\beginchapter Chapter 3. 控制系列

\origpageno=7

%Your keyboard has very few keys compared to the large number of symbols that you
%may want to specify. In order to make a limited keyboard sufficiently versatile,
%one of the characters that you can type is reserved for special use, and
%it is called the {\sl ^{escape character}}. Whenever you want to type
%something that controls the format of your manuscript, or something that
%doesn't use the keyboard in the ordinary way, you should type the escape
%character followed by an indication of what you want to do.
\1你的键盘所含的字符与所期望得到的符号相比太少了。%
为了充分利用有限的键盘,你可以键入的某个字符被保留为特殊用途,即所谓%
的转义符(escape cha\-racter)。%
只要你要键入控制文稿格式的信息或不能直接使用键盘的内容,
就应该键入转义符,再在后面跟上对应的指令。

%Note: Some computer terminals have a key marked `|ESC|', but that is {\sl not\/}
%your escape character! It is a key that sends a special message to the operating
%system, so don't confuse it with what this manual calls ``escape.''
注意:有些计算机终端上有一个标记为`|ESC|'的键,
但不是你的转义符!
它是一个向操作系统发送特殊指令的键,所以不要把它与本手册的``转义(escape)''混淆。

%\TeX\ allows any character to be used for escapes, but the ``^{backslash}''
%character `|\|' is usually adopted for this purpose, since backslashes are
%reasonably convenient to type and they are rarely needed in ordinary text.
%Things work out best when different \TeX\ users do things consistently,
%so we shall escape via backslashes in all the examples of this manual.
\TeX\ 允许使用任何字符作为转义符,但是一般采用反斜线字符`|\|',
因为反斜线较方便键入且在一般文本中很少使用。%
当不同的 \TeX\ 用户统一规则时,做事就最有效,
所以我们将在本手册的所有例子中使用反斜线字符作为转义符。

%Immediately after typing `|\|' (i.e., immediately after an escape
%character) you type a coded command telling \TeX\ what you have in mind.
%Such commands are called {\sl ^{control sequences}}. For example, you might type
%^^{markup commands, see control sequences}
%\begintt
%\input MS
%\endtt
%which (as we will see later) causes \TeX\ to begin reading a file called
%`|MS.tex|'; the string of characters `^|\input|' is a control sequence.
%Here's another example:
%\begintt
%George P\'olya and Gabor Szeg\"o.
%\endtt
%% sic; this is the spelling used in opening pages of their famous books
%% but I give the Hungarian spellings in the index
%\TeX\ converts this to `George P\'olya and Gabor Szeg\"o.' There are two
%^^{Polya}^^{Szego}^^{acute}^^{umlaut}
%control sequences, ^|\'| and ^|\"|, here; these control sequences
%have been used to place ^{accents} over some of the letters.
键入`|\|'后,紧接(即在转义符后面紧接)着键入一个代码命令来告诉 \TeX\ 你的想法。%
这样的命令称为{\KT{10}控制系列}。%
例如,你可以键入
\begintt
\input MS
\endtt
(就象我们后面将讲述的那样)它让 \TeX\ 开始读入名字为`|MS.tex|'的文件;
字符串 `|\input|'是一个控制系列。%
另一个例子是
\begintt
George P\'olya and Gabor Szeg\"o.
\endtt
\TeX\ 把它转换为`George P\'olya and Gabor Szeg\"o.'。
这里有两个控制系列,|\'| 和 |\"|;
这些控制系列用来把重音号放在字母上。

%Control sequences come in two flavors. The first kind, like |\input|,
%is called a {\sl^{control word}\/}; it
%consists of an escape character followed by one or more {\sl letters}, followed
%by a space or by something besides a letter. \ (\TeX\ has to know where the
%control sequence ends, so you must put a space after a control word if
%the next character is a letter. For example, if you type `|\inputMS|',
%\TeX\ will naturally interpret this as a control word with seven
%letters.) \ In case you're wondering what a ``^{letter}'' is, the answer
%is that \TeX\ normally regards the 52 symbols |A...Z| and |a...z| as
%letters. The digits |0...9| are {\sl not\/} considered to be
%letters, so they don't appear in control sequences of the first kind.
控制系列有两种类型。%
第一类象|\input|一样,称为{\KT{10}控制词};
它由转义符后跟一个或多个{\KT{10}字母}组成,后面跟一个空格或非字母的字符。%
(\TeX\ 必须知道控制系列何时结束,因此,如果下一个字符是字母,你必须在控制词后%
加一空格。%
例如,如果你键入`|\inputMS|', \TeX\ 将认为它是一个7个字母组成的控制词。)
 要是对``字母''有疑问,答案就是,\TeX\ 一般把52个字母 |A...Z| 和 |a...z| 看作字母。%
数字 |0...9| 不被看作字母,所以不能出现在第一类控制系列中。

%A control sequence of the other kind, like |\'|, is called a {\sl
%^{control symbol}\/}; it consists of the escape
%character followed by a single {\sl nonletter}. In this case you don't
%need a space to separate the control sequence from a letter that follows,
%since control sequences of the second kind always have exactly one
%symbol after the escape character.
另一类控制系列,象 |\'| 一样,被称为{\KT{10}控制符号};
它由转义符后跟一个单个{\KT{10}非字母}组成。%
在这种情况下,你不必用空格把控制系列与其后的字母分开,
因为第二类控制系列在转义符后永远只有一个符号。

%\exercise What are the control sequences in `|\I'm \exercise3.1\\!|'\thinspace?
%\answer |\I|, |\exercise|, and |\\|. (The last of these is of type~2, i.e.,
%a control symbol, since the second backslash is not a letter; the first
%backslash keeps the second one from starting its own control sequence.)
\exercise 在`|\I'm \exercise3.1\\!|'中,控制系列有哪些?
\answer |\I|、|\exercise| 和 |\\| 。(最后一个属于第二种,即控制符号,
因为第二个反斜线不是一个字母;第一个反斜线阻止第二个反斜线开始自己的控制系列。)

%\exercise We've seen that the input |P\'olya| yields `P\'olya'. Can
%you guess how the French words `math\'ematique' and `centim\`etre'
%should be specified?
%\answer |math\'ematique| and |centim\`etre|.^^|\'|^^|\`|
\exercise \1我们已经知道键入 |P\'olya| 就得到`P\'olya'。%
你猜出来法语单词`math\'ematique' 和`centim\`etre'怎样键入吗?
\answer |math\'ematique| 和 |centim\`etre| 。^^|\'|^^|\`|

%When a space comes after a control word (an all-letter control
%sequence), it is ignored by
%\TeX; i.e., it is not considered to be a ``real'' space belonging to the
%manuscript that is being typeset. But when a space comes after a control
%symbol, it's truly a space.
当空格出现在控制词(一个完全由字母组成的控制系列)后面时,
它被 \TeX\ 所忽略;即,它不会被看作属于所排版文稿的``实际''空格。%
但是,当空格出现在控制符号后面时,它就是一个实际空格。

%Now the question arises, what do you do if you actually {\sl want\/} a
%space to appear after a control word? We will see later that \TeX\
%treats two or more consecutive spaces as a single ^{space}, so the answer
%is {\sl not\/} going to be ``type two spaces.'' The correct answer is to
%type ``control space,'' ^^|\ | namely
%\begintt
%\|]
%\endtt
%(the escape character followed by a blank space); \TeX\ will treat this as
%a space that is not to be ignored. Notice that |\|\] is a control
%sequence of the second kind, namely a control symbol, since there is a
%single nonletter (\]) following the escape character. Two consecutive
%spaces are considered to be equivalent to a single space, so further
%spaces immediately following |\|\] will be ignored. But if you want to
%enter, say, three consecutive spaces into a manuscript you can type
%`|\|\]|\|\]|\|\]'.  Incidentally, typists are often taught to put two
%spaces at the ends of sentences; but we will see later that \TeX\ has its
%own way to produce extra space in such cases. Thus you needn't be
%consistent in the number of spaces you type.
现在,出现问题了,如果真要在控制词后面得到一个空格该怎样做呢?
后面我们会看到,\TeX\ 把两个或多个连续空格看作单个空格,
所以``键入两个空格''不解决问题。
正确的方法是键入``控制空格'', 即
\begintt
\|]
\endtt
(转义符后面跟一个空格);
\TeX\ 将把它看作不能忽略的空格。%
注意,控制空格是第二类控制系列,即控制符号,因为跟在转义符后面的是一个单个非字母(\])。%
两个连续空格被看作等价于单个空格,所以跟在 |\|\] 后的多余的空格将被忽略。%
但是,如果你要在文稿中键入3个连续空格,就应该键入`|\|\]|\|\]|\|\]'。%
顺便说一下,打字员通常在句子后面要键入两个空格;
但是以后我们将知道,在这种情况下,\TeX\ 用自己的方式来得到额外的间距。
因此,你不需在意所键入的空格数目的一致性。

%\danger Nonprinting control characters like \<return> might follow
%an escape character, and these lead to distinct control sequences according
%to the rules. \TeX\ is initially set up to treat |\|\<return> and
%|\|\<tab> ^^|\<return>|^^|\<tab>|
%the same as |\|\] (control space); these special control sequences
%should probably not be redefined, because you can't see the difference
%between them when you look at them in a file.
%^^{carriage-return, see <return>}
\danger 象 \<return> 这样不能显示出来的控制符也可以跟在转义符后面,
并且按照上面的规则,它们是一些独特的控制系列。%
\TeX\ 从开始就把 |\|\<return> 和 |\|\<tab> 看作同 |\|\]~(控制空格)一样;
这些特殊的控制系列不可能重新被定义,因为在文件中无法区分它们。

%It is usually unnecessary for you to use ``control space,'' since control
%sequences aren't often needed at the ends of words. But here's an example
%that might shed some light on the matter: This manual itself has been
%typeset by \TeX, and one of the things that occurs fairly often is the
%tricky ^{logo} `\TeX', which requires backspacing and lowering the E.
%There's a special control word
%\begintt
%\TeX
%\endtt
%that produces the half-dozen or so instructions necessary to typeset `\TeX'.
%When a phrase like `\TeX\ ignores spaces after control words.' is
%desired, the manuscript renders it as follows:
%\begintt
%\TeX\ ignores spaces after control words.
%\endtt
%Notice the extra |\| following ^|\TeX|; this produces the control space
%that is necessary because \TeX\ ignores spaces after control words.
%Without this extra |\|, the result would have been
%\begindisplay
%\TeX ignores spaces after control words.
%\enddisplay
%On the other hand, you can't simply put |\| after |\TeX| in all contexts.
%For example, consider the phrase
%\begintt
%the logo `\TeX'.
%\endtt
%In this case an extra backslash doesn't work at all; in fact,
%you get a curious result if you type
%\begintt
%the logo `\TeX\'.
%\endtt
%Can you guess what happens? \ Answer: The |\'| is a control sequence denoting
%an acute accent, as in our |P\'olya| example above; the effect is
%therefore to put an accent over the next nonblank character,
%which happens to be a period. In other words, you get an accented
%period, and the result is
%\begindisplay
%the logo `\TeX\'.
%\enddisplay
%Computers are good at following instructions, but not at reading your mind.
对你而言,通常不需要使用``控制空格'',
因为控制系列不经常出现在单词的尾部。%
但是,这里有一个这方面的典型例子:
本手册本身是用 \TeX\ 排版的,并且相当频繁出现的一个东西就是乖巧的`\TeX',
它要求回退间距和降低 E。%
这是一个特殊的控制词
\begintt
\TeX
\endtt
为了排版`\TeX', 它使用了大概半打的指令。%
当要排版出`\TeX\ ignores spaces after control words.'\allowbreak 这样的语句时,
所键入的文稿如下:
\begintt
\TeX\ ignores spaces after control words.
\endtt
注意,\TeX\ 后面有一个额外的 |\|; 因为 \TeX\ 忽略掉了控制词后面的空格,
所以它生成了必需的控制空格。%
如果没有这个额外的 |\|, 结果就是
\begindisplay
\TeX ignores spaces after control words.
\enddisplay
\1另外,不能在任何情况下都在 \TeX\ 后面直接跟一个 |\|。%
比如,考虑一下下列语句
\begintt
the logo `\TeX'.
\endtt
在这种情况下,额外的 |\| 根本就不对;
实际上,如果你键入
\begintt
the logo `\TeX\'.
\endtt
就会得到稀奇古怪的结果。
你知道得到什么吗?
答案是:|\'| 是重音符的控制系列,就象上面例子中的 |P\'olya| 一样;
因此结果是在下一个非空白的字符——凑巧是句点——上面加重音。%
换句话说,你得到一个带重音的句点,排版结果是
\begindisplay
the logo `\TeX\'.
\enddisplay
计算机正确地执行了后面的指令,但没有正确表达你的意思。

%\TeX\ understands about 900 control sequences as part of its built-in
%vocabulary, and all of them are explained in this manual somewhere. But
%you needn't worry about learning so many different things, because you won't
%really be needing very many of them unless you are faced with unusually
%complicated copy. Furthermore, the ones you do need to learn actually fall into
%relatively few categories, so they can be assimilated without great difficulty.
%For example, many of the control sequences are simply the names of special
%characters used in math formulas; you type `^|\pi|'~to get~`$\pi$',
%`^|\Pi|'~to get~`$\Pi$',
%`^|\aleph|'~to get~`$\aleph$',
%`^|\infty|'~to get~`$\infty$',
%`^|\le|'~to get~`$\le$',
%`^|\ge|'~to get~`$\ge$',
%`^|\ne|'~to get~`$\ne$',
%`^|\oplus|'~to get~`$\oplus$',
%`^|\otimes|'~to get~`$\otimes$'.
%Appendix~F contains several tables of such symbols.
\TeX\ 能执行大约 900 个控制系列,这是它的内置语汇的一部分,
所有这些控制系列都阐述在本手册的某个地方。%
但是别为要掌握这么多不同的内容而担心,因为只要不是对付不常用的复杂排版,
你是不需要掌握很多控制系列的。%
再者,你需要实际掌握的那些只在相对少的几个种类\hbox{中,} 所以它们不太难理解。%
例如,许多控制系列就是数学公式中特殊字符的名字;
键入`|\pi|'就得到 $\pi$,
`|\Pi|' 得到 `$\Pi$',
`|\aleph|' 得到 `$\aleph$',
`|\infty|' 得到 `$\infty$',
`|\le|' 得到 `$\le$',
`|\ge|' 得到 `$\ge$',
`|\ne|' 得到 `$\ne$',
`|\oplus|' 得到 `$\oplus$',
`|\otimes|' 得到 `$\otimes$'。%
附录 F 包含了这类符号的几个列表。

%\danger There's no built-in relationship between ^{uppercase} and ^{lowercase}
%letters in control sequence names. For example, `|\pi|' and `|\Pi|'
%and `|\PI|' and `|\pI|' are four different control words.
\danger 在控制系列的名称中,大写和小写字母之间没有内在的等价关系。%
例如,`|\pi|', `|\Pi|, `|\PI|' 和 `|\pI|' 是四个不同的控制词。

%The 900 or so control sequences that were just mentioned actually aren't
%the whole story, because it's easy to define more.  For example, if you
%want to substitute your own favorite names for math symbols, so that you
%can remember them better, you're free to go right ahead and do it;
%Chapter~20 explains how.
刚刚提到的大约 900 个控制系列并不是全部,因为可以容易地定义更多的控制系列。%
例如,如果你想用自己喜欢的名字来定义数学符号,那么就更好记住它们,
你可以随心所欲地命名它们;
第 20 章告诉你这方面的内容。

%About 300 of \TeX's control sequences are called {\sl ^{primitive}\/}; these
%are the low-level atomic operations that are not decomposable into simpler
%functions. All other control sequences are defined, ultimately, in terms
%of the primitive ones. For example, ^|\input| is a primitive operation,
%but ^|\'| and ^|\"| are not; the latter are defined in terms of an
%^|\accent| primitive.
大约 300 个控制系列称为{\KT{10}原始控制系列};
它们是不能分解为更简单作用的低级控制系列。%
所有其它控制系列归根结底都由这些原始控制系列来定义。%
例如,|\input| 是原始控制系列,|\'| 和 |\"| 却不是;
后者用原始控制系列 |\accent| 来定义。

%People hardly ever use \TeX's primitive control sequences in their
%manuscripts, because the primitives are $\ldots$ well $\ldots$ so
%{\sl primitive}. You have to type a lot of instructions when you are
%trying to make \TeX\ do low-level things; this takes time and invites
%mistakes. It is generally better to make use of higher-level control
%sequences that state what functions are desired, instead of typing
%out the way to achieve each function each time. The higher-level control
%sequences need to be defined only once in terms of primitives. For
%example, |\TeX| is a control sequence that means ``typeset the \TeX\ logo'';
%|\'| is a control sequence that means ``put an acute accent over the
%next character''; and both of these control sequences might require different
%combinations of primitives when the style of type changes. If \TeX's logo
%were to change, the author would simply have to change one definition, and the
%changes would appear automatically wherever they were needed. By contrast,
%an enormous amount of work would be necessary to change the logo if it
%were specified as a sequence of primitives each time.
在文稿中,人们几乎用不到 \TeX\ 的原始控制系列,
因为原始控制系列是那么 $\ldots$ 地{\KT{10}原始}。%
如果你试图做 \TeX\ 的低级事情,就必需键入许多指令;
这浪费时间且容易犯错误。%
一般最好利用规定相应功能的高级控制系列,而不要每次都把得到本功能的%
所有东西都键入。%
高级控制系列只需要用原始控制系列定义一次。%
\1例如,|\TeX| 是``排版 \TeX\ 的标识符''的控制系列;
|\'| 是``在下一字符上放重音''的控制系列;
并且当排版格式变化时,所有这些控制系列只需原始控制系列的不同组合即可。%
如果 \TeX\ 的标识符要改变,作者只必须改变一个定义,并且改变后的东西自动出现%
在需要出现的地方。%
相反,如果它每次作为一系列的原始控制系列而确定时,改变标识符将要花费大量劳力。

%At a still higher level, there are control sequences that govern the
%overall format of a document. For example, in the present book the author
%typed `^|\exercise|' just before stating each exercise; this |\exercise|
%command was programmed to make \TeX\ do all of the following things:
在更高级方面,有管理文档整个格式的控制系列。%
例如,在前面作者在给出每个练习之前,键入了`|\exercise|';

%\nobreak\medskip
%\item\bull compute the exercise number (e.g., `3.2' for the second
%exercise in Chapter~3);
%\smallskip
%\item\bull typeset `\thinspace{\manual\char'170\rm\kern.15em
%  \ninebf EXERCISE \bf3.2}' with the appropriate typefaces, on a line by
%  itself, and with the triangle sticking out in the left margin;
%\smallskip
%\item\bull leave a little extra space just before that line, or begin
%  a new page at that line if appropriate;
%\smallskip
%\item\bull prohibit beginning a new page just after that line;
%\smallskip
%\item\bull suppress indentation on the following line.
%\medbreak\noindent
%It is obviously advantageous to avoid typing all of these individual
%instructions each time. And since the manual is entirely described in
%terms of high-level control sequences, it could be printed in a radically
%different format simply by changing a dozen or so definitions.
%% and sweating over the page layout in the math and alignment chapters!
\nobreak\medskip
\item\bull 算出练习的编号(比如,对第三章的第二个练习得到`3.2');
\smallskip
\item\bull 在同一行上,用相应的字体排版出`\thinspace{\manual\char'170\rm\kern.15em
  \ninebf EXERCISE \bf3.2}', 并且让三角形顶左边;
\smallskip
\item\bull 在行前留出额外小间距,如果需要的话在此行换页;
\smallskip
\item\bull 禁止在此行后换页;
\smallskip
\item\bull 保证接下来的行不缩进。
\medbreak\noindent
明显的好处是不必每次键入所有这些单个指令。%
并且因为本手册完全用高级控制系列所描述,所以通过改变大约一打的定义,
就可以用完全不同的格式来打印出它来。

%\danger How can a person distinguish a \TeX\ primitive from a control sequence
%that has been defined at a higher level? There are two ways: \ (1)~The index
%to this manual lists all of the control sequences that are discussed, and each
%primitive is marked with an asterisk. \ (2)~You can display the meaning of a
%control sequence while running \TeX\null. If you type `^|\show||\cs|'
%where |\cs| is any control sequence, \TeX\ will respond with its current
%meaning. For example, `|\show\input|' results in \hbox{`|> \input=\input.|'},
%because |\input| is primitive. On the other hand, `|\show|^|\thinspace|' yields
%\begintt
%> \thinspace=macro:
%->\kern .16667em .
%\endtt
%This means that |\thinspace| has been defined as an abbreviation for
%`|\kern|~|.16667em|~'. By typing `|\show|\penalty0|\kern|' you can verify
%that ^|\kern| is primitive. The results of\/ |\show| appear on your
%terminal and in the ^{log file} that you get after running \TeX.
\danger 怎样从高级控制系列中区分出 \TeX\ 的原始控制系列呢?
有两种方法:
(1)~本手册的索引罗列了讨论过的所有控制系列,每个原始控制系列都用星号标记了。
(2)~当运行 \TeX\ 时,你可以显示出某个控制系列的含义。
如果你键入 `|\show||\cs|'(其中 |\cs| 是任意控制系列),\TeX\ 将回答出它的当前含义。
例如,`|\show\input|' 得到 \hbox{`|> \input=\input.|'},
这是因为 |\input| 是原始控制系列。另一方面,`|show||\thinspace|' 得到
\begintt
> \thinspace=macro:
->\kern .16667em .
\endtt
这意味着 |\thinspace| 已经被定义为 `|\kern|~|.16667em|~'。
通过键入 `|\show|\penalty0|\kern|',你可以验证 |\kern| 是原始命令。
|\show| 的结果出现在你的终端和运行 \TeX\ 结束后的日志文件中。

%\dangerexercise Which of the control sequences |\|\] and
%|\|\<return> is primitive?
%\answer According to the index, |\|\] is primitive but
%|\|\<return> isn't. The command `|\def\^^M{\ }|' in
%Appendix~B is what actually defines |\|\<return>, since a
%return is representable as |^^M|. Asking \TeX\ to |\show\^^M|
%\looseness-1
%produces the response `|>| |\^^M=macro:->\|\]|.|'.
\dangerexercise 控制系列 |\|\] 和|\|\<return>中,哪个是原始的?
\answer 从索引中可以看到,|\|\] 是原始的,而 |\|\<return> 不是。
附录 B 中的 `|\def\^^M{\ }|' 定义了 |\|\<return>,因为回车符用 |^^M| 表示。
用 |\show\^^M| 可以让 \TeX\ 显示 `|>| |\^^M=macro:->\|\]|.|' 。

%In the following chapters we shall frequently discuss ``^{plain \TeX}''
%format, which is a set of about 600 ^{basic control sequences} that are
%defined in Appendix~B\null. These control sequences, together with the 300
%or so primitives, are usually present when \TeX\ begins to process a
%manuscript; that is why \TeX\ claims to know roughly 900 control sequences
%when it starts.  We shall see how plain \TeX\ can be used to create
%documents in a flexible format that meets many people's needs, using some
%typefaces that come with the \TeX\ system. However, you should keep in
%mind that plain \TeX\ is only one of countless ^{formats} that can be
%designed on top of \TeX's primitives; if you want some other format, it
%will usually be possible to adapt \TeX\ so that it will handle whatever
%you have in mind. The best way to learn is probably to start with plain
%\TeX\ and to change its definitions, little by little, as you gain more
%experience.
在后面的章节,我们将频繁讨论``plain \TeX''格式,
它大约有 600 个基本控制系列,这些都在附录 B 中定义出。
当 \TeX\ 开始处理文稿时,这些控制系列及 300 个左右的原始控制系列通常一起出现,
\1这就是为什么 \TeX\ 在一开始声称大概有 900 个控制系列。%
我们将看到 plain \TeX\ 怎样被用来得到灵活格式的文档以满足不同人们的需要,
不过要使用 \TeX\ 系统所带的一些字体。%
但是要记住,plain \TeX\ 只不过是在 \TeX\ 的原始控制系列上设计的无数格式的一种;
如果想要其它格式,通常可以改编 \TeX\ 以得到你心中想要的。%
领会的最好方法可能是从 plain \TeX\ 开始,随着经验的积累,一点一点地修改它的定义。

%\danger Appendix E contains examples of formats that can be added to
%Appendix~B for special applications; for example, there is a set of
%definitions suitable for business correspondence. A complete specification
%of the format used to typeset this manual also appears in Appendix~E\null.
%Thus, if your goal is to learn how to design \TeX\ formats, you will
%probably want to study Appendix~E while mastering Appendix~B\null. After you
%have become skilled in the lore of control-sequence definition, you
%will probably have developed some formats that other people will want
%to use; you should then write a supplement to this manual, explaining
%your style rules.
\danger 附录 E 中含有格式的例子,它们可以与附录 B 一起满足特殊要求;
例如,其中有一组适用于商业通信的定义。%
排版本书稿的完整的格式规定也在附录 E 中。%
因此,如果你的目的是设计 \TeX\ 格式,在学会附录 B 后你可能要研究研究附录 E。%
在经过定义控制系列的知识的训练后,你就可以为其他人们编写一些格式;
那么你应该为本手册写一个附录,解释一下你的设计规则。

%The main point of these remarks, as far as novice \TeX\ users are concerned, is
%that it is indeed possible to define nonstandard \TeX\ control sequences.
%When this manual says that something is part of ``plain \TeX,'' it means
%that \TeX\ doesn't insist on doing things exactly that way; a person
%could change the rules by changing one or more of the definitions in
%Appendix~B\null. But you can safely rely on the control sequences of plain
%\TeX\ until you become an experienced \TeX nical~typist.
就 \TeX\ 新手所关心的而言,这些讨论的重点是,的确可以定义一些非标准的 \TeX\ 控制系\hbox{列。}%
当 \TeX\ 讲什么属于``plain \TeX''时,就意味着 \TeX\ 不是只能按照这种方法才可;
你可以通过改变附录 B 中一个或多个定义来改变结果。%
但是在你成为有经验的 \TeX\ 排版专家前,确实可以依赖 plain \TeX\ 的控制系列。

%\ddangerexercise How many different control sequences of length~2
%(including the escape character) are possible? How many of length~3?
%\answer There are 256 of length~2; most of these are undefined when \TeX\
%begins. \ (\TeX\ allows any character to be an escape, but it does not
%distinguish between control sequences that start with different escape
%characters.) \
%If we assume that there are 52 letters, there are exactly $52^2$
%possible control sequences of length~3 (one for each pair of letters, from
%|AA| to |zz|). But Chapter~7 explains how to use ^|\catcode| to change any
%character into a ``^{letter}''; therefore it's possible to use any of
%$256^2$ potential control sequences of length~3.
\ddangerexercise 有多少 2 个字符的不同控制系列(转义符算一个字符)?
有多少 3 个字符的?
\answer 两个字符长的有 256 个;其中大部分在 \TeX\ 开始运行时是未定义的。%
(\TeX\ 允许任何字符作为转义符,但它不区分以不同转义符开始的控制系列。)
如果我们假设有 52 个字母,则总共有 $52^2$ 个可能的三个字符长的控制系列%
(从 |AA| 到 |zz|,一对字母对应一个控制系列)。
但第 7 章将介绍如何用 ^|\catcode| 将任何字符改为 ``^{字母}'';
因此我们可以使用 $256^2$ 个潜在的三字符长的控制系列中的任何一个。

\endchapter

Syllables govern the world.
\author JOHN ^{SELDEN}, {\sl Table Talk\/} (1689) % section on Power

\bigskip

I claim not to have controlled events,
but confess plainly that events have controlled me.
\author ABRAHAM ^{LINCOLN} (1864) % letter to A. G. Hodges, April 4

\vfill\eject\byebye
