% -*- coding: utf-8 -*-

\input macros

%% the table of contents
%\titlepage
%\vbox to 8pc{
%\rightline{\titlefont Contents}
%\vfill}
%^^{Contents of this manual, table}
%\def\rhead{Contents}
%\tenpoint
%\begingroup
%\countdef\counter=255
%\def\diamondleaders{\global\advance\counter by 1
%  \ifodd\counter \kern-10pt \fi
%  \leaders\hbox to 20pt{\ifodd\counter \kern13pt \else\kern3pt \fi
%    .\hss}}
%\baselineskip 15pt plus 5pt
%\def\\#1. #2. #3.{\line{\strut
%    \hbox to\parindent{\bf\hbox to 1em{\hss#1}\hss}%
%    \rm#2\diamondleaders\hfil\hbox to 2em{\hss#3}}}
%\\1. The Name of the Game. 1.
%\\2. Book Printing versus Ordinary Typing. 3.
%\\3. Controlling \TeX. 7.
%\\4. Fonts of Type. 13.
%\\5. Grouping. 19.
%\\6. Running \TeX. 23.
%\\7. How \TeX\ Reads What You Type. 37.
%\\8. The Characters You Type. 43.
%\\9. \TeX's Roman Fonts. 51.
%\\10. Dimensions. 57.
%\\11. Boxes. 63.
%\\12. Glue. 69.
%\\13. Modes. 85.
%\\14. How \TeX\ Breaks Paragraphs into Lines. 91.
%\\15. How \TeX\ Makes Lines into Pages. 109.
%\\16. Typing Math Formulas. 127.
%\\17. More about Math. 139.
%\\18. Fine Points of Mathematics Typing. 161.
%\\19. Displayed Equations. 185.
%\\20. Definitions (also called Macros). 199.
%\\21. Making Boxes. 221.
%\\22. Alignment. 231.
%\\23. Output Routines. 251.
%\eject
%\vbox to 8pc{}
%\\24. Summary of Vertical Mode. 267.
%\\25. Summary of Horizontal Mode. 285.
%\\26. Summary of Math Mode. 289.
%\\27. Recovery from Errors. 295.
%\null
%\leftline{\indent\bf Appendices}
%\\A. Answers to All the Exercises. 305.
%\\B. Basic Control Sequences. 339.
%\\C. Character Codes. 367.
%\\D. Dirty Tricks. 373.
%\\E. Example Formats. 403.
%\\F. Font Tables. 427.
%\\G. Generating Boxes from Formulas. 441.
%\\H. Hyphenation. 449.
%\\I\hskip 1pt. Index. 457.
%\\J\hskip 1pt. Joining the \TeX\ Community. 483.
%\null % 17 lines so far to balance the 23 on the other page
%\null % 18
%\null % 19
%\null % 20
%\null % 21
%\null % 22
%\null % 23
%\eject
%\endgroup

% the table of contents
\titlepage
\vbox to 8pc{
\rightline{\titlefont 目录}
\vfill}
\bookmark{1}{目录}
^^{Contents of this manual, table}
\def\rhead{目录}
\tenpoint
\begingroup
\countdef\counter=255
\def\diamondleaders{\global\advance\counter by 1
  \ifodd\counter \kern-10pt \fi
  \leaders\hbox to 20pt{\ifodd\counter \kern13pt \else\kern3pt \fi
    .\hss}}
\baselineskip 15pt plus 5pt
\def\\#1. #2. #3.{\line{\strut
    \hbox to\parindent{\bf\hbox to 1em{\hss#1}\hss}%
    \rm#2\diamondleaders\hfil\hbox to 2em{\hss#3}}}
\\1. 此名有诗意. 1.
\\2. 书籍排版与普通排版. 3.
\\3. 控制系列. 7.
\\4. 字体风格. 12.
\\5. 编组. 19.
\\6. 运行程序. 22.
\\7. 工作原理. 37.
\\8. 字符输入. 43.
\\9. 罗马字体. 51.
\\10. 尺寸. 57.
\\11. 盒子. 63.
\\12. 粘连. 68.
\\13. 模式. 85.
\\14. 分段为行. 91.
\\15. 组行为页. 109.
\\16. 数学公式. 128.
\\17. 数学排版进阶. 141.
\\18. 精致的数学排版. 165.
\\19. 陈列公式. 189.
\\20. 宏定义. 203.
\\21. 盒子制作. 227.
\\22. 对齐阵列. 236.
\\23. 输出例行程序. 258.
\eject
\vbox to 8pc{}
\\24. 垂直模式总结. 275.
\\25. 水平模式总结. 295.
\\26. 数学模式总结. 299.
\\27. 错误修复. 306.
\null
\leftline{\indent\bf 附录}
\\A. 练习答案. 318.
\\B. 基本控制系列. 357.
\\C. 字符编码. 391.
\\D. 诡计多端. 399.
\\E. 版式举例. 437.
\\F. 字体表. 469.
\\G. 公式的盒子. 489.
\\H. 连字算法. 499.
\\I. 索引. 509.
\\J. 加入社团. 535.

\null % 17 lines so far to balance the 23 on the other page
\null % 18
\null % 19
\null % 20
\null % 21
\null % 22
\null % 23
\eject
\endgroup
\byebye

